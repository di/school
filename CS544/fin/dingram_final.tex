\documentclass{article}
\usepackage{fullpage}
\usepackage{booktabs}
\usepackage{amsmath}
\usepackage{amssymb}
\usepackage[noend]{algorithmic}
\usepackage[nothing]{algorithm}
\usepackage{tikz}
\usepackage{latexsym}
\usepackage{float}
\usepackage{hyperref}
\usetikzlibrary{arrows,automata}
\providecommand{\e}[1]{\ensuremath{\times 10^{#1}}}
\renewcommand{\thealgorithm}{}
\renewcommand*{\thefootnote}{[\arabic{footnote}]}
\title{CS 544: Computer Networks \\ }
\author{Dustin Ingram}
\begin{document}
\maketitle
\section{Routing Tables \& Algorithms}
The three major columns of a routing table are:
\begin{enumerate}
    \item Destination;
    \item Which link to use;
    \item Cost of that route.
\end{enumerate}
There are three major types of routing algorithms:
\subsection{Static Routing Algorithms}
Static routing algorithms only have a single route to the destination. They may use shortest path routing, flooding, or flow-based routing to send data. They don't have as much need for a RT.
\subsection{Dynamic Routing Algorithms}
Dynamic routing algorithms attempt to be adaptive by having every network entity maintain a different routing table, and attempt to use the link with the lowest `cost' which is determined by the algorithm. There are two major types:
\begin{itemize}
    \item DVR -- Distance Vector Routing sents the entire RT at major network events to all one's neighbors, to recompute the RT. This reacts to positive link changes quickly (new routes), but not to negative ones (links down).
    \item LSR -- Link State Routing discovers neighbors, measures the delay to them, constructs a RT based on the information, sends this to all routers, and then computes a new RT. 
\end{itemize}
\subsection{Hierarchal Routing Algorithms}
Hierarchical routing algorithms choose routes based on the address of the message, dividing the network into hierarchies, which only certain parts must be considered to route to them. This means that the RT size is generally smaller, but the path lengths might be much longer, as messages have to ascend and descend the entire hierarchy (e.g., mailing a letter to your neighbor must go through the central post office for distribution).

If we must expand the goal of a routing algorithm to find 2 disjoint paths to every destination, we would now have to know which portions of the network are disjoint from other parts, where there are bottlenecks (single path, therefore disjoint multi-path would be impossible), etc. Distance Vector Routing would likely be better for the task, as it would have a better overview of the entire network.

\section{Routing and IPv4}
IPv4 uses ``hierarchical routing'' to determine a route from the source to the
destination. By breaking the source and destination into hierarchies, it allows
the autonomous system to route to a common hierarchy first and then deal with
more specific, lower level routing. There are three techniques to split the
address and represent the hierarchical addressing:
\begin{enumerate}
    \item Classful -- determined by address class of IPv4 address
    \item Subnetting -- Like classful, but further divide the host ID 
    \item Subnet Mask -- used to tell what portion of the address is also in
    the destination subnet
\end{enumerate}

\section{Encapsulation, Tunneling, and Translation}
\subsection{Encapsulation}
Encapsulation is putting data from a lower-level protocol into a container for a higher level protocol. In the ``railroad analogy'' this is like putting a car into a railcar, so that it can be transported on the railroad and not a paved road.
\subsection{Tunneling}
As with the previous analogy, tunneling is the act of transporting the car, via encapsulation, across the railroad.
\subsection{Translation}
Translation is the changing of data encapsulation from one protocol to another. For example, it is like taking a shipping container off of a rail car and putting it onto a cargo boat. Specifically with NAT, this translates packets destined for one source to another another IP address or port.
\subsection{Application}
It is possible to tunnel and encapsulate a message until we push the limits of message size, etc. However, translation can only occur on the most recently encapsulated layer, as it does not concern itself with the data which the message contains. 

\section{Contention and Collision}
Contention is two or more systems trying to use the same ``media'' to send
data. A collision occurs when these two systems do not resolve a contention and
send data simultaneously, resulting in garbage data. The two major categories of a shared MAC protocol are:
\begin{itemize}
    \item Static, where each ``station'' gets a certain amount of broadcast
    time or bandwidth;
    \item Dynamic, where each ``station'' actively bids for the use and amount
    of time that they can broadcast. Subcategories of Dynamic MACs include
    Scheduled and Random MACs.
\end{itemize}
Within a Static MAC protocol, there is never any contention, and therefore, no
collisions, as stations are specifically assigned a channel -- this is the
solution to this issue. In dynamic, there are two ways to deal with contention
and collision: either to use some form of scheduling algorithm to determine
which station can broadcast (scheduled), or randomly send when they do not
notice any other station broadcasting (random). Scheduled works well, similar
to static, as some algorithm is making sure there is no contention and no
collisions, however, there may be additional overhead. With a random MAC
protocol, however, there may potentially be collisions due to contention,
however with zero overhead. 

Each take a different approach to provide support where it is needed -- in most
cases, this might be some type of hybrid protocol that acts like multiple types
when they are needed. Contention might create issues with multiple stations,
requiring algorithms to oversee assignment, however it ensures that in a shared
MAC protocol each station is able to get some time to broadcast.

\section{Virtual Circuits and Datagrams}
\subsection{Virtual Circuits}
Virtual circuits are connection-oriented, where all data travels across the
same path.Upon setup of the connection, a single destination address is
provided. In a virtual circuit, sequencing is always in order, so higher-level
applications do not have to worry about segmentation and reassembly. Routing is
performed on a per-connection basis, which improves robustness and only causes
bandwidth waste when the routers need to be updated about the state of the
network. Flow control is possible across the network as data always takes the
same path. 

\subsection{Datagrams}
Datagrams are connectionless, where all data is independently routed within
what is typically known as a packet-switched network. Because packets may take
different paths through different routers, the destination address must be
included in each packet, and flow control is only possible at the source and
destination, resulting in some degree of wasted bandwidth in each message.
Unlike in virtual circuits, sequencing is not always in order because data may
arrive via different routes at different speeds.

\subsection{Flow}
A flow is a series of packets traveling in the same direction over a common
``connection'', which share information, and can identified by their:
\begin{itemize}
    \item source IP;
    \item source port;
    \item destination IP;
    \item destination port;
    \item protocol.
\end{itemize}
or by some other form of a ``flow label''. A flow is similar to a route in a
virtual circuit, but for a packet-switched network. A number of network
protocols may result in ``flows'' of data. An example flow might be a stream of
packets over HTTP for a streaming video service, from a server to the client.
Another example is the dual-flow that is created whenever a TCP connection is
created.

\section{Traffic Selectors}
A Traffic Selector allows a protocol to ``describe'' traffic, and to filter or
direct it. It does this via a series of ranges and wildcards which allow for a
description of various aspects of the traffic (protocol, port, IP, etc). This
can be  used for Quality of Service, routing, etc. \\
\textbf{Example:} IPv4 uses a form of traffic selectors to perform flow
bindings \\
\textbf{Example:} QoS uses traffic selectors to support end-to-end quality of
service over IP \\

\section{Goals of Security}
The seven goals of security are as follows:
\begin{enumerate}
    \item \textbf{Authentication \& Verification} \\
    This requires the validation and identification of a person, client,
    server, to prove that they are who they say they are or that information is
    valid. \\
    \textbf{Example:} SASL gives an authentication and security layer to other
    protocols, via a series of ``authentication mechanisms'' and
    challenge/response negotiation.
    \item \textbf{Access Control} \\
    This goal involves giving the information and resources to the correct
    people, restricting or allowing access as necessary. \\
    \textbf{Example:} MAC provides a media access control protocol, using
    addressing and channels.
    \item \textbf{Data Integrity} \\
    Used to prove that data has not been modified or changed. \\
    \textbf{Example:} SSH includes a MAC (Message Authentication Code, not to
    be confused with Media Access Control) in each message, which is a code
    produced via shared secret algorithms agreed upon by the client and server.
    \item \textbf{Confidentiality} \\
    Protecting information so that it is available only to allowed parties. \\
    \textbf{Example:} TLS/SSL provides transport- and application-layer
    security via cryptography for confidentiality of network communications.
    \item \textbf{Availability} \\
    Ensuring that a given resource will always be available to interested and
    allowed parties. \\
    \textbf{Example:} A reliable Persistence protocol, such as classical
    flooding or gossip-based algorithms.
    \item \textbf{Non-Repudiation} \\
    Allowing a protocol transaction to only occur the intended number of times.
    \\ 
    \textbf{Example:} PGP, specifically it's certificates and certificate
    authorities.
    \item \textbf{Trust} \\
    Establishing trust amongst clients/services using a protocol. \\
    \textbf{Example:} This requires a combination of previous security goals,
    and some notion of ``trust'' occurs in all aforementioned protocols.
\end{enumerate}
\end{document}
