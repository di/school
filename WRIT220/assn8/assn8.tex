\documentclass[letterpaper]{article}
\title{Week 9 Reviews}
\date{}
\usepackage{setspace}
\pdfpagewidth 8.5in
\pdfpageheight 11in
\begin{document}
    \maketitle
    %\thispagestyle{empty}
    \onehalfspacing
    \section{By Brie Jacob:}
    This piece is full witty, well-balanced lines, each of which really add to the story. For example,
    \begin{quote}
    ``The Apocalypse seemed a pleasant alternative to telling my mother that I'd given up on a future as a translator"
    \end{quote}
    The juxtaposition between these two things is great! The thought that the Apocalypse is \emph{pleasant} to think about gives exactly the perfect description of what you've avoiding doing. Similarly, like we've already discussed in class, the viewers/fans comparison is spot on.

    Your first paragraph is not a very good introduction to the piece. Without any prior knowledge of you, or your friend, we get lost in some details (e.g., why does it matter that she's in LA?) and miss a proper introduction--maybe explaining how you slowly became tired of school?
   
    Additionally, towards the end of the piece, you start to get into some details of the show itself (e.g., ``You can't just watch your friend die...'') which are completely lost on the reader, as they don't know what happened in the series like you did, and must infer what you're talking about. However, the wrap-up is great, you clearly show how there was a positive change over the course of the piece, and explained how you evaluated yourself, etc.

    The largest problem I had with this piece, is that while it is well written, it does not immediately show how it related to a larger social issue, although I can see how it could, as I believe that you are not the only person who's had this same (or similar) experience.

    \section{By Thea Olsen:}
    This piece has a strong thread running through it which holds it together: the combination of your personal experiences as well as your knowledge of the philosophy of suicide adds an interesting facet to an otherwise regular discussion of both issues. Your thoughts are intact, and together, they work well to make the individual piece, your voice clearly runs through it. I really enjoyed it, as well as the resulting (small) amount of reading on JSM that I did.

    The only issue I had was that each sentence seems to break down a little on it's own, either through lack of punctuation or disorganization. I imagine this piece was difficult to write and possibly read, but it seems as if it could have used a revision or two.

    My one additional suggestion is that while the piece involves personal experience, you should avoid becoming too personal, and deviating from the `larger social issue' which is suicide. For example, the very last line is great, and would be perfect for a purely personal essay, but might not be the best conclusion to this nonfiction piece.

    \section{By Tab Cusack:}
    This piece isn't a discussion of a social concept, it's a privileged rant against a public transit system and the poor which use it.

    Perhaps a better discussion would be \emph{why} the population of SEPTA irk the author so much. In fact, I might just write \emph{my} paper on it.
    \section{By Maura Hanley:}
    
    The first sentence of this piece is somewhat misleading or sensationalist: `Google' is neither a noun nor a verb--although it may have begun to enter the popular vernacular as such, it is not recognized as  one by any of the foremost curators of the English language, and additionally, the company itself has been actively trying to \emph{prevent} it from becoming such. The second sentence exhibits some ignorance to the topic at hand: Google is far beyond an `internet search site', and is in fact the world's largest advertising agency. The third sentence sounds like something my mother or even grandmother would say. I could go on...

    This piece could have an interesting focus (the increasing trend or dependency on a single service in a users personal and business life) but sadly, it does not. Instead, it takes a strangely exaggerated tone as it describes a myriad uses for the topic at hand. The voice is close to sarcasm, but... not quite.

    \section{By Claire Pratty:}
    This piece is fantastically written, and the voice carries both anger and self-disgust extremely well. The introduction is perfect, the exposition is perfect, and the conclusion is perfect. It is a clear-cut example of what a `social concept' piece should be. There is a direct and effective correlation between a personal issue and the greater social problem.

    My only issue, and I mean \emph{only} issue with the piece is the line ``Bullying is normal.'' While I get what you mean--that bullying is commonplace, or here to stay, or happens to everyone--it implies a tone of helplessness, or of acceptance. Yes, bullying happens often, especially at such an age, but it is far from a `normal' social human interaction. Other than that, though, bravo.

    \section{By Grey Shenk:}
    While this piece is a great personal story of one person's experiences with the Chicago, the direction is unclear. It seems there are two lines which could each support their own story: either personal experience with (and history of) the The Taste of Chicago, or a historical perspective of Chicago's parks, buildings and infrastructure.

    The description is thorough--I could clearly see the crowds in Grant Park, watching the fireworks, etc., but at times there was simply too much detail, which really could bog down the reading. With a generous editing, and an increased focus on a greater social concept, this could be a very well-written and interesting essay. 
\end{document}
