\documentclass[letterpaper]{article}
\title{Extra Credit: Habit}
\date{}
\usepackage{setspace}
\pdfpagewidth 8.5in
\pdfpageheight 11in
\begin{document}
    \maketitle
    %\thispagestyle{empty}
    \onehalfspacing
    To be, as always, entirely honest, I had not planned on completing the Extra Credit Assignment for this class. I had skimmed the description, to be sure, but I received exactly zero understanding of the goal. Perhaps thoughts of roast turkey and tart cranberries muddled my cognizance--it is yet to be determined.

    However, entirely inadvertently, and without any regard to the Assignment, I actively participated in both avoiding an existing habit, and resurrecting an old one, completely unrelated to one another. Upon re-reading the assignment description, I realized I had perhaps subconsciously understood and completed what I needed to do to participate in the assignment. Although, there may be other factors at play...

    To call `drinking alcohol' a `habit' evokes quite a negative image, one of AA meetings, folding chairs and coffee in styrofoam cups, or perhaps DUIs and police intervention and jail time and maybe even the untimely death of that four year-old on a trike who lives next door to you. Well, maybe not, but it's still a pretty grim representation of my habitual self, and one that absolutely does not fit. My drinking habits are, on average, perfectly average for a twenty-something such-and-such.

    Unfortunately for my track record (and my liver), last week presented itself with a Very Big Life Moment in the form of a Very Messy And Depressing Break-Up. Such events, I have found, can only be rectified by the immediate imbibition (most definitely not the correct usage of the word) of a Very Large Quantity Of Alcohol. While cathartic, the process was also somewhat... purging, if you will, and left me with the desire to take a break for a little bit.

    The holidays provided two arenas in which to test my newfound habit of all-encompassing sobriety: The Family Dinner and The High-School Reunion. Unexpectedly, The Family Dinner was easier than anticipated: the company of simply immediate family provided far less speculation and discussion on my Newfound Bachelorhood and The Fate Of Future (Great-)Grandchildren than expected, as well as the resulting stress. Thank goodness.

    The (unofficial) High-School Reunion proved to be more difficult. Seated around a table at a bar with a crowd of faces which could only be foggily recalled until recently, as well as a fair share of those which were previously unknown, I simply did not know what to do. Where do I put my hands? Look around--everyone has a grip on some piece of glass. I'll spin this pocket-item around. I'll order some fries, that'll keep me busy. Now I'll chew some gum. Should I order a glass of water? Do they serve non-alcoholic beer here? That's a waste of money for sure...

    In the end, everyone was positively sloshed and having a fantastic time, while I was left feeling awkward and sleepy. Why? Was it the absence of a habit I had become accustomed to? Was it infamous peer pressure in action? Could it simply be boredom? Boredom was, without a doubt, the root for the resuscitation of my old habit.

    As the joke goes, the way to know if a person is/was/has/does \_\_\_\_ is: They'll tell you. Well, I don't have a TV. I've kicked the cable habit. But, as the family packed up for an out-of-state college football game without me, and my own plans fell through, I realized I would have two whole days where my only companion would be that evil, addictive, trashy, stupid, boob tube.

    So I succumbed.  I watched Fight Club, Ocean's Eleven, Boondock Saints. I watched cartoons. I watched The News, and I watched Faux News. I avoided a medley of Star Wars for the entire Transporter Series. I watched countless numbers of trashy sitcoms, talk-shows and `reality'-television. I watched several episodes of It's Always Sunny, and I even caught the Christmas episode of Fresh Prince of Bel Air. When I fell asleep, the drone of Paid Programming lulled me to dreamland.

    I did not read a book. I did not get any work done. I didn't leave the house, or speak to a single other human being. I consumed as much terrible fiction and targeted advertising as I possibly could, and I learned nothing. In terms of creative apathy, this was a nuclear bomb of dispassion. But perhaps it was exactly what I needed.
\end{document}
