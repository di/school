\documentclass[letterpaper]{article}
\title{Week 7 Reviews}
\date{}
\usepackage{setspace}
\pdfpagewidth 8.5in
\pdfpageheight 11in
\begin{document}
	\maketitle
	%\thispagestyle{empty}
	\onehalfspacing
	\section{By Claire Pratty:}
		This piece is pretty cute, and well written, and at times pretty funny. I was able to get a great image of Bear, however it's not hard to describe stereotypical `puppy' characteristics. The juxtaposition between the old, neat narrator and the current narrator with the puppy seems to be nicely revealing. 
		
		Ultimately, the piece seemed lacking towards the end, as I was hoping for more retrospective on how the narrator had changed and what the results were. There is a little bit (in the last paragraph, ``Bear has taught me...'') but it does not seem to come full circle, as it easily could. Overall, there are very few grammatical/syntactical errors, and it flowed logically and at a good pace.
	\section{By Breanna Johnson:}
		The choice to portray the evolution of the narrator's personal expression was an interesting one, and it worked well throughout the story. I believe there is some interesting thoughts that could have been introduced via doing so with a thrift-store or consignment-store `find' (or rather, anything which has already perhaps served it's intended purpose) but this piece is OK without it. 

		I had a particular issue with the description of the sweater: Although it is an integral part of the piece, and it is relatively thoroughly described, the visual image did not stick with me or remain as I continued reading the story, although I'm not sure why this is. A larger issue, however, is the inconsistency between tenses, and the lack of what could be some useful punctuation and formatting in a few areas.
	\section{By Jeletta Jose:}
		I am a sucker for second-person narratives, and I found this to be a pretty good one, so it was thoroughly enjoyed. Second-person can be difficult to pull off, but I believe the author did it relatively well. I particularly like balloon imagery, but feel that it could have possibly been further expounded upon, e.g. tying a string to it to hold it close/prevent escape, or pulling it around with you everywhere via the string (although I don't know if this would have worked for the author).
		
I felt that it was a little too short, but only because it seems to lack the continued exposition the reader may desire. Perhaps not enough information is revealed at the end. 
		
Again, second-person can be difficult. There was a momentary point where I was confused about who was who, but this was partly because I first read the narrator as being a woman speaking to a man, and not some form of inner-dialog. Although it was rectified early on in the piece, I became a little confused again when more characters were introduced. It could use a little reorganization, but I believe you essentially pulled it off.
	\section{By Heidi Arnold:}
		The narrator's room is definitely a unique place, and is clearly worth describing. Images such as the bookcase can tell the reader a lot about the owner, especially if they are portrayed as the author has.
		
However, the story seems to start with one tone (very proper, the room is kept `secret' and not shown to anyone), but the narration quickly changes gears to describe how warm and inviting it is. While it may have such duality, some work needs to be done on the transition between the two, as it should be like revealing a walled garden.
		
Furthermore, the piece is perhaps a little long, and could perhaps use some critical reduction. There are sections where the author simply begins to talk about themselves, rather than describe the room, which should be done by the description instead.
	\section{By Max Gerschman:}
		I enjoyed this piece. The author not only made a great argument for television (which could be arguably hard to do) but allowed it to describe the memories of their childhood as well in an effective way. I had a hard time finding issues with this piece... my only disappointment was that there was not as much humor behind the writing as I originally had anticipated. This in part, however, stemmed from thought that the phrase ``massive entertainment centers'' was quite humorous, although maybe it was not intended to be.
		
Overall, there were little to no errors. My only suggestion would be to maybe split it into more than just two paragraphs.

   \end{document}
