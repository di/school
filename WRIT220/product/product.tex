\documentclass[letterpaper]{article}
\title{The Product Being Sold}
\date{}
\usepackage{setspace}
\pdfpagewidth 8.5in
\pdfpageheight 11in
\begin{document}
    \maketitle
    \thispagestyle{empty}
    \onehalfspacing
    Imagine a corporation which provides a simple service. If you sign up, they will send someone to your house to follow you around and help you carry your things -- a personal butler, if you will. At first, you're not sure you really need the help, but then you notice that your friends all have personal butlers as well, and they seem to be really useful. Your friends can carry more groceries, don't get tired from carrying their child, never have to carry a purse or backpack again. The personal butler even goes so far as to carry your phone, laptop, and address book for you, so if you need to make contact anybody, they dial the number for you and hold the phone to your ear, or take a dictation for an email.

The best is, this service is provided for free. You don't have to pay the butler, or tip him, nor does he even ask you for money. You think this sounds pretty good, so you sign up--and join lots of other people as well--because who couldn't use a hand every now and then, if nothing else?

The only thing is, this personal butler also quietly records everything you do. He writes down who you talk to, what you say, what you're wearing, what you buy at the store, the stores you go to, how much your salary is, what kind of food you eat, when you have sex, when you talk about a person, or place, or thing. Every time the butler takes a photo for you, they keep a copy for themselves. Every time the butler sends an email for you, they also send it to themselves.

Each day, the butler writes a short report about you and what you've done, and sends this portrait back to the company for which they work. This company takes all the information they've gathered from your butler, as well as everyone else's butlers, and builds a model which it then sells to advertisers. The advertisers then come up with targeted ads for each model, which the company passes back to the butlers.

The butler then works this advertising into your everyday life. If they know you like Coke, and they see that you're thirsty, they're instructed to casually mention Coca-Cola, or even go get one without you asking, and offer to charge it to your account. 

While this scenario may be cost-prohibitive or extreme when extrapolated into real life, with the power of the Internet and technology, it is essentially possible, and is what happens online if one does not pay attention. Unfortunately, there is a growing generation of young of young people who either don't understand the importance of their privacy, or simply don't care. This generation values the dead-simple social interaction provided by immense internet technology and social network corporations such as Google, Twitter, and Facebook far more than they value their personal information. Instead, they freely share details of their lives which they would normally keep from most strangers.

I believe that the ignorance of those in this generation stem not from their unwillingness to understand how important their privacy is, but rather the difficulty of conceptualizing the ultimate goal of such large, complex companies. In an online age, where we expect content and services to be open and available with little to no cost, it seems hard to remember the scale of some of these web companies, and the fact that they \emph{are} companies, and thus, seek to make a profit.

The truth is, you're getting a service for free, you're not the customer--you're the product being sold.

\end{document}
