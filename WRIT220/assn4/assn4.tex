\documentclass[letterpaper]{article}
\title{Week 7 Response}
\date{}
\usepackage{setspace}
\pdfpagewidth 8.5in
\pdfpageheight 11in
\begin{document}
	\maketitle
	%\thispagestyle{empty}
	\onehalfspacing
	\section{Gerald Early's \emph{Life with Daughters}:}
		Gerald Early's piece provides a perspective of the evolution stereotypically white Americanism from the viewpoint of an black man ingrained with knowledge in the philosophical, psychological, and physical effects of America's great race divide. While I find his viewpoint to be accurate and without qualm, and additionally wholly without racism itself, I cannot understand why he even bothers to give any more attention to the Miss America Pageant than his own thought. He seems to regard it as an entity which cannot be changed, a barometer for American culture and racism which is some form of imperial rule.

		I ask, why even bother? Why give it more attention that it has already amassed? When you declare the first black winner, Vanessa Williams, as ``our own kind of royalty'', you are not only validating the results the complex racial tensions that have produced her, but also rejected her since the 1920s. It seems he should learn to judge the farce with the same light humor that his daughters do, and encourage his readers to do the same, but he does not. He declares their resulting `wisdom', but does not seem to realize that by simply diminishing America's attention to the Pageant, and his own as well, one can diminish the faux-weight which is has come to carry, and, hopefully, eliminate it from popular American culture.

		Well, one can hope, right?

   \end{document}
