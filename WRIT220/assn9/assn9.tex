\documentclass[letterpaper]{article}
\title{Laws of Habit Abstract}
\date{}
\usepackage{setspace}
\pdfpagewidth 8.5in
\pdfpageheight 11in
\begin{document}
    \maketitle
    \thispagestyle{empty}
    \onehalfspacing
    Incredibly, William James is able to postulate his entire theory of the Laws Of Habit without once mentioning sloth, laziness, or efficiency. He describes habit as being a result of an optimally functioning nervous system, of actions slowly becoming purely mechanical and wholly without thought. However, he disregards why certain `habits' become habits to begin with. To borrow from his examples of reflex actions, activities such as dressing, undressing, eating, drinking, etc. have become `habitual' (or more precisely, mechanical) simply because our actions have evolved over a time of trial and error, not unlike Darwin's mockingbirds. Our habits are simply the most efficient developed reaction to a problem, repeated each time the problem must be solved.

    James' correlation between the practice of students and teachers is slim at best. Although good educational endeavors should appear habitual, the need to develop a habit seems to imply a development of the desire to engage oneself habitually. Rather, the good student does not practice, but discovers a thirst for knowledge, for understanding, and for greater perspective, which leads into a habitual education--simply because it is the most efficient way to achieve said goals.

    James speaks of what one might call Darwin's atrophy of creative habit, but fails to recognize the source of this miniature death of self. Instead he encourages the reader to, essentially, just do it. He points out that one forgets, postpones, and that this digs our graves. But he does not out the source, although the answer is simple: if we are machines designed to find the most efficient way to exist, ten minutes of day of poetry must not produce a ideal result. It is widely understood the majority of Darwin's evident success came at the cost of immense stress and his health--perhaps meditation, music, and poetry is not the answer?

\end{document}
