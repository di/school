\documentclass[letterpaper]{article}
\title{Week 4 Readings}
\date{}
\usepackage{setspace}
\pdfpagewidth 8.5in
\pdfpageheight 11in
\begin{document}
	\maketitle
	\thispagestyle{empty}
	\onehalfspacing
   \section{From \emph{My Remarkable Uncle}:}
   This essay provides a sympathetic view of the fall of E. P. Leacock in two different senses. In the author's childhood, E. P. is a revered, mysterious oddity--but as the author grows, he becomes more and more aware of his uncle's fa\c{c}ade. Interestingly, the author shows reverence rather than resent upon this realization, even though he seems to have swindled his own family as well. This reverence is somewhat echoed by his final `resting' place at the monastery. In a similar way, the story also shows E. P.'s fall from power on a much larger scale, and the difference between how the rest of society (barman, etc) versus his nephew reacts. It is a little hard to find the source of the author's lack of anger upon revelation, though.

   \section{From \emph{Goldberg as Artifact}:}
   The author's goal in this piece is to portray Goldberg as someone beyond a simple pizza pusher, an appearance which would be common to nearly everyone he interacted with, save his few pals. Here, we see Goldberg as a man who feels compelled to do more with his life, even though he has already succeeded, someone with girl problems, weight issues, etc. However, it remains a very light piece. Using the `artifact' of the neon sign as a vehicle for some kind of elevation of status, we see he is still the same person--dreaming up new inventions and looking for a wife.

   \section{From \emph{Reflections on Gandhi}:}
   Orwell's writings on Gandhi, similar to the previous piece on Goldberg, portray him not as an average `non-saint' would see him--as perfectly pious, without fault, a saint from youth to death. Such portrayal, while seeming to ground a saint or demi-god, at the same time can elevate our perspective. Gandhi was not born a saint, he was born a man, and in the choice between saint and man that we all have made, he uniquely chose the difficulties of being a saint.

   \section{From \emph{Newton Wicks}:}
   This short story is an excellent description of the forced friendships of childhood, but reveals little about the main character, Newton, until the very end, as he matures into a teen. At this point, the story seems to diverge from images of friendship, and gives hint to his relationship with his father, etc. However, it's not clear to me what this story intends to show, besides a description of some other person. It ends quickly, with a bit of melancholy that does not run through the rest of the story.

   \end{document}
