\documentclass[letterpaper]{article}
\title{The Pine Desk}
\date{}
\usepackage{setspace}
\pdfpagewidth 8.5in
\pdfpageheight 11in
\begin{document}
	\maketitle
	%\thispagestyle{empty}
	\onehalfspacing
	I was seven years old when I first decided to `engrave' the top of the desk in my room. I was a relatively artistic child, usually drawing on paper or my own hands and arms with pen, but it is hard to produce anything artistic with the crudest of implements. I had chosen the sharp, pointed end of my drawing compass as my chisel, and over a number of days, scratched a messy, somewhat symmetrical abstract pattern consisting of curved grooves, lines and holes gouged into the soft pine desktop.

	I remember my handiwork distinctly, yet could not possibly give the motivation behind it. The desk was not actually mine, obviously--it had been my father's, from his own childhood, and while it had faintly shown the signs of thirty years of wear, it didn't, by any means, deserve the destruction I placed upon it. It was small, and simple, with two drawers with large, round, wooden knobs, and a matching chair. The entire thing was sealed and coated with an enamel which made it shine and glow a deep honey color.

	I'm also not sure how I planned to let my work go unnoticed. The surface was visibly marred throughout it's entirety, and small bits of wood, slightly larger than sawdust, littered the floor around it. It was an eventuality that my father would see what I had done, it was simply a matter of when. I figured he might be mad, but it was an old desk, and to me, it was worthless. Frankly, I didn't care.

	When my father saw what I had done, however, he taught me a lesson in a different way. He didn't get mad, or react at all. He simply looked from the desk, to me, and quietly said,

	``Put your shoes on. We're going to the store.''

	My young mind raced. Wait, what? What store? Holy crap, Dad's going to sell me back to the store! No, what are we going to buy? A paddle? A whip? Why would he need to bring me with him? To get the right size? On the ten-minute ride to the store, my father remained silent, while I quickly wound myself into a frenzy of apologies for anything and everything I could think of. This was above and beyond the average yelling or time-out, and I knew not what to expect.

	As it turns out, the trip had brought us to the hardware store. By this point, I was too bewildered to even cry, so the two of us just went inside quietly. We purchased three things: sandpaper, a small can of enamel, and a paintbrush. We put them in the car, and drove home.

	The real work began once we were home, though. My father cleared the desk, took the drawers out, and carried it into the basement, near his workbench. He handed me the sandpaper and the brush, and said,

	``OK. Fix it.''

	I stared at him, completely dumbfounded. I'd never done anything like this before, and had no clue where to begin.

	He ended up sitting with me for most of the afternoon, quietly coaching me through every step. I sanded the top down, millimeter by millimeter, checking that it was level and even, until you could only see the very deepest of small holes, and the surface had completely lost it's gloss. Then, layer by layer, I applied the enamel, until every hole was filled, and the 
surface was smooth, hard, and brilliant.
	
	Working on that desk was one of the best lessons I've ever learned. Instead of getting upset or yelling at me, he let me experience exactly what the consequences of my actions were. He taught me the first of many skills which would come in handy multiple times in my life since then, but most importantly I really learned that damaging anything, especially if it doesn't belong to you, accomplishes nothing besides makes more work for yourself.

	From that day on, that house was sacred, and I took great care of everything in it. If anything broke, I'd help him fix it up, be it plumbing or a broken window or anything else. I grew up a lot that day, thanks to my father.

\end{document}
