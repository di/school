\documentclass[letterpaper]{article}
\title{Week 1 Readings}
\date{}
\usepackage{setspace}
\addtolength{\topmargin}{-40pt}
\addtolength{\textheight}{40pt} 
\begin{document}
	\maketitle
	\thispagestyle{empty}
	%\doublespacing	
	\onehalfspacing
   \section{From \emph{The Old Stone House}:}
   This selection, at first read, seemed to be an overwhelmingly descriptive yet monotonous tale of the author's love affair with a house, which ultimately failed to motivate a second reading. I have nothing else to say.

   \section{From \emph{Tripp Lake}:}

   The author's relationship with her mother bears no resemblance to my own, and therefore I found it to be a little difficult to fully comprehend or enjoy reading. It simply left me feeling cold towards both characters. However, when she first encounters her newfound hobby, I could immediately relate:

   \begin{quote}
      "Today, whenever I enter a bar and smell that smell, I do a Proustian plunge back to that first barn."
   \end{quote}

   After eight years of riding, I can attest this plunge can be magnified a hundred-fold. 

   \begin{quote}
   "Riding. It is about becoming one with the animal that bears you along.  It is about learning to give and take, give the hose his head, take the rein and bring him up... It is, more than anything else, about relationship and balance."
   \end{quote}

   I wish the author had spoken more about developing her relationship with her horse... I believe it to be indicative of a person's relationship with themselves ('largely a singular sport') as well as how they form relationships with others ('give and take') and may have provided insight to her relationship with her mother.

   \section{From \emph{They All Just Went Away}:}

   I am, most definitely, a 'dog person'--even the most tropical portrayal of a dog in film or literature strikes a deeply resonating chord with my heartstrings. Such with 'Slossie':

   \begin{quote}
   "...a mixed breed with a stumpy, energetic tail and a sweet disposition, sand-colored, rheumy-eyed, as hungry for affection as for the scraps we sometimes fed her..."
   \end{quote}

   However, interestingly enough, the following gave me little to no emotional reaction whatsoever:

   \begin{quote}
   "There was a summer day--I was eleven years old--that Mr. Weidel shot Slossie. We heard the poor creature yelping and whimpering for what seemed like hours."
   \end{quote}

   I was a bit shocked that this, as well as the following act, was not able to (or was written not to) affect me. However, once I thought the author was done ('The fire was the following year...') I was hit with this:

   \begin{quote}
   "After the Weidels were gone from Millersport and the house stood empty, I discovered Slossie's grave. I'm sure it was Slossie's grave. It was beyond the dog hurth, in the weedy back yard, a sunken patch of earth measuring about three feet by four with one of Mrs. Weidel's big white-washed rocks at the head"
   \end{quote}

   I was immediately plunged into the depth of the act Mr. Weidel had committed, which I had previously failed to see. He had not killed some mangey, homely-yet-affectionate dog, but had deliberately done it to torture Mrs. Weidel in the cruelest way possible, short of shooting her instead, leaving her to bury what was likely her best friend on her own.

   \section{From \emph{How It Feels to Be Colored Me}:}

   It was enjoyable to see what seems at first to simply be the author's childhood naivety turn into an incredibly positive outlook on life:

   \begin{quote}
   "Someone is always at my elbow reminding me that I am the grand-daughter of slaves. It fails to register depression with me. Slavery is sixty years in the past."
   \end{quote}

   Instead of 'fails to register with me', she says 'fails to register depression with me'--which gives the reader the impression that no, she is not ignorant, that instead she fully understands the past, but chooses to live in the present moment.

   Unrelated, but the following passage seemed to resemble synesthesia quite strongly:

   \begin{quote}
   "Music! The great blobs of purple and red emotion have not touched him. He has only heard what I felt."
   \end{quote}

   A veritable cross-wiring of the senses, if I've ever read one.

   \end{document}
