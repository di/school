\documentclass[letterpaper]{article}
\title{Modern Capitalism as Slavery}
\date{}
\usepackage{setspace}
\pdfpagewidth 8.5in
\pdfpageheight 11in
\renewcommand{\abstractname}{Background}
\begin{document}
	\maketitle
	%\thispagestyle{empty}
	\onehalfspacing
    \begin{abstract}
On January 21st, 2010, the US Supreme Court decided on the case of \emph{Citizens United v. Federal Election Commission}, now widely regarded to be a landmark ruling. As a result, the Court struck down laws which banned corporations from using their own money to support or oppose political candidates, under the premise that this would impede their First Amendment rights. However, as the dissenting ruling states, giving such power to a corporate entity ``threatens to undermine the integrity of elected institutions across the Nation.''
\end{abstract}

The \emph{Citizens United v. Federal Election Commission} ruling seeks to protect the free speech of corporations as individuals under the First Amendment. In recent main-stream media discussions, this has been reduced to the ideological foundation that ``Corporations are People''. If we apply basic logic, this makes sense: every corporation is a collection of individuals, each of whom have the right to Free Speech, therefore any collection of individuals, be it one, two, or twenty thousand, still have the same rights as before. 

However, let's continue with the logic. Corporations continuously buy and sell one another, and a smaller corporation is often owned by a larger one. If we can apply the First Amendment to corporate entities as individuals, can we also apply the Thirteenth Amendment, which abolished slavery (one individual owning another)?

In his writings, Noam Chomsky makes a similar argument likening capitalism to psychological slavery. He explains that, for hundreds of years, slavery had been wholly and globally unchallenged. The ethical and social implications were not considered until the 19th century, and, before this, there were but a few small uprisings which granted greater freedoms. The argument was that, ideally, slaves would not be marginalized, de-humanized, or treated as cogs in a machine, but rather as if they were property, and would be maintained or taken care of not unlike a rented house or common livestock.

If you are able to suspend the glaring and abhorrent issue of slavery itself, this makes sense. However, is this not what corporations are doing to its employees and consumers today?

Hasn't capitalism, too, gone unquestioned for hundreds of years, with but a few major uprisings in the last century? Don't major corporations directly tout their investments of `human capital'? The ideal corporation doesn't treat their workers like cogs in a wheel either--they invest in their employees. But what if the fundamental premise--capitalism--is wrong?

   \end{document}
