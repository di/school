\documentclass[letterpaper]{article}
\title{Syd's Gift}
\date{}
\usepackage{setspace}
\pdfpagewidth 8.5in
\pdfpageheight 11in
\begin{document}
	\maketitle
	\thispagestyle{empty}
	\onehalfspacing
Syd brought it by one evening when we were having a party, and handed it to me as casually as if it were a six pack. Stifling a giggle that made him seem twenty years younger than us instead of older, he said, ``I got you a present,'' and smiled.

It was one of the ugliest pieces of mass-produced consumer crap-art I've ever seen, and he very well knew it. It was a picture frame for an 8x5 that was easily three times the size it needed to be. The main body of the frame was a solid piece of unpainted wood, an inch thick, with an off center hole cut for the photo. Outlining this frame, as well as the hole in it, were some random sticks, that had been cut down to size and at all angles, and hot-glued in place, giving a good attempt at that `rustic' look.

Immediately to the right of the photo were glued two jaggedly cut pine trees, resembling what children would do if handed a jigsaw. To cover the base of the trees, a merrily rolling hillside was placed -- and had it been painted blue, and not green, it could have easily been a river. Glued on top the largest tree was a sorry excuse for a goose, with one wing easily twice the size of the other. And behind this goose was either a sun or a moon -- but we'll leave that up for interpretation.

It was clear that the previous owner -- as Syd had undoubtedly taken this monstrosity out of a pile of trash on the way to my house -- had thought similarly, as it still held it's original stock photo: an image of a forest of pine trees. As if anyone would want to display a photo of trees in a frame with more trees on it.

Looking at it, I thought, How many of these exist? Probably thousands, if not hundreds of thousands. However, I was determined to become the owner of the most unique identical piece of art amongst the entire crowd. I quickly found, printed, cut to fit and placed an image of a glowing mushroom cloud, rising to meet a bright blue sky, undoubtedly pouring fire and ash upon the surrounding wildlife, and laying waste to what I can only hope to be thousands of trees.

Throwing the frame on a nail, I proudly called Syd -- who, I should add, is a total hippie and prolific artist -- over to admire my work. He was far from amused, his plan had backfired. This piece would exist for a much longer time than it should. But I, I loved the absolute absurdity of it, and to his chagrin, it still hangs on my wall today.
\end{document}
