\documentclass{article}
\usepackage{fullpage}
\title{BLAW201 - Final Exam}
\author{Dustin Ingram}
\date{\today}
\begin{document}
	\maketitle
	\section*{Chapter 15 - Consideration}
	Consideration is a legal, bargained for detriment on both sides. It is what each party to a contract gives up to the other in making their agreement.
	  \begin{itemize}
	    \item Leagal Detriment:
		\begin{itemize}
			\item Taking on some duty, task, or obligation that you had no pre-existing duty to take on, or giving up some right you had no pre-existing duty to give up.
			\item If the doing of an act would be legal detriment to the do-er, then the promise to do it is considered legal detriment as well.
		    \item Peppercorn Rule: Any detriment is resonable detriment, e.g. Early payment is detriment.	
			\item Payment to a different person, to a different place, at a different time, or in a different medium than was originally required.
		\end{itemize}
	    \item Bargained-For Exchange: Something of value must be given or primised in return for the performance or promise of performance of the other.
		\item Gifts: Promises to make a gift are unenforceable promises because of the lack of consideration. There is no bargained-for exchange.
		\item Addequacy of Consideration: Because the adequacy of consideration is ignored, it is immaterial that consideration is so slight that the transaction is in part a ``gift.''
	    \item Pre-existing Duty Rule: 
		\begin{itemize}		
			\item If you have a pre-existing legal obligation to do something, it can't be considered legal detriment
			\item Similarly, a promise to refrain from doing what one has no legal right to do is not consideration
		\end{itemize}
	    \item Lack of Liquidation: Where a creditor promises to take less from a debtor where the amount is not in dispute, that is not legal detriment
	    \item Implied Terms: Personal debts must be delivered to the creditors home
	    \item Promissory Estoppel: A promisor may be prevented from asserting that a peromise is unenforceable because the promisee gave no consideration.
	    \begin{itemize}
	      \item Reasonable promise without consideration
	      \item Promisor expects reliance on the promise
	      \item The promisee relies on the promise
	      \item Enforcement of the promise avoids injustice
	    \end{itemize}
	  \end{itemize}
	  \section*{Chapter 19 - Discharge}
	  Discharge is the release of obligations from a contract, legally
	    \begin{itemize}
	      \item Reasons for Discharge:
	      \begin{itemize}
	        \item The destruction of the subject matter discharges the contract
	        \item Intervening illegality - the contract is discharged if a law is enacted making the performance illegal
			\item Death in regards to a highly skilled or personal performance
			 \item Frusteration of Purpose: contract can only be discharged if both parties are aware and the purpose no longer exists
	      \end{itemize}
		\item Not Reasons for Discharge:
		  \begin{itemize}
			\item Death of a party does not discharge a contract
			\item Impossibility: if the reason is one that is reasonably within the contemplation of the parties at the time the contract is enacted
			\end{itemize}	     
	    \end{itemize}
	    \section*{Chapter 20 - Remedies}
	    Remedies are damages for Breach of Contract
	    \begin{itemize}
	      \item The philosophy of contract damages is to place the injured party inthe position he/she would have been if the contract had been performed, as best as money can
          \item Mitigation Duty: Party has a duty to make the breaching parties damages as little as possible
	      \item Compensatory Damages
	      \begin{itemize}
	        \item Buyer breaches - Seller attempts to re-sell to another buyer, and recieves the differences from the contract
	        \item Seller breaches - Buyer attempts to find another seller, and recieves the differences from the contract
        \end{itemize}
        \item Incidental Damages
        \begin{itemize}
          \item Additional costs as a result of the breach
        \end{itemize}
        \item Nominal Damages - small amounts
        \begin{itemize}
          \item It is the duty of the plantiff to prove damages with reasonable certainty
        \end{itemize}  
        \item Liquidated Damages
        \begin{itemize}
          \item Allows the parties to agree to an estimate of damages, within reason
          \item Cannot legally be allowed to be unreasonably large, to act as a club or penalty
        \end{itemize}
        \item Consequential Damages
        \begin{itemize}
          \item Similar to Incidental damages, costs as a result of the breach
        \end{itemize}	        
        \item Specific Performance
        \begin{itemize}
          \item Force to perform the contract
          \item Where the item is rare and money is not a reasonable recompense
          \item Personal service contracts are not enforcable
        \end{itemize}
	    \end{itemize}
\end{document}

