\documentclass[11pt,a4paper]{article}
\usepackage{ifpdf}
\usepackage[utf8]{inputenc}
\usepackage[USenglish]{babel}
\title{BLAW 201 - Supplemental Questions}
\author{Dustin Ingram}
\date{\today}
\ifpdf
\pdfinfo {
	/Author (Dustin Ingram)
}
\fi
\begin{document}
	\maketitle
	\section{Chapter 12 - Nature and Classes of Contracts}
    \subsection{Fruit: Apples \& Oranges}
      Here, A is suing B because under certain contact of B's, B was under a quasi contractual liability to pay A. To counter, B claimed that he had never made any agreement with A, so therefore was not required to pay. This case is like comparing apples and oranges--a quasi contract implies that there was never any actual contract, so B's argument that he had never made any agreement is true, but not a valid argument.
    \subsection{Plumbing}
      Here, since the contractee is informed of the changes to the implied contract, and makes no objections, the implied conduct requires additional payment and it is valid under said contract. In another situation, if there was additional conduct which needed to be taken to complete the previously agreed-upon job, and the contractee was not informed, it is possible than an implied contract could be justified, simply because the initial agreement was to complete the task at hand. However, in another situation, if the plumber noticed other problems which would require additional work, but were not related to completing the already contracted job, he would not be able to claim an implied contract to fix the additional problem without informing the contractee, or creating a new contract, whether written or implied.    
    \subsection{Advertising}
      Here, the argument is that the comissioners refused to pay the advertising fee because there was no written contract, however, this is not a vaild argument because a written document is not required to form a contract. Had the state statute said that a written contract was required, this would be a valid argument. This case requires some more information though, such as: Was the Greensburg Times the only publisher who could publish the proceedings? Had the Greensburg Times published the proceedings before?
    \subsection{Car Theft}
      Here, Lombard is not liable for the repair bill, but not because this is contrary to the contract between Lombard and the insurance company. Lombard is not liable because he never entered into a contract with the auto repair service. The auto repair service should not have sued Lombard for unjust benefit, but rather sued the insurance company for breech of contract. 
   \section{Chapter 13 - Formation of Contracts}
\end{document}
