\documentclass{article}

\usepackage{url}
\usepackage{tikz}
\usepackage{float}
\usepackage{amsmath}
\usepackage{enumitem}

\usetikzlibrary{matrix, shapes, snakes}

\title{CS 610: Midterm\\Winter 2013}

\author{Dustin Ingram}

\date{\today}

\begin{document}

\maketitle

\begin{enumerate}[start=0]

\item{} % 1

\begin{enumerate}

\item{} % a
    We consider three probabilities:
\begin{itemize}
    \item{The probability of being in the initial start state, $S_0$, where:}
    \begin{align*}
        S_0 &= \text{(\texttt{LanguageA}/Anti-Polka)}\\
        P_0 &= 0.7 \cdot 0.6 = 0.42
    \end{align*}
    \item{The probability of a transition from $S_0$ to $S_1$, where:}
    \begin{align*}
        S_1 &= \text{(\texttt{LanguageB}/Anti-Polka)}\\
        P_1 &= 0.15 \cdot 0.7 = 0.105
    \end{align*}
    \item{The transition from $S_1$ to $S_2$, where:}
    \begin{align*}
        S_1 &= \text{(\texttt{LanguageC}/Dummy)}\\
        P_1 &= 0.7 \cdot 0.3 = 0.21
    \end{align*}
\end{itemize}
Thus, the probability that \emph{at the beginning of the day} the machine will
produce chips in the order $S_0 \to S_1 \to S_2$ is:
\begin{align*}
    P_0 \cdot P_1 \cdot P_2 &= 0.42 \cdot 0.105 \cdot 0.21\\
    &= 0.009261\\
    &\approx 1\%
\end{align*}
\item{}
    We consider the first, heavy TV. For simplicity, we use the following
    notation:
    \begin{align*}
    P(C|W_h) &= P(\text{chip}_1, \text{chip}_2 | \text{weight} = \text{heavy})\\
    P(W_h|C) &= P(\text{weight} = \text{heavy} | \text{chip}_1, \text{chip}_2)\\
    P(W_h) &= P(\text{weight}=\text{heavy})\\
    P(C_s) &= P_{start}(\text{chip}_1, \text{chip}_2)
    \end{align*}
    The probability for the chips at the start of the run, $P(C_s)$ is given in
    part (a) and is as follows:
    \begin{figure}[H]
    \centering
    \begin{tabular}{|c||c|c|c|}
        \hline
        & \texttt{LanguageA} & \texttt{LanguageB} & \texttt{LanguageC}
        \\\hline\hline
        Anti-Polka & 0.42 & 0.14 & 0.14 \\\hline
        Dummy & 0.18 & 0.06 & 0.06 \\\hline
    \end{tabular}
    \end{figure}
    We are given $P(W_h|C)$ by the table in the question. $P(W_h)$ is given as
    follows:
    \begin{align*}
        P(W_h) = \sum{\left( P(W_h|C)\cdot P(C_s)\right)} = \textbf{0.486}
    \end{align*}
    Using Bayes' theorem:
    $$ P(C|W_h) = \frac{P(W_h|C) \cdot P(C_s)}{P(W_h)} $$
    Thus, the resulting values for $P(C|W_h)$ for the first, heavy TV are as follows:
    \begin{figure}[H]
    \centering
    \begin{tabular}{|c||c|}
        \hline
        $P(\text{Anti-Polka},\text{\texttt{LanguageA}}|\text{weight}=\text{heavy})$
        & \textbf{0.60} \\\hline
        $P(\text{Anti-Polka},\text{\texttt{LanguageB}}|\text{weight}=\text{heavy})$
        & \textbf{0.14}\\\hline
        $P(\text{Anti-Polka},\text{\texttt{LanguageC}}|\text{weight}=\text{heavy})$
        & \textbf{0.12}\\\hline
        $P(\text{Dummy},\text{\texttt{LanguageA}}|\text{weight}=\text{heavy})$ &
        \textbf{0.11}\\\hline
        $P(\text{Dummy},\text{\texttt{LanguageB}}|\text{weight}=\text{heavy})$ &
        \textbf{0.01}\\\hline
        $P(\text{Dummy},\text{\texttt{LanguageC}}|\text{weight}=\text{heavy})$ &
        \textbf{0.01}\\\hline
    \end{tabular}
    \end{figure}
    For the second, light TV, we use the following
    \begin{align*}
    P(C|W_l) &= P(\text{chip}_1, \text{chip}_2 | \text{weight} = \text{light})\\
    P(W_l|C) &= P(\text{weight} = \text{light} | \text{chip}_1, \text{chip}_2)\\
    P(W_l) &= P(\text{weight}=\text{light})\\
    P(C_t) &= P_{transition}(\text{chip}_1, \text{chip}_2)
    \end{align*}

    We no longer need to consider the probabilities of the starting state, but
    rather the transition between any two states:
    \begin{figure}[H]
    \centering
    \begin{tabular}{|c||c|c|c|c|c|c|}
        \hline
            & AP, \texttt{LA} &
            AP, \texttt{LB}&
            AP, \texttt{LC}&
            D, \texttt{LA}&
            D, \texttt{LB}&
            D, \texttt{LC}\\\hline
            AP, \texttt{LA}&
            0.490 & 0.105 & 0.105 & 0.210 & 0.045 & 0.045\\\hline
            AP, \texttt{LB}&
            0.105 & 0.490 & 0.105 & 0.045 & 0.210 & 0.045\\\hline
            AP, \texttt{LC}&
            0.105 & 0.105 & 0.490 & 0.045 & 0.045 & 0.210\\\hline
            D, \texttt{LA}&
            0.210 & 0.045 & 0.045 & 0.490 & 0.105 & 0.105\\\hline
            D, \texttt{LB}&
            0.045 & 0.210 & 0.045 & 0.105 & 0.490 & 0.105\\\hline
            D, \texttt{LC}&
            0.045 & 0.045 & 0.210 & 0.105 & 0.105 & 0.490\\\hline
    \end{tabular}
    \end{figure}
    However, this probability of transition is assuming that we are equally
    likely to be in any given preceding state. As we have shown in the first
    half of the question where the first TV is heavy, in this case there is a
    known probability for being in a certain state before the transition. We
    must multiply this across the transition table to get the probability of
    transition if the first TV was heavy:
    \begin{figure}[H]
    \centering
    \begin{tabular}{|c||c|c|c|c|c|c|}
        \hline
            & AP, \texttt{LA} &
            AP, \texttt{LB}&
            AP, \texttt{LC}&
            D, \texttt{LA}&
            D, \texttt{LB}&
            D, \texttt{LC}\\\hline
            AP, \texttt{LA}&
            0.296 & 0.064 & 0.064 & 0.127 & 0.027 & 0.027 \\\hline
            AP, \texttt{LB}&
            0.015 & 0.071 & 0.015 & 0.006 & 0.030 & 0.006 \\\hline
            AP, \texttt{LC}&
            0.012 & 0.012 & 0.056 & 0.005 & 0.005 & 0.024 \\\hline
            D, \texttt{LA}&
            0.023 & 0.005 & 0.005 & 0.054 & 0.012 & 0.012 \\\hline
            D, \texttt{LB}&
            0.001 & 0.003 & 0.001 & 0.001 & 0.006 & 0.001 \\\hline
            D, \texttt{LC}&
            0.001 & 0.001 & 0.003 & 0.001 & 0.001 & 0.006 \\\hline
    \end{tabular}
    \end{figure}
    And then sum the probabilities that result in the same state to get $P(C_t)$
    for a TV following a heavy TV:
    \begin{figure}[H]
    \centering
    \begin{tabular}{|c||c|c|c|}
        \hline
        & \texttt{LanguageA} & \texttt{LanguageB} & \texttt{LanguageC}
        \\\hline\hline
        Anti-Polka & 0.348 & 0.154 & 0.143 \\\hline
        Dummy & 0.196 & 0.082 & 0.077 \\\hline
    \end{tabular}
    \end{figure}
    Again, using Bayes' theorem:
    $$ P(C|W_l) = \frac{P(W_l|C) \cdot P(C_t)}{P(W_l)} $$
    and:
    \begin{align*}
        P(W_l) = \sum{\left( P(W_l|C)\cdot P(C_t)\right)} = \textbf{0.211}
    \end{align*}
    Gives us the following probabilities for $P(C|W_h)$ for the second, light TV:
    \begin{figure}[H]
    \centering
    \begin{tabular}{|c||c|}
        \hline
        $P(\text{Anti-Polka},\text{\texttt{LanguageA}}|\text{weight}=\text{light})$
        & \textbf{0.165} \\\hline
        $P(\text{Anti-Polka},\text{\texttt{LanguageB}}|\text{weight}=\text{light})$
        & \textbf{0.146}\\\hline
        $P(\text{Anti-Polka},\text{\texttt{LanguageC}}|\text{weight}=\text{light})$
        & \textbf{0.204}\\\hline
        $P(\text{Dummy},\text{\texttt{LanguageA}}|\text{weight}=\text{light})$ &
        \textbf{0.186}\\\hline
        $P(\text{Dummy},\text{\texttt{LanguageB}}|\text{weight}=\text{light})$ &
        \textbf{0.116}\\\hline
        $P(\text{Dummy},\text{\texttt{LanguageC}}|\text{weight}=\text{light})$ &
        \textbf{0.182}\\\hline
    \end{tabular}
    \end{figure}






\item{}
\end{enumerate}
\end{enumerate}

\end{document}
