\documentclass[conference]{IEEEtran}
%\usepackage{draftwatermark}
\usepackage{graphicx}
\usepackage{cite}
\usepackage[belowskip=-10pt,aboveskip=2pt]{caption}
\usepackage[cmex10]{amsmath}
\usepackage{tikz}
\usetikzlibrary{patterns}
\usetikzlibrary{positioning}
\usetikzlibrary{arrows}
\usetikzlibrary{decorations.markings}
\newcommand{\todocomment}[2]{{\bf\sc (* from #1: #2 *) }}

\begin{document}

\title{Behavior Identification and Analysis of Philadelphia Parking Authority Agents}

\author{
\IEEEauthorblockN{Dustin~S.~Ingram}
\IEEEauthorblockA{Department of Computer Science\\
Drexel University, Philadelphia, PA\\
Email: dustin@cs.drexel.edu}
}

\maketitle

\begin{abstract}
In this project, I will use and compare various machine learning techniques in
an attempt to discover spatio-temporal patterns within time-coded location-based
data on Philadelphia parking ticket data. I will attempt to use the found
patterns to build a predictive model for the most probably current location of
an agent at a given day and time, as well as identify interesting
characteristics and behaviors within the data.

\end{abstract}

\section{Related Work and Novelty}
\label{sec:intro}
Finding patterns in movement data is a problem that has roots in a diverse array
of problem sets, from searching for suggestions for improvements of public and
mass transit in pedestrian, to analysis of traffic bottlenecks based on driver
data, to using ship movement data to detect over-used and under-used shipping
channels, to analyzing the movements of herds of animals.

This paper will attempt to use machine learning to perform a spatio-temporal
analysis of a body of data which consists of a large portion of parking ticket
data for the City of Philadelphia, spanning the months of December, January and
February.

The Philadelphia Parking Authority distributes over three thousand tickets per
day over a broad geographical area which comprises the metropolitan area of
Philadelphia for a variety of infractions. There has been little to no
publically available analysis of the behavior or effectiveness of patterns of
individual agents, either in Philadelphia or other cities.


\begin{figure}[h]
\small{\texttt{
    \{ "\_id" : 58686207, "resolved" : false, "violation" : "METER EXPIRED CC",
    "issueTime" : ISODate("2013-02-09T17:02:00Z"), "violationCode" : "1210051
    C", "location" : "300 BLK SOUTH ST SS"\}}
\caption{A sample of individual ticket data.}}
\end{figure}


\section{Approach}
\label{sec:approach}
The approach will be to collect data and process it, including geocoding, which
requires building a data-collection system and corresponding database for
storing the retrived data. Then, the data will be given a preliminary evaluation
based on the amount collected at time, and may be limited to a certain
geographic area. Because the data is discretized at the per-block level, it is a
prime candidate for use with a Hidden Markov Model, as there is a number of
distinct data points with very accurate corresponding time-stamps, however on
the general path, there may be points missing from one day to the other which
can be filled in using the model.

Once a model for various agents as been built, this model can potentially be
used to produce a predictive model for agent location at a given day and time.
An underlying assumption of this analysis is that individual paths (and thus,
agents) will be easy to determine and separate from one another. This may not be
the case, as the analysis will show, and an auxiliary approach might be to
produce a ``best-guess'' of a given agent's path.

\section{Evaluation Approach}
\label{sec:results}
Ideally, these results will show deterministic patterns in the paths of the
agents, which will be easily identifiable over a long period of time. If not,
the data will be analyzed to show areas which might be undervisited, and the
analysis will be used to attempt to create artificial agents which can be
trained to develop the most efficient pattern based on the historical data. If
the results do show a deterministic pattern, the data will be used to build a
model of the agent's behavior, specific to geographic location, time of day, day
of week, etc. These models will then be analyzed for their effectiveness as
well. This data will reveal and identify interesting patterns or behaviors of
the agents.

\section{Milestones}
Several milestones for the project are as follows:
\begin{itemize}
    \item \textbf{Mid-January:} Finalize data collection framework;
    \item \textbf{Feb 1}: Collect over 100K records;
    \item \textbf{Early Feb}: Geo-code all location data;
    \item \textbf{Early Feb}: Perform basic analysis of data (location frequency analysis, etc.);
    \item \textbf{Mid Feb}: Machine learning algorithm, introductory analysis;
    \item \textbf{Late Feb}: In-depth analysis based on preliminary results;
    \item \textbf{March 1}: Prepare final results, presentation and paper.
\end{itemize}

\section{Bibliography}
%\bibliographystyle{IEEEtran}
%\bibliography{ppa}

\end{document}
