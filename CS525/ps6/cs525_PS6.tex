\documentclass{article}
\usepackage{booktabs}
\usepackage{amsmath}
\usepackage{amssymb}
\usepackage[noend]{algorithmic}
\usepackage[nothing]{algorithm}
\usepackage{tikz}
\usepackage{latexsym}
\usepackage{float}
\usetikzlibrary{arrows,automata}
\providecommand{\e}[1]{\ensuremath{\times 10^{#1}}}
\renewcommand{\thealgorithm}{}
\title{CS 525: Theory of Computation\\ Problem Set 6}
\author{Dustin Ingram, Aaron Rosenfeld, Eric Simon}
\begin{document}
\maketitle
\begin{enumerate}
    \item[6.4] \textbf{Solution:}
    For the purpose of contradiction, assume ${A_{TM}}^\prime$ is decidable by a TM $A^\prime$.  Let $N$ be a TM defined
    as:\\ \\
    $N$ = On input $(m,c)$ where $m$ is a TM and $c$ is the code of $m$:
    \begin{enumerate}
        \item Simulate $A^\prime$ on $(m, c)$
        \item If $A'$ accepts \textit{reject}; else \textit{accept}.
    \end{enumerate}

    Then, run $N$ on input $(N, m)$ where $m$ is $N$'s code.  The output provides a contradiction: $N$ accepts $w$ if and only if $N$ rejects $w$,
    proving that ${A_{TM}}^\prime$ is undecidable relative to $A_{TM}$.

    \item[6.13] \textbf{Solution:}
    Unlike $\text{Th}(\mathcal{N}, +, \times)$ wherein the possible results of $a \times b$ is unbounded, in
    $\text{Th}(\mathcal{Z}_m, +, \times)$, all results of $a \times b$ must be bounded by $\mathcal{Z}_m$ and therefore
    the TM accepting $\text{Th}(\mathcal{Z}_m, +, \times)$ can test every possibility, allowing a theorem to be decided.

    \item[6.14] \textbf{Solution:}
    Let $J=\left(A \times \{0\}\right) \cup \left(B \times \{1\}\right)$.  That is, a TM $M$ deciding $J$
    decides words in the form $w'=(w, c)$.  To decide if $w \in A$, pass $M$ the input $(w, 0)$.  This will accept if
    and only if the word is in $A$.  Similarly, pass $M$ the word $(w, 1)$ to decide if $w \in B$.

    \end{enumerate}
\end{document}
    \item[6.15] \textbf{Solution:}
    For any language $A$, we can create a language $B$ that consists of the
    set of all strings that map to Turing Machines that decide $A$. If we
    have an oracle for $B$, then the following Turing Machine, $T^{B}$,
    decides $A$: \\ \\
    $T^{B} = \text{On input $w$, where $w$ is a string}$:
    \begin{enumerate}
        \item Enumerate strings and test for membership in $B$. Start with strings
    of length 1, and then length 2, and so on.
        \item If $A$ is decidable, then eventually a string $\langle M\rangle$ in $B$ will be found
    since some Turing Machine must decide $A$.
        \item Convert the string $\langle M\rangle$ into TM $M$, and simulate $\langle M,w \rangle$. Since $M$ decides $A$, it will halt on $w$. If $M$ accepts, accept. If
    $M$ rejects, reject.
    \end{enumerate}
    If such an oracle for $B$ exists, then $A$ is decidable. Therefore, $A\leq_{T}B$.
    On the other hand, if an oracle for $A$ exists, we can't use it to decide
    $B$.

    \item[6.16] \textbf{Solution:}
    The languages $A_{TM}$ and $MIN_{TM}$ are not Turing-comparable.
    The proof that $MIN_{TM}$ is not Turing-recognizable does not rely
    on $A_{TM}$ being undecidable, so an oracle for $A_{TM}$ will not
    decide $MIN_{TM}$, and therefore $MIN_{TM}\nleq_{T}A_{TM}$. Similarly,
    an oracle for $MIN_{TM}$ can't decide $A_{TM}$. In the language of
    $A_{TM}$, we must determine if a given Turing machine $M$ will accept
    a given string $w$. The oracle for $MIN_{TM}$can determine if $M$ is
    a minimal Turing Machine, but that does yield any information about
    whether or not $M$ will accept a string.

    \item[6.18] \textbf{Solution:}
    A Turing Machine $M^{A_{TM}}$ with an oracle for $A_{TM}$ can be
    encoded as a string $\langle M^{A_{TM}}\rangle$. Since the set of all strings
    is countable, the set of all Turing Machines $M^{A_{TM}}$ is also
    countable. Since the set of all languages is uncountable, there is
    no way to map a correspondence from all languages to all Turing Machines
    with oracle for $A_{TM}$. Thus, some languages are not recognizable
    by these machines.
