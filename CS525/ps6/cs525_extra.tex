\documentclass{article}
\usepackage{booktabs}
\usepackage{amsmath}
\usepackage{amssymb}
\usepackage[noend]{algorithmic}
\usepackage[nothing]{algorithm}
\usepackage{tikz}
\usepackage{latexsym}
\usepackage{float}
\usetikzlibrary{arrows,automata}
\providecommand{\e}[1]{\ensuremath{\times 10^{#1}}}
\renewcommand{\thealgorithm}{}
\title{CS 525: Theory of Computation\\ Problem Set 5, Re-do}
\author{Dustin Ingram, Aaron Rosenfeld, Eric Simon}
\begin{document}
\maketitle
\begin{enumerate}
    \item[6.15] \textbf{Solution:}
    For any language $A$, we can create a language $B$ that consists of the
    set of all strings that map to Turing Machines that decide $A$. If we
    have an oracle for $B$, then the following Turing Machine, $T^{B}$,
    decides $A$: \\ \\
    $T^{B} = \text{On input $w$, where $w$ is a string}$:
    \begin{enumerate}
        \item Enumerate strings and test for membership in $B$. Start with strings
    of length 1, and then length 2, and so on.
        \item If $A$ is decidable, then eventually a string $\langle M\rangle$ in $B$ will be found
    since some Turing Machine must decide $A$.
        \item Convert the string $\langle M\rangle$ into TM $M$, and simulate $\langle M,w \rangle$. Since $M$ decides $A$, it will halt on $w$. If $M$ accepts, accept. If
    $M$ rejects, reject.
    \end{enumerate}
    If such an oracle for $B$ exists, then $A$ is decidable. Therefore, $A\leq_{T}B$.
    On the other hand, if an oracle for $A$ exists, we can't use it to decide
    $B$.
    \end{enumerate}
\end{document}
