\documentclass{article}
\usepackage{amsmath}
\begin{document}

\begin{itemize}
    \item[5.15] Let $Q$ be the states of $M$ and $\Gamma$ be the tape alphabet.
    
    Simulate $M$ on input $w$:
    \begin{enumerate}
        \item Simulate $M$ for $|\Gamma| \cdot |Q|^{|w|}$ steps.
        \item At each step mark the current tape position.
        \item If the tape head ever reads a marked tape position, \textit{accept}.
        \item If $M$ rejects or accepts the input, or the number of steps is exceeded without encountering a marked tape
        position, \textit{reject}.
    \end{enumerate}

    \item[5.16] TODO
    
    \item[5.24] It is possible to reduce $\overline{A_{TM}}$ to $J$, indicating $J$ could be used to determine if $w \in
    \overline{A_{TM}}$, which is an undecidable problem:  to check if $w \in \overline{A_{TM}}$ simply check if $0w$ is
    in $J$.

    Similarly, it would be possible to check if $w \in \overline{A_{TM}}$ by simply checking if $0w \in \overline{J}$,
    indicating $\overline{J}$ is also not Turing-recognizable.

    \item[5.29] The first property requires that the property be non-trivial.  This is required as trivial properties
    would allow all TMs in the language, or none at all.  In the first case, one could decide on the property by simply
    accepting on all TMs.  In the second case, one could simply reject all input TMs.

    The second property requires that the property be of the language, not of the TM itself.  This is required as
    properties of TMs are sometimes decidable.  For example, one could count how many states are in a TM, but not if any
    of the states are useless.

    \item[5.30] To prove that each language is undecidable using Rice's theorem, we must show that the languages have
    both properties in question 28:
    \begin{itemize}
        \item[(b)] Let $L(M_1)=\{1011\}$ and $L(M_2)=\emptyset$.  Therefore the property is non-trivial and property 1
        is satisfied.  For any two TMs accepting the same language, either both accept 1011 no neither do, satisfying
        property 2.
        \item[(c)] Let $L(M_1)=\Sigma^*$ and $L(M_2)=\emptyset$.  Therefore the property is non-trivial and property 1
        is satisfied.  For any two TMs accepting the same language, either both have infinite (and equal) languages, or
        neither have infinite (and equal) languages, satisfying property 2.
    \end{itemize}

    \item[5.31] Let $M_a$ be a TM that runs $f(x)$ iteratively and accepts if it encounters a 1.  Clearly this machine
    may never terminate if there is an $x$ for which a 1 is never encountered.  Instead, let $M^{'}_a$ use $H$ to
    determine if $M_a$ accepts $x$ which will terminate.

    Next, let $M_b$ be a TM that ignores its input and iterates $x=1 ... \infty$ and uses $M^{'}_a$ to determine if $x$
    terminates in a 1.  However, this machine may also never terminate.  Instead, use $H$ to determine if $M_b$ accepts
    an arbitrary input.
\end{itemize}

\end{document}
