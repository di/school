\documentclass{article}
\usepackage{booktabs}
\usepackage{amsmath}
\usepackage{amssymb}
\usepackage[noend]{algorithmic}
\usepackage[nothing]{algorithm}
\usepackage{tikz}
\usepackage{latexsym}
\usepackage{float}
\usetikzlibrary{arrows,automata}
\providecommand{\e}[1]{\ensuremath{\times 10^{#1}}}
\renewcommand{\thealgorithm}{}
\title{CS 525: Theory of Computation\\ Problem Set 4}
\author{Dustin Ingram, Aaron Rosenfeld, Eric Simon}
\begin{document}
\maketitle
\begin{enumerate}
    \item[5.13] \textbf{Solution:}
    Let $E = \{\langle M \rangle | M $ is TM with a useless state$\}$. Assume that TM $R$ decides $E$. Construct $S$ that uses $R$ to decide $E_{TM}$, thus arriving at a contradiction, as follows: \\ \\ 
    $S = $ on input $\langle M\rangle$:
    \begin{enumerate}
        \item Run $R$ on $M$. If there are no useless states, then the accepting state is reachable by some input, and therefore $R$ rejects.
        \item If there are useless states in $M$, then we must determine if the accepting state is one of them. For the set of states $Q$, create sets of states where each set represents a case where at least one non-accepting state is missing. 
        \item For each of the sets of states created in step (a), create a machine $M_{i}$ that is identical to $M$ except that it only contains the subset of states.
        \item Run $R$ on each machine $M_{i}$: 

        \begin{enumerate}
            \item If at least one instance of $M_{i}$ is found with no useless states, then the accepting state in this instance of $M_{i}$ is reachable. If we can reach the accepting state in $M_{i}$ then we can reach the accepting state in $M$, since $M$ has all the transitions and states that $M_{i}$ contains. $S$ will reject. 
            \item If every instance of $M_{i}$ has a useless state, then there is no way to reach the accepting state in $M$. $L(M)$ is $\textrm{Ø}$ and therefore $S$ will accept.
        \end{enumerate}
    \end{enumerate}
    $S$ decides if $L(M)$ is empty, which is a contradiction, therefore $R$ does not exist.

    \item[5.14] \textbf{Solution:}
    Let $D = \{\langle M\rangle | M $ is TM which attempts to move its head left when its head is on the left-most tape cell when run on some input $\}$. Assume that TM $R$ decides $D$. Construct $S$ that uses $R$ to decide $A_{TM}$, thus arriving at a contradiction. \\ \\
    $S = $ on input $\langle M, w\rangle$:
    \begin{enumerate}
        \item Create a new machine, $M_{2}$, which is identical to $M$ but with some differences:
        \begin{enumerate}
            \item For every state $q_{i}$, make a state $q_{Fi}$ which moves the tape head to the left and transitions to state $q_{i}$;
            \item Place a dot over the left-most tape cell;
            \item For every transition that moves the tape head to the left, make a duplicate of the transition that operates on the ``dotted'' input, and which writes a dotted version of the same output, moves the tape head to the right, and then moves to state $q_{Fi}$ if the original transition moved to state $q_{i}$.
        \end{enumerate}
        \item $M_{2}$ will accept an input $w$ if and only if $M$ accepts $w$. The only difference is that $M_{2}$ will never move the tape head to the left from the left-most position. 
        \item Create TM $T$ using $M_{2}$ such that: \\ \\
        $T = $ on any input $\langle x \rangle$:
        \begin{enumerate}
            \item Simulate $M_{2}$ on input $w$;
            \item If $M_{2}$ accepts $w$, move the tape head all the way to the left until it reaces the left-most position, and then attempt to move to the left again.
        \end{enumerate}
        \item Use $R$ to decide if $T$ ever attempts to move the tape head to the left from the left-most position. If $R$ accepts, then $M_{2}$ accepts $w$ and so $M$ accepts $w$. If $R$ rejects, then $M$ does not accept $w$.
    \end{enumerate}
    $S$ decides if $M$ accepts $w$, which is a contradiction, therefore $R$ does not exist.

    \item[5.15] \textbf{Solution:}
    Let $Q$ be the states of $M$ and $\Gamma$ be the tape alphabet.
    Let $D = \{\langle M\rangle | M $ is TM which attempts to move its head left when run on some input $\}$. \\ \\
    Simulate $M$ on input $w$:
    \begin{enumerate}
        \item Simulate $M$ for $|\Gamma| \cdot |Q|^{|w|}$ steps.
        \item At each step mark the current tape position.
        \item If the tape head ever reads a marked tape position, \textit{accept}.
        \item If $M$ rejects or accepts the input, or the number of steps is exceeded without encountering a marked tape
        position, \textit{reject}.
    \end{enumerate}
\end{enumerate}
\end{document}
    \item[5.9] \textbf{Solution:}
    We create a a Turing Machine called $R$ that hypothetically decides
    $T$. The specification of this machine is as follows:\\ \\
    Process the input $\langle M, w \rangle$:
    \begin{itemize}
        \item If $M$ accepts $w$ and $w^{R}$, then accept;
        \item If $M$ accepts $w$ but not all $w^{R}$, then reject;
        \item If $M$ does not accept $w$, then accept. (this includes $M$ rejecting $w$, and $M$ not halting on $w$).
    \end{itemize}
    We use $R$ to build a TM $S$ which decides $A_{TM}$. Since $A_{TM}$ is undecidable, we arrive at a contradiction and therefore $R$ does not exist. $S$ must process input $\langle M,w \rangle$ and decide if $M$ accepts $w$, thus deciding $A_{TM}$. Therefore, $S$ is defined as follows: \\ \\
    $S = $ on input $\langle M,w \rangle$:
    \begin{enumerate}
        \item Construct the following TM $M_{2}$ as such: \\ \\
            M$_{2}$= on input $\langle x\rangle$:
            \begin{enumerate}
                \item If $x$ consists of a repeating sequence then find this sequence $w$ such that $w^{R} = x$ and $R\geq1$. 
                \begin{enumerate}
                    \item Run $M$ on $w$. If $M$ accepts $w$, then $M_{2}$ will accept $x$ (i.e, $M_{2}$ will accept $w^{R}$).
                    \item If $M$ does not accept $w$, then $M_{2}$ will reject $x$ (i.e, $M_{2}$ will reject $w^{R}$).
                \end{enumerate}
                \item If $x$ does not contain a repeating sequence, then set $w = x$ and $M_{2}$ will accept $w$.
            \end{enumerate}
            Note that $M_{2}$ is designed to accept $w$ and $w^{R}$ if and only if $M$ accepts $w$. Otherwise, it will accept $w$ and reject $w^{R}$.

        \item Run TM $R$ on input $\langle M_{2}, w \rangle$. 
        \begin{enumerate}
            \item If $R$ accepts, that means that $M_{2}$ accepts $w$ and $w^{R}$, and therefore $M$ accepts $w$, and so $S$ will accept; 
            \item If $R$ rejects, then that means that $M_{2}$ accepts $w$ and rejects $w^{R}$, and therefore $M$ does not accept $w$, and so $S$ will reject.
        \end{enumerate}
    \end{enumerate}
    Since $A_{TM}$ is undecidable, TM $R$ must not exist and therefore $T$ is undecidable.

    \item[5.12] \textbf{Solution:}
    Let $D = \{\langle M \rangle | M $ is single-tape TM which writes a blank symbol over a non-blank symbol when run on some input$\}$. Assume that TM $R$ decides $D$. Construct $S$ that uses $R$ to decide $A_{TM}$, thus arriving at a contradiction. \\ \\
    $S = $ on input $\langle M, w \rangle$:
    \begin{enumerate}
        \item Parse the alphabet $\Gamma$ for $M$, and pick a symbol, $\Omega$, that does not appear in $\Gamma$.
        \item Create a new machine, $M_{2}$, which is identical to $M$ but with two differences:
        \begin{enumerate}
            \item Replace all transitions that write a blank symbol to the tape with an equivalent transition which writes the special symbol $\Omega$. 
            \item For any transition that has the blank symbol on the tape as an input, make a duplicate of this transition that uses $\Omega$ as input. 
        \end{enumerate}
        \item $M_{2}$ will accept an input $w$ if and only if $M$ accepts $w$.
        \item Create TM $T$ using $M_{2}$ as such: \\ \\
        $T = $ on any input $\langle x\rangle$:
        \begin{enumerate}
            \item Simulate $M_{2}$ on input $w$
            \item If $M_{2}$ accepts $w$, then write a blank symbol over any non-blank symbol on the tape
        \end{enumerate}
        \item Use $R$ to decide if $T$ writes a blank symbol over any non-blank symbol. If so, then $M_{2}$ accepts $w$ and therefore $M$ accepts $w$.
    \end{enumerate}
    $S$ decides if $M$ accepts $w$, which is a contradiction, therefore $R$ does
    not exist.

    \item[5.16] \textbf{Solution:}
     If $BB$ was computable, the halting problem could be decided.  As $BB(k)$ is the maximum number of 1s written to the tape for a $k$-state TM, it is also the maximum number of steps that can occur in a halting TM with $k$-states.  Thus, any TM taking more than $BB(k)$ steps to halt, will never halt.

    \item[5.24] \textbf{Solution:}
    It is possible to reduce $\overline{A_{TM}}$ to $J$, indicating $J$ could be used to determine if $w \in
    \overline{A_{TM}}$, which is an undecidable problem:  to check if $w \in \overline{A_{TM}}$ simply check if $0w$ is
    in $J$.

    Similarly, it would be possible to check if $w \in \overline{A_{TM}}$ by simply checking if $0w \in \overline{J}$,
    indicating $\overline{J}$ is also not Turing-recognizable.

    \item[5.29] \textbf{Solution:}
    The first condition requires that the property be non-trivial.  This is required as trivial properties
    would allow all TMs in the language, or none at all.  In the first case, one could decide on the property by simply
    accepting on all TMs.  In the second case, one could simply reject all input TMs.

    The second condition requires that the property be of the language, not of the TM itself.  This is required as
    properties of TMs are sometimes decidable.  For example, one could count how many states are in a TM, but not if any
    of the states are useless.

    \item[5.30] \textbf{Solution:}
    To prove that each language is undecidable using Rice's theorem, we must show that the languages have
    both properties in question 28:
    \begin{itemize}
        \item[b)] Let $L(M_1)=\{1011\}$ and $L(M_2)=\emptyset$.  Therefore the property is non-trivial and condition 1
        is satisfied.  For any two TMs accepting the same language, either both accept 1011 no neither do, satisfying
        condition 2.
        \item[c)] Let $L(M_1)=\Sigma^*$ and $L(M_2)=\emptyset$.  Therefore the property is non-trivial and condition 1
        is satisfied.  For any two TMs accepting the same language, either both have infinite (and equal) languages, or
        neither have infinite (and equal) languages, satisfying condition 2.
    \end{itemize}

    \item[5.31] \textbf{Solution:}
    Let $M_a$ be a TM that runs $f(x)$ iteratively and accepts if it encounters a 1.  Clearly this machine
    may never terminate if there is an $x$ for which a 1 is never encountered.  Instead, let $M^{'}_a$ use $H$ to
    determine if $M_a$ accepts $x$ which will terminate.

    Next, let $M_b$ be a TM that ignores its input and iterates $x=1 ... \infty$ and uses $M^{'}_a$ to determine if $x$
    terminates in a 1.  However, this machine may also never terminate.  Instead, use $H$ to determine if $M_b$ accepts
    an arbitrary input.

