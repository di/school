\documentclass[mathserif]{beamer}

\usepackage{beamerthemeshadow}
\usepackage{graphicx}
\usepackage{graphics}
\usepackage{pgf}
\usepackage{tikz}
\usetikzlibrary{arrows,automata,petri,positioning}
\usepackage[latin1]{inputenc}

\author{Aaron M. Rosenfeld \\ Dustin S. Ingram}

\title{STAGE Prototype}
\date{\today}

\begin{document}

\frame{\titlepage}

\frame
{
    \frametitle{Overview}
    STAGE is an event-based simulation environment used to compare the
    effectiveness of different combinations of software agents, network
    configurations, and sensor data in real-world environments.  It is comprised of
    a distributed simulation engine, visualizer, and programming interface through
    which developers create agent software and network topologies.  Communication
    between virtual nodes is also simulated, providing highly realistic scenarios.
}

\frame
{
    \frametitle{Existing Emulators}

    \begin{itemize}
        \item \textbf{CORE:} Runs unmodified software on virtual machines (Linux namespaces), easy drag-and-drop configuration.  Developed by Boeing/NRL.
        \item \textbf{EMANE:} Purely pre-scripted, one-to-one physical architecture, pub-sub event channel.  Developed by CenGen/NRL.
        \item \textbf{NS 2\&3:} Requires re-writing of existing protocols \& software, proven-accurate radio models.
        \item \textbf{AGLOBE} 
    \end{itemize}
}

\frame
{
    \frametitle{Goals}
    \begin{itemize}
        \item Allow for human-in-the-loop interaction
        \item Make writing agents extremely simple
        \item Easy to define scenarios and network topologies
        \item Entirely event-driven
        \item Agents can be intelligent, pre-scripted, or live-feeds
        \item Visualization flexibility
        \item Distributed
        \item Future: integration of high-quality network models (NS2/3)
    \end{itemize}
}

\frame
{
    \frametitle{Definitions}
    \textbf{Agent API}
    \begin{itemize}
        \item Communication between agents
        \item Publication and subscription to common event channel
    \end{itemize}

    \textbf{Scenario}
    \begin{itemize}
        \item Virtual node positioning
        \item Agent-node mapping
    \end{itemize}

    \textbf{Network}
    \begin{itemize}
        \item Interface specifications
        \item Physical/MAC layer definitions
        \item Routing protocols
    \end{itemize}
}

\end{document}
