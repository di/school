% Plane partition
% Author: Jang Soo Kim
\documentclass{minimal}
\usepackage{tikz}
%%%<
\usepackage{verbatim}
\usepackage[active,tightpage]{preview}
\PreviewEnvironment{tikzpicture}
\setlength\PreviewBorder{5pt}%
%%%>

\begin{comment}
:Title: Plane partition

Illustration of a `plane partition`_.

.. _plane partition: http://mathworld.wolfram.com/PlanePartition.html

\end{comment}
% Three counters
\newcounter{x}
\newcounter{y}
\newcounter{z}

% The angles of x,y,z-axes
\newcommand\xaxis{210}
\newcommand\yaxis{-30}
\newcommand\zaxis{90}
\newcommand\cubeline{none}

% The top side of a cube
\newcommand\topside[3]{
  \fill[fill=yellow, draw=\cubeline,shift={(\xaxis:#1)},shift={(\yaxis:#2)},
  shift={(\zaxis:#3)}] (0,0) -- (30:1) -- (0,1) --(150:1)--(0,0);
}

% The left side of a cube
\newcommand\leftside[3]{
  \fill[fill=red, draw=\cubeline,shift={(\xaxis:#1)},shift={(\yaxis:#2)},
  shift={(\zaxis:#3)}] (0,0) -- (0,-1) -- (210:1) --(150:1)--(0,0);
}

% The right side of a cube
\newcommand\rightside[3]{
  \fill[fill=blue, draw=\cubeline,shift={(\xaxis:#1)},shift={(\yaxis:#2)},
  shift={(\zaxis:#3)}] (0,0) -- (30:1) -- (-30:1) --(0,-1)--(0,0);
}

% The cube 
\newcommand\cube[3]{
  \topside{#1}{#2}{#3} \leftside{#1}{#2}{#3} \rightside{#1}{#2}{#3}
}

% Definition of \planepartition
% To draw the following plane partition, just write \planepartition{ {a, b, c}, {d,e} }.
%  a b c
%  d e
\newcommand\planepartition[1]{
 \setcounter{x}{-1}
  \foreach \a in {#1} {
    \addtocounter{x}{1}
    \setcounter{y}{-1}
    \foreach \b in \a {
      \addtocounter{y}{1}
      \setcounter{z}{-1}
      \foreach \c in {1,...,\b} {
        \addtocounter{z}{1}
        \cube{\value{x}}{\value{y}}{\value{z}}
      }
    }
  }
}

\begin{document} 

\begin{tikzpicture}
%\planepartition{{5,3,2,2},{4,2,2,1},{2,1},{1}}
%E
\cube{{-8}}{{8}}{{0}}
\cube{{-8}}{{9}}{{0}}
\cube{{-8}}{{10}}{{0}}
\cube{{-8}}{{8}}{{1}}
\cube{{-8}}{{8}}{{2}}
\cube{{-8}}{{9}}{{2}}
\cube{{-8}}{{10}}{{2}}
\cube{{-8}}{{8}}{{3}}
\cube{{-8}}{{8}}{{4}}
\cube{{-8}}{{9}}{{4}}
\cube{{-8}}{{10}}{{4}}

%G
\cube{{-6}}{{6}}{{0}}
\cube{{-6}}{{7}}{{0}}
\cube{{-6}}{{8}}{{0}}
\cube{{-6}}{{6}}{{1}}
\cube{{-6}}{{8}}{{1}}
\cube{{-6}}{{6}}{{2}}
\cube{{-6}}{{7}}{{2}}
\cube{{-6}}{{8}}{{2}}
\cube{{-6}}{{6}}{{3}}
\cube{{-6}}{{6}}{{4}}
\cube{{-6}}{{7}}{{4}}
\cube{{-6}}{{8}}{{4}}


%A
\cube{{-4}}{{4}}{{0}}
\cube{{-4}}{{4}}{{1}}
\cube{{-4}}{{4}}{{2}}
\cube{{-4}}{{5}}{{2}}
\cube{{-4}}{{4}}{{3}}
\cube{{-4}}{{4}}{{4}}
\cube{{-4}}{{5}}{{4}}
\cube{{-4}}{{6}}{{0}}
\cube{{-4}}{{6}}{{1}}
\cube{{-4}}{{6}}{{2}}
\cube{{-4}}{{6}}{{3}}
\cube{{-4}}{{6}}{{4}}

%T
\cube{{-2}}{{3}}{{0}}
\cube{{-2}}{{3}}{{1}}
\cube{{-2}}{{3}}{{2}}
\cube{{-2}}{{3}}{{3}}
\cube{{-2}}{{2}}{{4}}
\cube{{-2}}{{3}}{{4}}
\cube{{-2}}{{4}}{{4}}

% S
\cube{{0}}{{0}}{{0}}
\cube{{0}}{{1}}{{0}}
\cube{{0}}{{2}}{{0}}
\cube{{0}}{{2}}{{1}}
\cube{{0}}{{0}}{{2}}
\cube{{0}}{{1}}{{2}}
\cube{{0}}{{2}}{{2}}
\cube{{0}}{{0}}{{3}}
\cube{{0}}{{0}}{{4}}
\cube{{0}}{{1}}{{4}}
\cube{{0}}{{2}}{{4}}
\end{tikzpicture}

\end{document} 
