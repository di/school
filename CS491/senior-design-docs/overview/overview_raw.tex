
The goal of this project is to automate the passive discovery, recognition,
and consumption of semantic web services.  ServiceSniffer does not require any
central registry of services but will examine network traffic patterns to
identify specific services, how to invoke them, and report specific, user
chosen information.  As well as running with known service definitions, the
system should be able to identify new services and analyze relevant network
traffic to determine how to consume the service.  This application will be
constructed of an extensible core framework upon which future modules can be
developed and integrated.
\\\\
ServiceSniffer is split into 3 general pieces:  the data input, data
processing, and user interface modules.  The data input module is responsible
for collecting traffic off the network or using a Pcap file as a source instead
of a live network.  The data processing module, which is constructed from
various filters and processors (which will be described in more depth later),
applies filters to the network traffic from the data input module and converts
it into various and configurable reports and views depending on what the user
wants to see.  Some examples of these reports will be described later.  Lastly,
ServiceSniffer has both a CLI and a GUI for viewing the various data returned
by the data processors and filters.  These interfaces will allow users to
interact with the data returned, however the GUI will offer more functionality,
such as additional display filters (e.g. viewing traffic with a specific
destination IP).  Both interfaces will offer to users a report of what web
services appear to be available on the network and supply a report - an ad-hoc
registry of sorts - which users can use to consume the web service.
\\\\
Users of ServiceSniffer are chunked into 3 loose groups:  system/network
administrators who wish to control all of the ServiceSniffer pieces (data
input, data processing, and the user interface), developers who wish to or are
tasked with developing custom data filters and/or processors so users can have
new/custom reports, and UI users who only wish to analyze the data collected
and processed by ServiceSniffer.  Some examples of uses follow:


\subsection*{What web services travel over my network?}
Network administrators, among other technical people, want to know what web
services are being used on their network.  Some users may not report that
they're using specific web services on a network when they are, which may cause
conflicts with other services.  By being able to see all of the web service
invocations on a network, an administrator can determine where the rogue web
services are being used and better determine whether they offer something
useful on the network.  


\subsection*{How do I consume web services on a network that doesn't have a web
             service registry?}
Not all networks have a central registry of web services offered on the
network, and most don't have a record of what web services from outside the
network are consumed by users on the network.  More, many web services aren't
clearly defined in WSDL or an equivalent format, so it's unclear how to consume
the web service without a formal definition.  ServiceSniffer constructs an
ad-hoc registry of web services is observes being consumed in a defined way.
This registry can then be used to let users consume these previously undefined
web services.


\subsection*{Which services on my network are secure/insecure?}
Some networks have a requirement that all services being used on them be
encrypted as the network itself cannot be trusted.  ServiceSniffer is able to
show which services being used on a network are secure (either fully encrypted
traffic or encrypted data with cleartext headers) and which are insecure.  By
looking at the specific traffic an administrator can learn what pieces of the
network are vulnerable.


\subsection*{Who uses a specific web service?}
Is a web service actively being used?  Is there a pattern to who uses a
specific web service, or a pattern to when it's used?  Web service developers,
as well as network administrators, frequently want to know what users and/or
servers are using specific web services and when.  This allows them to
determine to whom a specific service is useful, who uses an unautherized web
service, and other similar situations.


\subsection*{What data is being sent during use of a specific web service?}
In debugging web services, developers need to see what data is being sent to
the web service during invocation and usage.  Instead of using a program like
Wireshark to have to look at the packets of a web service invocation,
developers can use ServiceSniffer to see where, if anywhere, the data actually
sent differs from what should be sent.

\subsection*{Is there a pattern to web service invocations?  Is anyone not
             following that pattern?}
Some web services are frequently used in discrete patterns.  Sometimes these
patterns suggest a specific usage, even if the data sent is encrypted (but the
headers are unencrypted).  For example, EXAMPLE HERE

