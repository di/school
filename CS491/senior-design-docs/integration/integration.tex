\documentclass[titlepage]{article}

\usepackage[table]{xcolor}
\usepackage{fullpage}   % Use the whole page
\usepackage{fancyhdr}   % Nice headers/footers
\usepackage{graphicx}   % Importing graphics
\usepackage{mdwlist}    % For itemize* and enumerate*
\usepackage{hyperref}   % Hyperlink references and URLs
\usepackage{vhistory}   % Version History
\hypersetup{colorlinks=true}
\usepackage{tabularx}
\usepackage{todonotes}

\hypersetup{
    colorlinks=true,
    linkcolor=blue,
    urlcolor=blue,
    pdftitle={ServiceSniffer Integration Test Plan},
    pdfauthor={Justin Courts; Philip Cristiano; Charles Rumford; Thomas Wambold}
}

% So we can number paragraphs too
\setcounter{secnumdepth}{5}

% For headers and footers
\setlength{\headheight}{15pt}
\setlength{\headsep}{25pt}
\pagestyle{fancy}
	
% Page style for the title page
\fancypagestyle{plain}{
    \fancyhf{}
    \renewcommand{\headrulewidth}{0pt}
    \renewcommand{\footrulewidth}{0pt}
}

% Just so we don't have to specify this twice
\newcommand\mytitle{Integration Test Plan}
\newcommand\mydate{April 6, 2010}

% Page style for every other page
\fancyhf{} % clear all header and footer fields
\fancyhead[L]{ServiceSniffer}
\fancyhead[C]{\mytitle}
\fancyhead[R]{\mydate}
\fancyfoot[C]{\thepage}
\renewcommand{\headrulewidth}{0.4pt}
\renewcommand{\footrulewidth}{0.4pt}

\title{\textbf{\mytitle}}
\author{
    Justin Courts \\\url{justin@servicesniffer.net}
    \and Philip Cristiano \\\url{phil@servicesniffer.net}
    \and Charles Rumford \\\url{charlesr@servicesniffer.net}
    \and Thomas Wambold \\\url{tom@servicesniffer.net}
}
\date{\mydate}

\newcommand{\testentry}[4]{
    \subsubsection{#1}
    \begin{center}
    \begin{tabular}{| l p{0.7\textwidth}|}
        \hline
        \bf Pre-Condition & #2 \\
        \bf Action & #3 \\
        \bf Post-Condition & #4 \\\hline
    \end{tabular}
    \end{center}
}

\begin{document}
\pagenumbering{roman}

% Logo
\begin{figure}
    \vspace{-6em}
    \centering
    \includegraphics[width=0.6\textwidth]{../logo}
    \vspace{-4em}
\end{figure}

\maketitle

\begin{abstract}
The goal of this project is to automate the passive discovery, recognition,
and consumption of semantic web services.  ServiceSniffer does not require any
central registry of services but will examine network traffic patterns to
identify specific services, how to invoke them, and report specific, user
chosen information.  As well as running with known service definitions, the
system should be able to (a) identify new services as unknown, and (b) analyze
relevant network traffic to determine how to consume the service.  This
application will be constructed of an extensible core framework upon which
future modules can be developed and integrated.
\end{abstract}

\pagebreak

\begin{versionhistory}
\vhEntry{1.0}{02-09-2010}{All}{Initial release document}
\vhEntry{2.0}{04-06-2010}{All}{Final release document}
\end{versionhistory}

\pagebreak

\setcounter{tocdepth}{4}
\tableofcontents
\pagebreak
\pagenumbering{arabic}

\rowcolors{1}{lightgray}{white}

\section{Introduction}

%------------------------------------------------------------------------------

\subsection{Purpose}
This document lays out the integration test plan for ServiceSniffer.  It
defines the test cases for the interaction between the four modules of the
system: Data Capture, Filters \& Processors, Kernel, and User Interface.  The
integration test plan for this system is based on the project requirements set
forth in the Software Requirements Specification document while taking into
consideration the system architecture as laid out in the Software Design
Specification.

%------------------------------------------------------------------------------

\subsection{Definitions, Acronyms, and Abbreviations%
  \label{definitions}%
}

\begin{description}
\item[Consumption of a Service]
    To become a user/client of the web service.

\item[Filter]
    A tool that returns a subset of its input, based on some parameters.

\item[JSON]
    JavaScript Object Notation.

\item[libpcap]
    Multiplatform packet capture library.

\item[PCAP]
    Packet CAPture.  An API for capturing and processing packets from live
    network devices, and files.

\item[Processor]
    A tool that transforms its input from one data type to another.

\item[Semantic Service]
    A web service which defines a context for the meaning of the data which it
    returns.

\item[Web Service]
    A software system designed to support interoperable machine-to-machine
    interaction over a network.

\item[WSDL]
    Web Services Description Language.  A format for describing web services.
\end{description}

%------------------------------------------------------------------------------

\subsection{References%
  \label{references}%
}
    For the project requirements for the ServiceSniffer project, please see the
    Software Requirements Specification and Software Design Specification
    documents for this project.  The most recent versions of these documents
    are available at \url{http://www.servicesniffer.net}.  Other information
    can be found at these locations:

\begin{itemize*}
    \item WSDL Specification: \url{http://www.w3.org/TR/wsdl20/}
    \item SOAP Specification: \url{http://www.w3.org/TR/soap/}
    \item libpcap Homepage: \url{http://www.tcpdump.org/}
    \item MVP-E Project (Team 10): \url{http://www.cs.drexel.edu/SeniorDesign/2010Material/Projects2010.html}
\end{itemize*}

%------------------------------------------------------------------------------

\section{Data Capture}
The data capture module brings in the raw data that is later processed. The
data can go do either a file or the system kernel. These interfaces must be
tested to make sure that data is correctly written to file or handed off to the
system kernel.

%------------------------------------------------------------------------------

\subsection{File}

\testentry{Live Capture Web Service Data}
    {The capture process is running in live capture mode.}
    {A web service request is made over the network interface.}
    {The process will capture the web service data.}
     
\testentry{Offline Capture of Web Service Data}
    {A PCAP file exists with web service data.}
    {The capture process is started in offline mode with the filename.}
    {The process will capture the web service data.}

\testentry{Valid File}
    {The program is not running.}
    {The user starts the program and supplies an existing file that they have
        permissions to read and that is in the PCAP format.}
    {The program is started and loads the input file.}

\testentry{Invalid File}
    {The program is not running.}
    {The user supplies a file, which does not exist, the user does not have
        permission to read, or is not in PCAP format.}
    {An error message is displayed, and the program is not started.}

\testentry{Valid Interface}
    {The program is not running.}
    {The user supplies a network interface, which exists and the user has
        permission to capture from it.}
    {The program is started, and reads from the network device.}

\testentry{Invalid Interface}
    {The program is not running.}
    {The user supplies a network interface, which does not exist or the user
        does not have permission to capture from it.}
    {An error message is displayed, and the program is not started.}

\testentry{Valid Output File Location}
    {The program is not running.}
    {The user starts the program and supplies a file name in a location where
        the user has write permissions.}
    {The program is started, and it outputs data connections of service
        invocations to the file in PCAP format.}

\testentry{Invalid Output File Location}
    {The program is not running.}
    {The user starts the program and supplies a file name in a location where
        the user does not have write permissions.}
    {An error message is displayed, and the program is not started.}

\testentry{Output File Consistency}
    {The program is running.}
    {The program is closed abruptly.}
    {The output file is a valid PCAP file.}

%------------------------------------------------------------------------------

\subsection{Kernel}

\testentry{UNIX Socket}
    {The data capture module is functioning correctly.}
    {The data capture module sends data over the socket to the kernel.}
    {The kernel receives the message over an established UNIX domain socket.}

\testentry{Message Container}
    {The data capture module is functioning correctly.}
    {The data capture module sends data over the socket to the kernel.}
    {The data is formatted in a JSON container.}

\testentry{PCAP Data}
    {The data capture module is functioning correctly.}
    {The data capture module sends data over the socket to the kernel.}
    {The data in the JSON container is of PCAP format.}

\testentry{Web Service Data Only}
    {The data capture module is functioning correctly.}
    {The data capture module sends data over the socket to the kernel.}
    {The data only contains web service data.}

%------------------------------------------------------------------------------

\section{Filters \& Processors}
The filters and processors module is the heart of the processing of the data
into a user-friendly format.  Filters and processors are created by the kernel
and communicate with either the kernel or other filters/processors.
Filters and processors can be chained together through their input and output
sockets, allowing for unlimited possibilities when processing data.  All filter
and processor chains must be acyclical and have their final output socket be
the kernel.

%------------------------------------------------------------------------------

\subsection{Other Filters \& Processors}

\testentry{UNIX Socket}
    {The data capture, kernel, and filter \& processor modules are functioning correctly.}
    {The filter \& processor module sends data over the socket to another filter or processor.}
    {The processor or filter receives the data over a UNIX domain socket.}

\testentry{Message Container}
    {The data capture, kernel, and filter \& processor modules are functioning correctly.}
    {The filter \& processor module sends data over the socket to another filter or processor.}
    {The received data is in a JSON container.}

\testentry{Message Format}
    {The data capture, kernel, and filter \& processor modules are functioning correctly.}
    {The filter \& processor module sends data over the socket to another filter or processor.}
    {The JSON container contains the message type, name, and data.}

\testentry{Chaining}
    {The data capture, kernel, and filter \& processor modules are functioning correctly.}
    {A acyclic chain of filters or processors are running.}
    {The output of the final filter or processor is returned to the kernel.}

\testentry{Processor Can Use Sockets}
    {Starting the processor with command line arguments including the path to a
        UNIX file socket.}
    {An initialization message is sent to the processor.}
    {The processor will read the socket and respond after initialization} 

\testentry{Secured Connections}
    {The program is running and has valid input.}
    {An encrypted connection is passed into a processor.}
    {The connection is marked as encrypted, and unable to determine if it is a
        web service, and will be reported to the user.}

\testentry{Unsecured Connections - Web Service}
    {The program is running and has valid input.}
    {An unencrypted connection is passed into a processor, and it is a web
        service.}
    {If the web service has not been seen before, it is added to a list
        containing: server name, server IP address, type of service (SOAP,
        etc.), service definition (WSDL if possible).  The invocation of the
        web service is added to a list containing: client IP address, timestamp
        of the invocation, and the data sent to the service.}

\testentry{Unencrypted Connections - Non-web Service}
    {The program is running and has valid input.}
    {An unencrypted connection is passed into a processor, and it is not a web
        service.}
    {The program ignores the connection.}

\testentry{Secured Web Service}
    {The program is running and has valid input.}
    {An unencrypted connection is passed into a processor, and it contains an
        encrypted web service payload.}
    {The web service is added to a list of web services, it is marked as
        secured, and is reported to the user.}

\testentry{Popular Web Services}
    {The program is running and has valid input.}
    {A web service is invoked more frequently than others, given a user defined
        threshold.}
    {The web service is marked as ``popular.''}

\testentry{Unpopular Web Services}
    {The program is running and has valid input.}
    {A web service is invoked less frequently than others, given a user defined
        threshold.}
    {The web service is marked as ``unpopular.''}

\testentry{Useful Web Service Combinations}
    {The program is running and has valid input.}
    {A number of web service invocations occur in a noticeable pattern.}
    {The combination is noted in a list and presented to the user.}

\testentry{Abnormal Web Service Combinations}
    {The program is running and has valid input.}
    {A set of web service invocations occur which deviates from known
        patterns.}
    {The combination is noted in a list and presented to the user.}

\testentry{Service Source IP Filter}
    {The program is running and has valid input.}
    {A user filters based on source IP.}
    {A service list subset containing only web services from this source IP is
        returned.}
 
\testentry{Service Destination IP Filter}
    {The program is running and has valid input.}
    {A user filters based on destination IP.}
    {A service invocation list subset containing only invocations from a
        particular IP address is returned.}

\testentry{Service Filter}
    {The program is running and has valid input.}
    {A user filters based on a specific web service.}
    {The web service invocation list containing invocations of a specific web
        service.}

%------------------------------------------------------------------------------

\subsection{Kernel}

\testentry{UNIX Socket}
    {The data capture, kernel, and filter \& processor modules are functioning correctly.}
    {The filter \& processor module sends data over the socket to the kernel.}
    {The data is sent over a UNIX domain socket.}

\testentry{Message Container}
    {The data capture, kernel, and filter \& processor modules are functioning correctly.}
    {The filter \& processor module sends data over the socket to the kernel.}
    {The data is formatted in a JSON container.}

\testentry{Message Format}
    {The data capture, kernel, and filter \& processor modules are functioning correctly.}
    {The filter \& processor module sends data over the socket to the kernel.}
    {The JSON container contains the message type, name, and data.}

%------------------------------------------------------------------------------

\section{Kernel}

The kernel is the central part of the system with many different interfaces to
other parts of the system. Being the central part of the system, the kernel
knows what is going on through out the system and be able to reliably
communicate with all of them.


%------------------------------------------------------------------------------

\subsection{Capture}

\testentry{Process Initiation}
    {The kernel is functioning correctly.}
    {The capture module process is started with configuration parameters.}
    {The capture module is functioning properly.}

%------------------------------------------------------------------------------

\subsection{Filters \& Processors}

\testentry{Process Initiation}
    {The kernel is functioning correctly.}
    {A filter or processor process is started with configuration parameters.}
    {The filter or processor is functioning properly.}

\testentry{UNIX Socket}
    {The data capture \& kernel modules are functioning correctly.}
    {The kernel module sends data over the socket to the filters or processors.}
    {The data is sent over a UNIX domain socket.}

\testentry{Message Container}
    {The data capture \& kernel modules are functioning correctly.}
    {The kernel module sends data over the socket to the filters or processors.}
    {The data is formatted in a JSON container.}

\testentry{PCAP Data}
    {The data capture \& kernel modules are functioning correctly.}
    {The kernel module sends data over the socket to the filters or processors.}
    {The data in the JSON container is of PCAP format.}

\testentry{Starting Filters \& Processors}
    {The data capture \& kernel modules are functioning correctly.}
    {The kernel starts a new filter or processor process, giving it an input and output socket.}
    {The filter or processor reads from its input socket and outputs to its output socket.}

\testentry{Stopping Filters \& Processors}
    {The data capture \& kernel modules are functioning correctly.}
    {The kernel stops a previously running filter or processor process.}
    % TODO: Do something different
    {The filter or processor's resources are reclaimed.}

%------------------------------------------------------------------------------

\subsection{User Interfaces}

\testentry{Process Initiation}
    {The kernel is functioning correctly.}
    {The user interface process is started with configuration parameters.}
    {The user interface module is functioning properly.}

\testentry{UNIX Socket}
    {The data capture, kernel, and filter \& processor modules are functioning correctly.}
    {The kernel module sends data over the socket to the user interface.}
    {The data is sent over a UNIX domain socket.}

\testentry{Message Container}
    {The data capture, kernel, and filter \& processor modules are functioning correctly.}
    {The kernel module sends data over the socket to the user interface.}
    {The data is formatted in a JSON container.}

%------------------------------------------------------------------------------

\section{User Interfaces}

The user interfaces are the user-facing modules of the system.  The user interfaces only
communicate with the system kernel in order to set up new filters or processors
and to receive their outputs.  The user interfaces then are responsible for
displaying the output to the user.

%------------------------------------------------------------------------------

\subsection{Kernel}

\testentry{UNIX Socket}
    {The data capture, kernel, and filter \& processor modules are functioning correctly.}
    {The user interface module sends data over the socket to the kernel.}
    {The data is sent over a UNIX domain socket.}

\testentry{Message Container}
    {The data capture, kernel, and filter \& processor modules are functioning correctly.}
    {The user interface module sends data over the socket to the kernel.}
    {The data is formatted in a JSON container.}

\testentry{Message Format}
    {The data capture, kernel, and filter \& processor modules are functioning correctly.}
    {The user interface module sends data over the socket to the kernel.}
    {The JSON container contains the message type, name, and data.}

\testentry{Starting Filters \& Processors}
    {The data capture, kernel, and filter \& processor, and user interface modules
        are functioning correctly.}
    {The user interface sends a command to the kernel to start a particular filter
        or processor with a particular input socket.}
    {The filter or processor is started with the correct input socket, and the
        output socket set to the kernel.}

\testentry{Stopping Filters \& Processors}
    {The data capture, kernel, and filter \& processor, and user interface modules
        are functioning correctly.}
    {The user interface sends a command to the kernel to stop a particular filter
        or processor.}
    {The filter or processor is stopped.}

\testentry{System Shutdown}
    {The data capture, kernel, and filter \& processor, and user interface modules
        are functioning correctly.}
    {The user interface sends a command to the kernel shut down the whole system}
    {The system exits.}
    
\testentry{Selecting a processor}
    {The program is running, and is displaying an user interface.}
    {The user selects a processor to display.}
    {The processor output is displayed.}

\testentry{Selecting a display filter}
    {The program is running, displaying an user interface, and displaying processor
        output.}
    {The user selects a display filter to activate.}
    {The processor output is filtered, and the results are displayed.}

\testentry{Deselecting a display filter}
    {The program is running, displaying an user interface, and displaying processor
        output.}
    {The user selects an activated display filter to deactivate.}
    {The processor output is displayed without using the deactivated filter.}

\testentry{Selecting Multiple display filters}
    {The program is running, displaying an user interface, displaying processor
        output, and already has an activated filter.}
    {The user selects another filter to activate.}
    {The filtered processor output is further filtered in the order the filters
        were applied, and displayed.}

\testentry{Deselecting Multiple display filters}
    {The program is running, displaying an user interface, displaying processor
        output, and already has an activated filter.}
    {The user selects another filter to deactivate.}
    {The processor output is displayed without using the deactivated filter.}

\end{document}
