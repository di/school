\documentclass[titlepage]{article}

\usepackage{fullpage}   % Use the whole page
\usepackage{fancyhdr}   % Nice headers/footers
\usepackage{graphicx}   % Importing graphics
\usepackage{mdwlist}    % For itemize* and enumerate*
\usepackage{hyperref}   % Hyperlink references and URLs
\usepackage{vhistory}   % Version History
\usepackage{tabularx}
%\usepackage{todonotes}

\hypersetup{
    colorlinks=true,
    linkcolor=blue,
    urlcolor=blue,
    pdftitle={ServiceSniffer Acceptance Test Plan},
    pdfauthor={Justin Courts; Philip Cristiano; Charles Rumford; Thomas Wambold}
}

% So we can number paragraphs too
\setcounter{secnumdepth}{5}

% For headers and footers
\setlength{\headheight}{15pt}
\setlength{\headsep}{25pt}
\pagestyle{fancy}
	
% Page style for the title page
\fancypagestyle{plain}{
    \fancyhf{}
    \renewcommand{\headrulewidth}{0pt}
    \renewcommand{\footrulewidth}{0pt}
}

% Just so we don't have to specify this twice
\newcommand\mytitle{Acceptance Testing}
\newcommand\mydate{December 1, 2009}

% Page style for every other page
\fancyhf{} % clear all header and footer fields
\fancyhead[L]{ServiceSniffer}
\fancyhead[C]{\mytitle}
\fancyhead[R]{\mydate}
\fancyfoot[C]{\thepage}
\renewcommand{\headrulewidth}{0.4pt}
\renewcommand{\footrulewidth}{0.4pt}

\title{\textbf{\mytitle}}
\author{
    Justin Courts \\\url{justin@servicesniffer.net}
    \and Philip Cristiano \\\url{phil@servicesniffer.net}
    \and Charles Rumford \\\url{charlesr@servicesniffer.net}
    \and Thomas Wambold \\\url{tom@servicesniffer.net}
}
\date{\mydate}

%\newcommand{\testentry}[4]{
%    \begin{itemize*}
%        \item \textbf{Name:} #1
%        \item \textbf{Pre-Condition:} #2
%        \item \textbf{Action:} #3
%        \item \textbf{Post-Condition:} #4
%    \end{itemize*}
%}

%\newcommand{\testentry}[4]{
%    \begin{center}
%    \begin{tabularx}{0.75\textwidth}{| l | X |}
%        \hline
%        \bf Name & #1 \\\hline
%        \bf Pre-Condition & #2 \\\hline
%        \bf Action & #3 \\\hline
%        \bf Post-Condition & #4 \\\hline
%    \end{tabularx}
%    \end{center}
%}

\newcommand{\testentry}[4]{
    \subsubsection{#1}
    \begin{itemize}
        \item \textbf{Pre-Condition:} #2
        \item \textbf{Action:} #3
        \item \textbf{Post-Condition:} #4
    \end{itemize}
    \vspace{0.5em}
}

\begin{document}
\pagenumbering{roman}

% Logo
\begin{figure}
    \vspace{-6em}
    \centering
%    \includegraphics[width=0.6\textwidth]{../logo}
    \vspace{-4em}
\end{figure}

\maketitle

\begin{abstract}
The goal of this project is to automate the passive discovery, recognition,
and consumption of semantic web services.  ServiceSniffer does not require any
central registry of services but will examine network traffic patterns to
identify specific services, how to invoke them, and report specific, user
chosen information.  As well as running with known service definitions, the
system should be able to (a) identify new services as unknown, and (b) analyze
relevant network traffic to determine how to consume the service.  This
application will be constructed of an extensible core framework upon which
future modules can be developed and integrated.
\end{abstract}

\pagebreak

\begin{versionhistory}
\vhEntry{1.0}{01-12-2009}{All}{Initial release document}
\vhEntry{2.0}{12-01-2009}{All}{Updated with examples of usage}
\vhEntry{3.0}{16-02-2009}{All}{Final document}
\end{versionhistory}

\pagebreak

\setcounter{tocdepth}{4}
\tableofcontents
\pagebreak
\pagenumbering{arabic}

% ACCEPTANCE TEST PLAN DOCUMENT

%The acceptance test plan describes the tests that will be executed when you
%have delivered the finished product. If your product passes this test it will
%be considered a success, otherwise a failure.
%
%A simple litmus test may be applied to your Acceptance Test Plan document to
%determine if it is complete. Imagine you are the person/group contracting out
%your project to a third party. The tests listed in the document must be
%complete enough to encompass every requirement of your system. If a system
%passes all of these tests, they you as the contracting agent will be required
%to pay in full for the project. Therefore, if an explicit or implicit
%requirement is not tested, it is possible for you to receive a incomplete
%project.
%
%The team administrator submits the acceptance test plan in PDF format via
%WebCT.

%\testentry{Foobar}
%    {The foo is barred.}
%    {We perform bazzing.}
%    {It works.}

\section{Introduction}

\subsection{Objective}
    This document includes the acceptance test plan for ServiceSniffer.  It
    defines the test cases for the modules of the system:  Data Capture,
    Processors and Filters, the System Kernel, and Interfaces.  The acceptance
    test plan for this system is based on the project requirements set forth in
    the Software Requirements Specification document.

\subsection{References}
    For the project requirements for the ServiceSniffer project, please see the
    Software Requirements Specification document for this project.  The most
    recent version of this document is available at
    \url{http://www.servicesniffer.net}

\section{Tests}

\subsection{Data Capture}

The Data Capture section of the system is comprised of the data capturing
interface (from a file, file handles, or the network card) and an output of a
PCAP stream.  There is also the option of saving the PCAP stream to a file.

\testentry{Valid Input File}{
    The program is not running.
}{
    The user starts the program, and supplies an existing input file, that they
    have permissions to read, and that is in the PCAP format.
}{
    The program is started, and loads the input file.
}

\testentry{Invalid Input File}{
    The program is not running.
}{
    The user supplies an input file, which does not exist, or the user does not
    have permission to read, or is not in PCAP format.
}{
    An error message is displayed, and the program is not started.
}

\testentry{Valid Interface}{
    The program is not running.
}{
    The user supplies a network interface, which exists, and the user has
    permission to capture from it.
}{
    The program is started, and reads from the network device.
}

\testentry{Invalid Network Interface}{
    The program is not running.
}{
    The user supplies a network interface, which does not exist, or the user
    does not have permission to capture from it.
}{
    An error message is displayed, and the program is not started.
}

\testentry{Valid Output File Location}{
    The program is not running.
}{
    The user starts the program and supplies a file name in a location where
    the user has write permissions and the file can be written to by the user.
}{
    The program is started, and it outputs data connections of service
    invocations to the file in PCAP format.
}

\testentry{Invalid Output File Location}{
    The program is not running.
}{
    The user starts the program and supplies a file name in a location where
    the user does not have write permissions, or a file which otherwise the
    user cannot write to.
}{
    An error message is displayed, and the program is not started.
}

\testentry{Output File Consistency}{
    The program is running.
}{
    The program is closed abruptly for any reason.
}{
    The output file is a valid PCAP file.
}

\subsection{Processors/Filters}

The Processors/Filters section of the system is comprised of the data filters,
which reduce the incoming information to subsets, and data processors, which
take filtered or unfiltered data and create meaningful data structures (such as
lists and graphs) for the user to later consume in the Display section.
\\
A precondition for all processors and filters is that the program is running,
and that it has valid input.

\testentry{Secured Connections}{
    The program is running and has valid input.
}{
    An encrypted connection is passed into a processor.
}{
    The connection is marked as encrypted, and unable to determine if it is a
    web service, and will be reported to the user.
}

\testentry{Unsecured Connections - Web Service}{
    The program is running and has valid input.
}{
    An unencrypted connection is passed into a processor, and it is a web
    service.
}{
    If the web service has not been seen before, it is added to a list
    containing: server name, server IP address, type of service (SOAP, etc.),
    service definition (WSDL if possible).

    The invocation of the web service is added to a list containing: client IP
    address, timestamp of the invocation, and the data sent to the service.
}

\testentry{Unencrypted Connections - Non-web Service}{
    The program is running and has valid input.
}{
    An unencrypted connection is passed into a processor, and it is not a web
    service.
}{
    The program ignores the connection.
}
    
\testentry{Secured Web Service}{
    The program is running and has valid input.
}{
    An unencrypted connection is passed into a processor, and it contains an
    encrypted web service payload.
}{
    The web service is added to a list of web services, it is marked as secured,
    and is reported to the user.
}

\testentry{Popular Web Services}{
    The program is running and has valid input.
}{
    A web service is invoked more frequently than others, given a user defined
    threshold.
}{
    The web service is marked as ``popular.''
}

\testentry{Unpopular Web Services}{
    The program is running and has valid input.
}{
    A web service is invoked less frequently than others, given a user defined
    threshold.
}{
    The web service is marked as ``unpopular.''
}

\testentry{Useful Web Service Combinations}{
    The program is running and has valid input.
}{
    A number of web service invocations occur in a noticeable pattern.
}{
    The combination is noted in a list and presented to the user.
}

\testentry{Abnormal Web Service Combinations}{
    The program is running and has valid input.
}{
    A set of web service invocations occur which deviates from known patterns.
}{
    The combination is noted in a list and presented to the user.
}

\testentry{Service Source IP Filter}{
    The program is running and has valid input.
}{
    A user filters based on source IP.
}{
    A service list subset containing only web services from this source IP is
    returned.
}

\testentry{Service Destination IP Filter}{
    The program is running and has valid input.
}{
    A user filters based on destination IP.
}{
    A service invocation list subset containing only invocations from a
    particular IP address is returned.
}

\testentry{Service Filter}{
    The program is running and has valid input.
}{
    A user filters based on a specific web service.
}{
    The web service invocation list containing invocations of a specific web
    service.
}

\subsection{Interfaces}

The Interfaces section of this system is where the user can specify what data
she wishes to see.  The filters and processors from the previous section create
the data that is given to the display layer, and the display is capable of
further filtering this data as outlined below.
\\
All displays (GUI, CLI) should conform to these tests.

\testentry{Selecting a processor}{
    The program is running, and is displaying an interface.
}{
    The user selects a processor to display.
}{
    The processor output is displayed.
}

\testentry{Selecting a display filter}{
    The program is running, displaying an interface, and displaying processor
    output.
}{
    The user selects a display filter to activate.
}{
    The processor output is filtered, and the results are displayed.
}

\testentry{Deselecting a display filter}{
    The program is running, displaying an interface, and displaying processor
    output.
}{
    The user selects an activated display filter to deactivate.
}{
    The processor output is displayed without using the deactivated filter.
}

\testentry{Selecting Multiple display filters}{
    The program is running, displaying an interface, displaying processor output,
    and already has an activated filter.
}{
    The user selects another filter to activate.
}{
    The filtered processor output is further filtered in the order the filters
    were applied, and displayed.
}

\testentry{Deselecting Multiple display filters}{
    The program is running, displaying an interface, displaying processor output,
    and already has an activated filter.
}{
    The user selects another filter to deactivate.
}{
    The processor output is displayed without using the deactivated filter.
}

%\paragraph{Valid input}
%        -X      : open a data stream from the network card directly
%        -X      : listen to a data stream coming in from stdin in Pcap format
%        -X file : open ``file'' which is in Pcap format
%        -X file : save in Pcap format to `file'
%
%
%    Pre:    App not running
%    Action: Starting app with {data stream} or {file} with an optional 
%    Post:   App starts running without requiring root access and saving to 
%
%    Pre:    App not running
%    Action: Starting app with invalid input
%    Post:   App does not start && App returns usage information
%    
%    Pre:    


\end{document}
