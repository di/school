\documentclass{article}

\newcommand\mydate{October 13, 2009}

\usepackage{fullpage}   % Use the whole page
\usepackage{fancyhdr}   % Nice headers/footers
\usepackage{lastpage}   % Macro for the last page
\usepackage{url}        % Deal with URLs better
\usepackage{graphicx}

% For headers and footers
\setlength{\headheight}{15pt}
\setlength{\headsep}{25pt}
\pagestyle{fancy}
	
% Page style for the title page
\fancypagestyle{plain}{
    \fancyhf{}
    \fancyfoot[C]{\thepage/\pageref{LastPage}}
    \renewcommand{\headrulewidth}{0pt}
    \renewcommand{\footrulewidth}{0pt}
}

% Page style for every other page
\fancyhf{} % clear all header and footer fields
\fancyhead[L]{ServiceSniffer}
\fancyhead[C]{Project Plan}
\fancyhead[R]{\mydate}
\fancyfoot[C]{\thepage/\pageref{LastPage}}
\renewcommand{\headrulewidth}{0.4pt}
\renewcommand{\footrulewidth}{0.4pt}

\title{\textbf{Project Plan}}
\author{
    Justin Courts \\\url{justin@servicesniffer.net}
    \and Philip Cristiano \\\url{phil@servicesniffer.net}
    \and Charles Rumford \\\url{charlesr@servicesniffer.net}
    \and Thomas Wambold \\\url{tom@servicesniffer.net}
}
\date{\mydate}

\begin{document}
\begin{figure}
    \vspace{-6em}
    \centering
    \includegraphics[width=0.6\textwidth]{../logo}
    \vspace{-4em}
\end{figure}
\maketitle

\begin{abstract}
The goal of this project is to automate the passive discovery, recognition, and
consumption of semantic web services.  The system will contain no central
registry but will examine network traffic patterns to identify specific
services and how to invoke them.  As well as running with known service
definitions, the system should be able to (a) identify new services as unknown,
and (b) analyze relevant network traffic to determine how to consume the
service.  This application will be constructed of an extensible core framework
upon which future modules can be developed and integrated.
\end{abstract}

\section{Project Framework}

The system will provide a framework to provide a structure to allow for
acquisition, processing, and reason of data and web services. This allows for
different modules to developed to process different types of data or process
them in different manners.  A common API will be provided between the
different layers that enables easy development pluggable modules.

Each part of the framework will have the following stages:
\begin{itemize}
    \item \textbf{Prototype} - A proof-of-concept of the component.
    \item \textbf{Initial Integration} - The component will be combined with
        other components.
    \item \textbf{Component Rewrite} - The component will be re-designed with
        information learned from prototyping and integration.
    \item \textbf{Final Integration} - All the components will be combined into
        the final system.
\end{itemize}

\subsection{Data Acquisition}

The Data Acquisition component component is responsible for pulling data from
the network for processing in subsequent components.  Also, preliminary
filtering of network data can be performed here.

\noindent\textbf{Lead:} Charles Rumford

\subsection{Protocol Identification}

The Protocol Identification component is responsible for taking the data
collected by the Data Acquisition component, and creating protocol-specific
data structures.  These data structures contain data at a higher level view,
rather than a single packet view.  For example, TCP streams could be
represented as a single logical object.

\noindent\textbf{Lead:} Justin Courts

\subsection{Semantic Reasoning}

The Semantic Reasoning component takes all the data collected and attempts to
extract useful information.  This could be the type of data in a stream
(weather data, etc.), or web service mappings.

\noindent\textbf{Lead:} Tom Wambold

\subsection{User Applications}

This could actually be a number of components that take the data from the other
components and actually does something useful with it.  This could be a number
of things:

\begin{itemize}
    \item Data visualization - Show mapping of network services
    \item Service configurator - Allow users to take advantage of public
        services
    \item Web service generator - Figure out how to utilize web services
        on-the-fly
\end{itemize}

\noindent\textbf{Lead:} Phil Cristiano

\section{Completion Dates}

\begin{itemize}
    \item Prototype: Mid-November (User Applications Late-November)
    \item Initial Integration: Late-December
    \item Component Rewrite: Mid-February
    \item Final Integration: Early-April
\end{itemize}

\end{document}
