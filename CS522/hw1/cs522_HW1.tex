\documentclass{article}
\usepackage{booktabs}
\usepackage{amsmath}
\usepackage{amssymb}
\usepackage[noend]{algorithmic}
\usepackage[nothing]{algorithm}
\usepackage{tikz}
\usepackage{latexsym}
\usepackage{float}
\providecommand{\e}[1]{\ensuremath{\times 10^{#1}}}
\renewcommand{\thealgorithm}{}
\title{CS 522: Data Structures and Algorithms II \\ Homework 1}
\author{Dustin Ingram}
\begin{document}
\maketitle
\begin{enumerate}
    \item \textbf{Solution:}
    If both \textsc{Increment} and \textsc{Decrement} operations were included
    in the $k$-bit counter, an amortized analysis of the cost of $n$ operations
    would cost as much as $\Theta(nk)$ time because we would no longer be able
    to consider each operation as a consecutive \textsc{Increment}, but rather
    as any combination of \textsc{Increment}s and \textsc{Decrement}s. Thus, in
    a worse-case scenario, it would be possible to alternate $n$ times between
    two operations which cost $O(k)$ each, resulting in a total cost of
    $\Theta(nk)$.

    \item \textbf{Solution:}
    last paragraph of 469

    \item \textbf{Solution:}

    \item \textbf{Solution:}

    \item \textbf{Solution:}

    \item \textbf{Solution:}

    \item \textbf{Solution:}

    \item \textbf{Solution:}
\end{enumerate}
\end{document}
