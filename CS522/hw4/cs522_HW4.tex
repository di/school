\documentclass{article}
\usepackage{booktabs}
\usepackage{amsmath}
\usepackage{amssymb}
\usepackage[noend]{algorithmic}
\usepackage[nothing]{algorithm}
\usepackage{tikz}
\usepackage{latexsym}
\usepackage{float}
\usepackage{mathrsfs}
\providecommand{\e}[1]{\ensuremath{\times 10^{#1}}}
\renewcommand{\thealgorithm}{}
\usetikzlibrary{arrows}
\title{CS 522: Data Structures and Algorithms II \\ Homework 4}
\author{Dustin Ingram}
\begin{document}
\maketitle
\begin{enumerate}
    \item \textbf{Solution:}
    We can reduce the subgraph-isomorphisim problem to the Hamiltonian cycle
    problem. We simply take the graph $G$, for which we want to test for a
    Hamiltonian cycle, and produce a second graph $G'$, which has the same
    number of vertices $n$ as in $G$, and is a simple cycle between all $n$
    nodes. If we can show that $G$ is isomorphic to a subgraph (here, the entire
    graph) of $G'$ in polynomial time, we have also shown that $G$ has a
    Hamiltonian cycle in polynomial time. However, since we know the Hamiltonian
    cycle problem to be NP-complete, we have therefore showed that the
    subgraph-isomorphism problem is NP-complete as well.

    \item \textbf{Solution:}
    To prove that the Hamiltonian-path problem is NP-complete, we must first
    show that we can solve the Hamiltonian-path problem in nondeterministic
    polynomial time. Given any graph $G$, we can solve the Hamiltonian-path
    problem in polynomial time by randomly choosing $n$ edges from $G$, where
    $n$ is the number of vertices in $G$. We can then check the path and verify
    that it visits all $n$ vertices, which can clearly be done in polynomial
    time.

    We then show that we can reduce a known NP-complete problem to the
    Hamiltonian-path problem. We use the Hamiltonian-cycle problem, which we
    assume to be proven NP-complete. By adding a single vertex to $G$ such that
    it is connected to all other vertices in $G$, it is clear that if there is a
    Hamiltonian path through the original vertices in $G$, then there is also a
    Hamiltonian cycle along this path, which trivially connects the endpoints of
    the path through the newly added vertex. Thus, the Hamiltonian-path problem
    is NP-complete.

    \item \textbf{Solution:}
    We give a simple recursive linear-time algorithm,
    \textsc{Greedy-Linear-Optimal-Tree-Vertex-Cover}, which visits each node
    exactly once, which takes as input the root of the entire tree $T$ and
    returns as output the same tree, with the optimal vertex cover represented
    by the vertices $v \in T$ for which $v.colored = true$.
    \begin{algorithm}[H]
        \begin{algorithmic}
            \floatname{algorithm}{Greedy-Linear-Optimal-Tree-Vertex-Cover($T$)}
            \caption{}
            \IF{$T.children = \emptyset$}
                \RETURN
            \ENDIF
            \FOR{$v \in T.children$}
                \STATE \textsc{Tree-Vertex-Cover($T$)}
                \IF{\NOT $T.colored$ \AND \NOT $v.colored$}
                    \STATE $T.colored = $ \TRUE
                \ENDIF
            \ENDFOR
        \end{algorithmic}
    \end{algorithm}

    \item \textbf{Solution:}
    No, the relationship between the vertex-cover problem and the clique problem
    does not give this implication. It is possible that any polynomial-time
    approximation algorithm with a constant approximation ratio will give a
    vertex covering which consists of all the nodes in a given graph $G$, thus,
    the complement of such an optimal vertex cover will not be a maximum-sized
    clique in the complement graph, but rather an empty set.

    \item \textbf{Solution:}
    If, for any pair of vertices $u, v$, $c(u,v) = 0$, either the triangle
    inequality must not hold, or the vertices must in fact be the same point, as
    there cannot be a zero-cost edge between any two non-identical vertices.

    \item \textbf{Solution:}
\end{enumerate}
\end{document}
