\documentclass{article}
\usepackage{booktabs}
\usepackage{amsmath}
\usepackage{amssymb}
\usepackage[noend]{algorithmic}
\usepackage[nothing]{algorithm}
\usepackage{tikz}
\usepackage{latexsym}
\usepackage{float}
\usepackage{mathrsfs}
\providecommand{\e}[1]{\ensuremath{\times 10^{#1}}}
\renewcommand{\thealgorithm}{}
\usetikzlibrary{arrows}
\title{CS 522: Data Structures and Algorithms II \\ Extra Credit 1}
\author{Dustin Ingram}
\begin{document}
\maketitle
\begin{enumerate}
    \item \textbf{17-1 Solution:}
    \begin{enumerate}
        \item The bit-reversal permutation algorithm:
        \begin{algorithm}[H]
        \begin{algorithmic}
        \floatname{algorithm}{Bit-Reversal-Permutation($A$)}
        \caption{}
        \FOR{$i \in 0\ldots n-1$}
        \IF{$rev_k(i) > i$}
        \STATE $swap(A[i], A[rev_{k}(i)]$
        \ENDIF
        \ENDFOR
        \end{algorithmic}
        \end{algorithm}
        \item This is nearly identical to a binary counter. The
        \textsc{Bit-Reversed-Increment} procedure simply flips every bit that is
        zero to a one and vice versa. The amortized analysis is the same and
        allows for the bit-reversal permutation of an $n$-element array in
        $O(n)$ time.
        \item Yes, it is possible, and is again similar to the binary counter.
        We would use the accounting method to make every shift ``pay forward''
        for a future, larger shift.
    \end{enumerate}
    \item \textbf{26-3 Solution:}
        \begin{enumerate}
            \item Because the cut $(S,T)$ is a finite capacity cut, only edges
            of the form $s\rightarrow A_{i}$ and $J_{i}\rightarrow t$ cross the
            cut, and since none of these edges can be infinite (a job cannot
            have infinite revenue or infinite cost), then if $J_{i}\in T$ then
            $A_{k}\in T$ for each $A_{k}\in R_{i}$.  \item The maximum net
            revenue, where $c_{G}$ is the capacity of the minimum cut of $G$ and
            corresponds to the cost per expert, is: $$
            \left(\sum_{i=1}^{m}{p_i}\right) - c_{G}$$ \item We will use a
            \textsc{Max-Flow} algorithm to find the maximum flow of the
            described flow network $G$. For every edge $(s, A_{i})$ which the
            algorithm selects, we hire expert $A_i$ and for every edge $(J_{i},
            t)$ which the algorithm selects, we accept job $J_i$. If we use the
            optimal max-flow algorithm, the complexity is $O(E\cdot f)$, where
            $E$ is the number of edges, and $f$ is the max flow. In this case,
            the maximum number of edges is:
            $$ E = m + mn + n $$
            Here, $f=r$. Thus the complexity of this algorithm is:
            $$ O(mnr) $$
        \end{enumerate}
    \end{enumerate}
\end{document}
