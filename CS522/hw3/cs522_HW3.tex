\documentclass{article}
\usepackage{booktabs}
\usepackage{amsmath}
\usepackage{amssymb}
\usepackage[noend]{algorithmic}
\usepackage[nothing]{algorithm}
\usepackage{tikz}
\usepackage{latexsym}
\usepackage{float}
\usepackage{mathrsfs}
\providecommand{\e}[1]{\ensuremath{\times 10^{#1}}}
\renewcommand{\thealgorithm}{}
\usetikzlibrary{arrows}
\title{CS 522: Data Structures and Algorithms II \\ Homework 3}
\author{Dustin Ingram}
\begin{document}
\maketitle
\begin{enumerate}
    \item \textbf{Solution:}
    First, consider $p_1$. The equation for the line that runs through this
    point and the origin would be:
    $$ y = \frac{y_1}{x_1}x $$
    Or,
    $$ 0 = x_1y-xy_1 $$
    Next, consider $p_2$. Any pair of coordinates $(x_2, y_2)$ which satisfy the
    previous equation would be on the line through $p_1$ and $(0,0)$ and thus
    would be neither clockwise nor counter-clockwise from $p_1$. There are two
    other cases:
    \begin{enumerate}
        \item{$x_1y_2-x_2y_1 > 0$}
        \item{$x_1y_2-x_2y_1 < 0$}
    \end{enumerate}
    We determine the orientation of $p_2$ using two additional points on the
    line:
    \begin{align*}
        (x_3, y_3) &= (x_2, \frac{y_1}{x_1}x_2) & (x_4, y_4) &= (\frac{x_1}{y_1}y_2, y_2)
    \intertext{The point $p_2$ will be clockwise from $p_1$ (and thus, $p_1$
    counter-clockwise from $p_2$) if:}
        x_2 &> x_4                  & y_3 &> y_2\\
        x_2 &> \frac{x_1}{y_1}y_2   & \frac{y_1}{x_1}x_2 &> y_2\\
        x_2y_1 &> x_1y_2 & x_2y_1 &> x_1y_2\\
        0 &> x_1y_2 - x_2y_1 & 0 &> x_1y_2 - x_2y_1
    \end{align*}
    We can use the same argument, in reverse, to prove counter-clockwise-ness.

    \item \textbf{Solution:}
    Given the set $S$ of points, and a point $p \in S$, we create a set of absolute
    values of the slope of each line between $p$ and every other point in $S$,
    and sort it. If there are two or more consecutive values, there are at least
    two or more points in the set collinear with $p$. Creating each sorted list
    takes $O(n\lg{n})$, which must be repeated for all $n$ points for a
    total complexity of $O(n^2\lg{n})$.

    \item \textbf{Solution:}
    For every disk in the set of $n$ disks, we compare it to every other disk in
    the set. Two disks intersect if the following is true:
    $$ r_1 + r_2 \geq \sqrt{|x_1 - x_2|^2 + |y_1 - y_2|^2} $$
    Since we only need to compare two disks once, we can reduce the complexity
    of this algorithm from $O(n^2)$ to $O(n\lg{n})$ by only comparing the
    current disk with those that have not been visited yet.

    \item \textbf{Solution:}
    The solution to this problem is given in it's entirety in the slides for
    Lecture 7.

    \item \textbf{Solution:}
    Let us assume that for some point $p \in \text{CH}(Q)$, there is some other
    point $q \not\in \text{CH}(Q)$ for which $\{p,q\}$ are the set of points
    most distant from each other. If this is true, then $q$ must lie within
    $\text{CH}(Q)$, and thus we can extend the line $\overline{pq}$ from $p$ through
    $q$ to some edge in $\text{CH}(Q)$. If the point where this line intersects
    with an edge in $\text{CH}(Q)$ is itself a vertex $\in \text{CH}(Q)$, then
    this vertex is collinear with $p$ and $q$ and is also clearly further from
    $p$ than $q$. Otherwise, the edge which intersects with this extended line
    \emph{must} be drawn between two points, at least one of which must be
    farther from $p$ than $q$. Thus, by contradiction, any two points which are
    the farthest from each other must be vertices of $\text{CH}(Q)$.

    \item \textbf{Solution:}
    For a new point $p \in Q$, we determine whether $p$ is inside or outside the
    existing convex hull $\text{CH}(Q)$ in $O(n)$ by comparing the polar
    angle of $p$ and every other point $q \in \text{CH}(Q)$. If the polar angle
    between $p$ and $q$ is either the largest or smallest between $q$ and it's
    two neighbors, $p$ must be outside the existing $\text{CH}(Q)$, and thus
    must be added to $\text{CH}(Q)$. Otherwise, $p$ is within $\text{CH}(Q)$ and
    we can ignore it. If we must add $p$ to $\text{CH}(Q)$, we iterate over all
    $q \in \text{CH}(Q)$ to determine the two points $r$ and $s$ which have the
    largest and smallest polar coordinates with $p$. We then add $p$ to
    $\text{CH}(Q)$, simultaneously removing any points which may exist between
    $r$ and $s$, as these points are now within the convex hull.

    \item \textbf{Solution:}
    \begin{enumerate}
        \item{} Here we use Jarvis' march to find the set of convex layers.
        Jarvis' march takes $O(nh)$, where $h$ is the number of vertices of each
        convex hull. Since each of the $i$ layers contains some subset of the
        $n$ points of size $l_i$, we know that:
        $$ \sum_{1}^{i}{nl_i} = n$$
        Thus the total complexity is:
        $$ O(nl_1 + nl_2 + \ldots + nl_i) = O(n^2) $$
        \item{} Here we simply show a linear-time reduction between the
        convex-layers problem and the sorting problem. Given a set of $n$ real
        numbers, which take $\Omega(n\lg{n})$ to sort, we can produce a set of three
        vertices which correspond to this real number, which also form a convex
        hull:
        $$ (n_i, 0), (-n_i, 0), (0, n_i) $$
        Since finding the ordered set of $n$ convex hulls in this graph would be
        analogous to ordering the set of $n$ real numbers, the reduction between
        the two would take linear time, and the model of computation requires
        $\Omega(n\lg{n})$ time to sort $n$ real numbers, we find that
        $\Omega(n\lg{n})$ time is also required to compute the convex layers of
        a set of $n$ points.
    \end{enumerate}
\end{enumerate}
\end{document}
