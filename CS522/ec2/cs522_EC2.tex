\documentclass{article}
\usepackage{booktabs}
\usepackage{amsmath}
\usepackage{amssymb}
\usepackage[noend]{algorithmic}
\usepackage[nothing]{algorithm}
\usepackage{tikz}
\usepackage{latexsym}
\usepackage{float}
\usepackage{mathrsfs}
\providecommand{\e}[1]{\ensuremath{\times 10^{#1}}}
\renewcommand{\thealgorithm}{}
\usetikzlibrary{arrows}
\title{CS 522: Data Structures and Algorithms II \\ Extra Credit 2}
\author{Dustin Ingram}
\begin{document}
\maketitle
\begin{enumerate}
    \item \textbf{33-4 Solution:}
    \begin{enumerate}
        \item The \textsc{Above-Or-Below} procedure:
        \begin{algorithm}[H]
        \begin{algorithmic}
        \floatname{algorithm}{Above-Or-Below($A, B$)}
        \caption{}
            \STATE {$x_1, y_1, z_1, x_2, y_2, z_2 = a$}
            \STATE {$x_3, y_3, z_3, x_4, y_4, z_4 = b$}
            \STATE {$det = (x_1 - x_2)(y_3 - y_4) - (y_1 - y_2)(x_3 - x_4)$}
            \IF {$det == 0$}
                \IF {$max(az_1,az_2) < max(bz_1,bz_2)$ \AND $min(az_1,az_2) < min(bz_1,bz_2)$}
                    \RETURN {$below$}
                \ELSE
                    \RETURN {$above$}
                \ENDIF
            \ELSE
                \STATE {$x = ((x_1y_2-y_1x_2)(x_3-x_4)-(x_1-x_2)(x_3y_4-y_3x_4))/det$}
                \STATE {$y = ((x_1y_2-y_1x_2)(y_3-y_4)-(y_1-y_2)(x_3y_4-y_3x_4))/det$}
                \IF {$x > max(x_1, x_2, x_3, x_4)$ \OR $x < min(x_1, x_2, x_3, x_4)$}
                    \RETURN {$unrelated$}
                \ELSIF {$y > max(y_1, y_2, y_3, y_4) or y < min(y_1, y_2, y_3, y_4)$}
                    \RETURN {$unrelated$}
                \ELSE
                    \STATE {$z_a = (((z_2-z_1)((x-x_1)/(x_2-x_1)+(y-y_1)/(y_2-y_1)))/2)+z_1$}
                    \STATE {$z_b = (((z_4-z_3)((x-x_3)/(x_4-x_3)+(y-y_3)/(y_4-y_3)))/2)+z_3$}
                    \IF {$z_a > z_b$}
                        \RETURN {$above$}
                    \ELSE
                        \RETURN {$below$}
                    \ENDIF
                \ENDIF
           \ENDIF
        \end{algorithmic}
        \end{algorithm}
        Here, we assume that sticks cannot physically intersect (e.g., the
        maximum and minimum z-values for the endpoints of two parallel sticks
        must both correspond). We check if the sticks are parallel (the
        determinant will be zero), then if they are unrelated, and finally we
        compute the z-values of their cross-point and determine which is on top.

        \item We simply reduce the 3D lines to their 2D ($x$ and $y$)
        representations. We use the \textsc{Any-Segments-Intersect} algorithm to
        determine at what point, if any, any segment intersects with any other
        segment. When we find an intersection, we use the
        \textsc{Above-Or-Below} procedure to determine ordering, and use this to
        build the legal order in which we can pick the sticks up in. This can
        simply be a sorted list similar to what \textsc{Any-Segments-Intersect}
        uses.
    \end{enumerate}
    \item \textbf{35-1 Solution:}
        \begin{enumerate}
            \item Here, we can reduce the bin-packing problem to the known
            NP-hard subset-sum decision problem. Given the set $S$ of $n$ items,
            we repeatedly try to find a maximally-sized subset in $S$ which sums
            to size $1$, such that the number of subsets is a minimum.
            \item Since every object must be packed into a bin, and $S$
            represents the total size of all objects, we must have $\lceil
            S\rceil$ total bins.
            \item The heuristic will not leave more than one bin less than half
            full. This is simply because if there were a second bin that was
            less than half full, its contents would have been put into the first
            bin before allocating a second bin.
            \item This will only be the case if every object is slightly greater
            in size than half of the size of the bin. Thus, every object will
            need to be packed into it's own bin, producing $n$ bins, where the
            total size $S$ is roughly $0.5n$, but the total number of bins is
            $n$, which will have a total size which will never be greater than
            $\lceil 2S\rceil$.
            \item It is clear from (b) that the optimal number of bins is
            precisely $\lceil S\rceil$ bins, and that from (d) that the
            worst-case number of bins is $\lceil 2S\rceil$ bins, thus the
            first-fit heuristic has an approximation ration of $2$.
            \item The algorithm is as follows. In a worst-case scenario, it
            tries to place $n$ objects into $n$ different bins, for a runtime of
            $O(n^2)$:
                \begin{algorithm}[H]
                \begin{algorithmic}
                \floatname{algorithm}{Above-Or-Below($A, B$)}
                \caption{}
                    \FOR{$o \in objects$}
                        \STATE{$found =$ \FALSE}
                        \FOR{$b \in bins$}
                            \IF{$b.emptySpace \geq o.size$}
                                \STATE{$b.add(o)$}
                                \STATE{$b.emptySpace -= o.size$}
                                \STATE{$found =$ \TRUE}
                            \ENDIF
                        \ENDFOR
                        \IF{\NOT $found$}
                            \STATE{$b =$ \textsc{New-Empty-Bin()}}
                            \STATE{$b.add(o)$}
                            \STATE{$bins.add(b)$}
                        \ENDIF
                    \ENDFOR
                \end{algorithmic}
                \end{algorithm}
        \end{enumerate}
    \end{enumerate}
\end{document}
