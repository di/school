\documentclass{beamer}
\usepackage{graphicx}
\usepackage{graphics}
\usepackage{beamerthemeshadow}
\begin{document}
  \title{How to Simply Encrypt and Decrypt a Message} 
  \author{Dustin Ingram, Hahna Kane, Jacob Lukas}
  \date{\today} 

  \frame{\titlepage} 

  \section{Introduction}
  	\frame{
      \frametitle{Intended Audience} 
      \begin{itemize}
				\item Someone with little to no background in cryptography
				\item Someone interested in solving puzzles 
			\end{itemize}
    }
  	
  	\frame{
      \frametitle{A Need for Cryptography} 
      \begin{itemize}
				\item Concealing information is prevalent throughout history
				\item Historical and modern applications in government, military, and personal domains
				\item Sensitive, personal, electronic data is exchanged daily: ATMs, Internet commerce, computer passwords, etc.
			\end{itemize}
    }
    \frame{
      \frametitle{Defining Cryptography} 
      Cryptography is:
      \begin{itemize}
				\item The science of designing systems to study and secure communication over non-secure channels
				\item The process of creating, analyzing and deciphering ciphers
			\end{itemize}
			A cipher is:
			\begin{itemize}
				\item A concealed message or collection of data
				\item Also known as a ``code'' or ``cryptogram''
			\end{itemize}
    }
    \frame{
      \frametitle{Conventions of Cryptography} 
			Before messages can be encrypted and decrypted, the following conventions need to be understood:
			\begin{enumerate}
				\item \textit{plaintext} will be presented as italicized, lowercase letters and CIPHERTEXT will be presented as all capital letters.
				\item The letters of the alphabet are assigned numbers as follows
				\begin{center}
					\scalebox{0.7}{%
					\begin{tabular}{ | c | c | c | c | c | c | c | c | c | c | c | c | c | }
						\hline
						a & b & c & d & e & f & g & h & i & j & k & l & m\\ \hline
						0 & 1 & 2 & 3 & 4 & 5 & 6 & 7 & 8 & 9 & 10 & 11 & 12\\ \hline
						\hline
						n & o & p & q & r & s & t & u & v & w & x & y & z\\ \hline
						13 & 14 & 15 & 16 & 17 & 18 & 19 & 20 & 21 & 22 & 23 & 24 & 25\\ \hline
					\end{tabular}
					}
				\end{center}				
				\item Spaces and punctuation are omitted. 
			\end{enumerate}
		}			
	\section{A Simple Cipher} 			
		\frame{
			\frametitle{A Simple Cipher}
			\begin{itemize}
				\item A cipher encrypts every string of characters by some algorithm
				\item	Modern ciphers require the use of a computer and are usually computationally impossible to break
				\item However, some ciphers are simple enough to execute by hand
			\end{itemize}
		}
		\frame{
			\frametitle{Caesar Cipher}
			The Caesar Cipher is a ``shifted'' cipher, first recorded to be used by Julius Caesar to protect messages of military significance.
			In general, shifted ciphers require a key \textit{$\kappa$} with $0\leq \textit{$\kappa$} \leq 25$. For encryption,
			\begin{equation}
			x \mapsto x+\kappa
			\end{equation}

			Similarly, decryption is
			\begin{equation}
			x \mapsto x-\kappa
			\end{equation}
			Traditionally, when using the Caesar cipher, $\kappa=3$.
		}
		\frame{
			\frametitle{The Caesar Cipher}
			Suppose Alice wanted to send the plaintext
			\begin{center}
			\textit{technical communication rocks}
			\end{center}
			to Bob, but she wanted to hide the message from Eve. Alice simply shifts each letter by three places\\
			\begin{center}
				\includegraphics [width=2in]{CCImg.png}
			\end{center}
			The resulting ciphertext becomes
			\begin{center}
				WHFKQLFDOFRPPXQLFDWLRQURFNV
			\end{center}
		}
	\section{Cryptanalysis}
		\frame{
			\frametitle{Cryptanalysis of the Caesar Cipher}
			\begin{itemize}
				\item Cryptanalysis is the process of attacking or breaking into a system to figure out what was communicated
				\item In the previous example, there are four possible scenarios in which Eve could decipher Alice's message without a key
				\begin{enumerate}
					\item Eve discovers Alice's encryption machine
					\item Eve discovers Bob's decryption machine
					\item Eve knows at least one letter of the plaintext and the letter it maps to in the ciphertext
					\item Eve only knows the ciphertext
				\end{enumerate}
			\end{itemize}
		}
		\frame{
			\frametitle{Cryptanalysis of the Caesar Cipher}
			\begin{itemize}
				\item Best strategy: perform an exhaustive search\\
				\begin{center}				
				\scalebox{0.7}{%
					\begin{tabular}{ | c || c | }
						\hline
						$\kappa$ & Plaintext\\ \hline
						\hline
						0 & WHFKQLFDOFRPPXQLFDWLRQURFNV \\ \hline
						1 & \textit{vgejpkecneqoowpkecvkqptqemu} \\ \hline
						2 & \textit{ufdiojdbmdpnnvojdbujpospdlt} \\ \hline
						3 & \textit{technicalcommunicationrocks} \\ \hline
						4 & \textit{sdbgmhbzkbnlltmhbzshnmqnbjr} \\ \hline
						5 & \textit{rcaflgayjamkkslgayrgmlpmaiq} \\ \hline
						\ldots & \textit{\ldots} \\ \hline
					\end{tabular}
				}
				\end{center}
				\item Next best alternative: examine the frequency count of the letters
			\end{itemize}
		}
	\section{Conclusion}
		\frame{
 			\frametitle{Conclusion}
			\begin{itemize}
				\item Learned the history and significance of ciphers
				\item Learned how to encrypt and decrypt a message using a Caesar cipher
				\item Learned how to attack a Caesar cipher message in four different ways
			\end{itemize}
			\begin{center}
			Questions?
			\end{center}
		}	
\end{document}
