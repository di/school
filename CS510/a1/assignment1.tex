\documentclass{article}

\usepackage{url}

\title{CS 510: Assignment 1\\Fall 2012}

\author{Dustin Ingram}

\date{\today}

\begin{document}

\maketitle

\section{Written}

\subsection{Question 1}

A finite space may produce an infinite search tree if duplicate states are not
rejected, thus forming loops in the search tree.

If the state space is a tree (implying no loops, thus it is finite) than the
search space is also guaranteed to be finite.

Any acyclic state space will produce a finite search tree.

\subsection{Question 2}

The ``iterative lengthening search'' algorithm is guaranteed to always produce
the solution with the lowest possible path cost simply because it only considers
nodes in order of their past cost -- it will never overlook a lower-cost path as
long as the length is large enough to include the length of the path of the
optimal solution. 

\subsection{Question 3}

The average branching factor of an 8-puzzle (without a check for cycles) is the
average number of possible moves for each of the nine valid positions:

$$ (1*4 + 4*3 + 4*2)/9 = 2.\overline{6} $$ 

\subsection{Question 4}

The easiest way to find this is by finding and counting the number of instances
which are $n$ moves from the solved state. This can be done by performing a
breadth-first search from the solved state, counting the number of
non-duplicated states at each level of the tree.

For the standard solved state (empty space in a corner), the root of the tree
(which requires 0 moves to reach the solved state) produces two states (which
require 1 move to reach the solved state), and so on.

Since the total number of states is:
$$ (9!/2) = 181440 $$
computing and counting each of these states in the tree is not computationally
trivial.  Luckily this is a known integer sequence\footnote{``Number of
configurations of the sliding block 8-puzzle that require a minimum of n moves
to be reached, starting with the empty square in one of the corners'', from
\emph{The Online Encyclopedia of Integer Sequences},
\url{http://oeis.org/A089473}}:
\\\\
\begin{tabular}{|l|c|c|c|c|c|c|c|c|c|c|c|c|c|c|}
\hline
\textbf{moves} & 0 & 1 & 2 & 3 & 4 & 5 & 6 & 7 & 8 & 9 & \dots & 30 & 31 \\
\hline
\textbf{instances} & 1 & 2 & 4 & 8 & 16 & 20 & 39 & 62 & 116 & 152 & \dots & 221 & 2 \\
\hline
\end{tabular}
\\\\

The resulting average is
$$\frac{\displaystyle\sum\limits_{i=0}^{31} instances_{i}*i}{9!/2} = 21.97$$

\section{Programming}

I implemented breadth-first, depth-first, and A* search algorithms. For A*, I
used the Manhattan distance as my heuristic.

\subsection{Efficiency Comparison}
I timed solving 1000 instances of the 8-puzzle problem with the BFS and A*
algorithms. I did not time DFS -- it was so slow that it was not worth my time
to wait for a reasonable number of trials to finish, and thus can be assumed to
be several magnitudes worse than BFS or A*.

The A* algorithm with Manhattan distance clearly out-performs BFS. I think I
could have improved the performance of A* even more as well; for example, I am
sorting the queue after every \texttt{insert}, rather then prior to every
\texttt{pop} (this was done to make the code simpler and more clear).

\subsubsection{BFS} 
\begin{verbatim}
Real time: 479.07156 sec.
Run time: 478.6979 sec.
Space: 23729296912 Bytes
GC: 16217, GC time: 85.64534 sec.
\end{verbatim}

\subsubsection{A*} 
\begin{verbatim}
Real time: 7.05169 sec.
Run time: 7.04844 sec.
Space: 124866648 Bytes
GC: 97, GC time: 0.264021 sec.
\end{verbatim}

\section{README} 

Running my implementation is identical to the example commands. First, load the
provided implementation:

\begin{verbatim}

(load "8puzzle.lisp")

\end{verbatim}

Then load my implementation (assumed to be in the same directory):

\begin{verbatim}

(load "a1.lisp")

\end{verbatim}

Then solve a random puzzle, either via BFS;

\begin{verbatim}

(let ((puzzle (random-puzzle))) 
  (print-puzzle puzzle)
  (solve-8puzzle 'BFS puzzle))

\end{verbatim}

via DFS;

\begin{verbatim}

(let ((puzzle (random-puzzle)))
  (print-puzzle puzzle)
  (solve-8puzzle 'DFS puzzle))

\end{verbatim}

or using A* and the Manhattan distance heuristic:

\begin{verbatim}

(let ((puzzle (random-puzzle)))
  (print-puzzle puzzle) 
  (solve-8puzzle 'A* puzzle))

\end{verbatim}

To time the search algorithms, use the \texttt{timeit} function and provide the
number of instances to solve, e.g.:

\begin{verbatim}

(timeit 1000)

\end{verbatim}

\end{document}
