\documentclass{article}

\usepackage{url}

\title{CS 510: Midterm\\Fall 2012}

\author{Dustin Ingram}

\date{\today}

\begin{document}

\maketitle

\begin{enumerate}

\item{}

\begin{enumerate}

\item{}

\item{}

\item{}

\item{}

\end{enumerate}

\item{}

\begin{enumerate}

\item{}

\item{\textbf{True}: A BFS search is guaranteed to arrive at a solution if the branching factor is finite. Although the depth may be infinite, the search will not fall into a cycle like a DFS search would because it does not evaluate states consecutively.}

\item{\textbf{False}: epending on the goal, A* might be useful for bi-directional search: e.g., if the goal is to find \emph{any} path between two  A* search is not generally useful for bi-directional search. 

\item{}

\end{enumerate}

\item{}

\begin{enumerate}

\item{}

\item{}

\item{}

\end{enumerate}

\item{}

\begin{enumerate}

\item{}

\item{}

\end{enumerate}

\item

\begin{enumerate}

\item{}

\item{}

\end{enumerate}

\item

\begin{enumerate}

\item{Pumpkin $\pi$}

\item{It is a quine, i.e., it outputs itself verbatim. It does this by creating
an anonymous expression which accepts a single parameter, which returns a
two-element list containing the result of the argument, and the argument fully
quoted. It then calls said anonymous function, with the fully quoted anonymous
expression as the argument.}

\item{Because \texttt{oct(31) == dec(25)}}

\end{enumerate}

\end{enumerate}

\end{document}
