\documentclass{article}

\title{CS 510: Assignment 1\\Fall 2012}

\author{Dustin Ingram}

\date{\today}

\begin{document}

\maketitle

\section{Written}

\subsection{Question 1}

Does a finite state space always lead to a finite search tree?  How about a
finite state space that is a tree? Can you precisely state what types of state
spaces always lead to finite search trees?

\subsection{Question 2}

Consider ``iterative lengthening search,'' an iterative analog to uniform cost
search.  The idea is to use increasing limits on path cost.  If a node is
generated whose path cost exceeds the current limit, it is immediately
discarded.  For each new iteration, the limit is set to the lowest path cost of
any node discarded in the previous iteration.  Is this algorithm optimal for
general path costs?  In other words, is it guaranteed to always produce the
solution with the lowest possible path cost?  Provide a formal proof either of
optimality or the contrary.

\subsection{Question 3}

The average branching factor of an 8-puzzle (without a check for cycles) is the
average number of possible moves for each of the nine valid positions:

$$ (1*4 + 4*3 + 4*2)/9 = 2.\overline{6} $$ 

\subsection{Question 4}

\emph{Extra Credit: What is the average number of ``slides'' required to solve
an 8-puzzle?}

\section{Programming}

\subsection{Efficiency Comparison}

\section{README} 

\begin{verbatim}

(let ((puzzle (random-puzzle))) (print-puzzle puzzle) (solve-8puzzle 'BFS
puzzle))

\end{verbatim}

\begin{verbatim}

(let ((puzzle (random-puzzle))) (print-puzzle puzzle) (solve-8puzzle 'DFS
puzzle))

\end{verbatim}

\begin{verbatim}

(let ((puzzle (random-puzzle))) (print-puzzle puzzle) (solve-8puzzle 'A*
puzzle))

\end{verbatim}

\begin{verbatim}

(timeit 100)

\end{verbatim}

\end{document}
