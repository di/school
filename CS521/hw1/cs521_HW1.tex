\documentclass{article}
\usepackage{booktabs}
\usepackage{amsmath}
\providecommand{\e}[1]{\ensuremath{\times 10^{#1}}}
\title{CS 521: Data Structures and Algorithms I \\ Homework 1}
\author{Dustin Ingram}
\begin{document}
\maketitle
\begin{enumerate}
\item \textbf{Problem 3-3, pg 61}
    \begin{enumerate} \item  A simple way to get a nearly-correct ordering of functions by asymptotic groth rate is to choose a sufficiently large $n$ and evaluate each function. Below, each ranked function is also shown with its value in the example case of $n=1000$. \\ \\
We can see that for comparison with functions using iterated logartithms, $n=1000$ is still not sufficiently large, but we know that such functions grow \emph{so} slowly that they will not out-grow any non-iterative functions. However, to properly order them amongst themselves, we must now use a sufficiently \emph{small} $n$, which again can be $n=1000$, to reveal their comparative growth rates. \\ \\
 \begin{minipage}{2in}
 \centering
 \begin{tabular}{l l} 
    \toprule
    $g_{i}(n)$ & $g_{i}(1000)$ \\
    \toprule
    $2^{2^{n+1}}$ & (large) \\ \hline
    $2^{2^{n}}$ & (less large) \\ \hline
    $(n+1)!$ & 4.02\e{2570} \\ \hline
    $n!$ & 4.2\e{2567} \\ \hline
    $e^{n}$ & 1.9\e{434} \\ \hline
    $n*2^{n}$ & 1.0\e{304} \\ \hline
    $2^{n}$ & 1.0\e{301} \\ \hline
    $(3/2)^{n}$ & 1.2\e{176} \\ \hline
    $n^{\lg(\lg(n)}$ & 8.9\e{9} \\
    $(\lg(n))^{\lg(n)}$ & 8.9\e{9} \\ \hline
    $(\lg(n))!$ & 3.34\e{6} \\ \hline
    $n^3$ & 1000000000 \\ \hline
    $n^2$ & 1000000 \\
    $4^{\lg(n)}$ & 1000000 \\ \hline
    $n\lg(n)$ & 9965 \\
\end{tabular}
\end{minipage}
\begin{minipage}{2in}
 \centering
 \begin{tabular}{l l} 
    \toprule
    $g_{i}(n)$ & $g_{i}(1000)$ \\
    \toprule
    $\lg(n!)$ & 8529 \\ \hline
    $n$ & 1000 \\
    $2^{\lg(n)}$ & 1000 \\ \hline
    $\lg^{2}(n)$ & 99.3 \\ \hline
    $(\sqrt{2})^{\lg(n)}$ & 31.6 \\ \hline
    $2^{\sqrt{2\lg(n)}}$ & 22.07 \\ \hline
    $\ln(n)$ & 6.9 \\ \hline
    $\sqrt{\lg(n)}$ & 3.15 \\ \hline
    $n^{1/(\lg(n))}$ & 2 \\ \hline
    $\ln(\ln(n))$ & 1.93 \\ \hline
    $2^{\lg^{*}(n)}$ & 16 \\ \hline
    $\lg^{*}(n)$ & 4 \\ \hline
    $\lg^{*}(\lg(n))$ & 3 \\ \hline
    $\lg(\lg^{*}(n))$ & 2 \\ \hline
    $1$ & 1 \\
\end{tabular}
\end{minipage}

    \item This is satisfied by making a faster-growing function trigonometric, e.g.\ $f(n) = cos(n)*2^{2^{n+1}}$.
    \end{enumerate}
\item \textbf{Problem 3-5, pg 62}
\begin{enumerate}
\item If $f(n) = O(g(n))$, then there exists some $n$ after which $0 \leq f(n) \leq c*g(n)$. If the function behaves in a trigonometric fashion, there may be more than one value for $n$ after which $f(n) \geq c*g(n) \geq 0$, which would make $f(n) = \overset{\infty}{\Omega}(g(n))$, while $f(n) = O(g(n))$ would remain true.
\\
However, if $f(n) = \Omega(g(n))$, then for all $n$ after some value, $f(n) \geq c(g(n) \geq 0$, and therefore $f(n) = O(g(n))$ cannot also be true.
\item An advantage to using $\overset{\infty}{\Omega}$ instead of $\Omega$ to characterize the run time of programs is that it would reveal to us the instances in which we could not expect there to be a single $n$ after which $f(n) \geq c*g(n)$. A disadvantage is that it is generally more useful to know that $f(n) \geq c*g(n)$ for all values from $n$ to infinity. 
\item Theorem 3.1 states: \\
\emph{For any two functions $f(n)$ and $g(n)$, we have $f(n)=\Theta(g(n))$ if and only if $f(n) = O(g(n))$ and $f(n)=\Omega(g(n))$.}\\
If we substitute $O'$ for $O$, but still use $\Omega$, the $\Omega$ remains the same, but the direction of $f(n) = O(g(n))$ becomes $f(n) \leq O(g(n))$.
\item Definitions are as follows: \\
$\overset{\sim}{\Omega}(g(n)) = \{f(n):$ there exists positive constants $c$, $k$ and $n_{0}$ such that $cg(n)lg^{k}(n) \geq f(n) \geq 0$ for all $n \geq n_{0}$ \\
$\overset{\sim}{\Theta}(g(n)) = \{f(n):$ if an only if there exists positive constants $c$, and $n_{0}$ such that $f(n) = \overset{\sim}{O}(cg(n))$ and $f(n) = \overset{\sim}{\Theta}(cg(n))$ for all $n \geq n_{0}$ \\ 
\end{enumerate}
\item \textbf{Problem 3-6, pg 63} The following bounds were determined by generating a subset of values for each function (generally where $n=\{1,...,32\}$ would suffice) and then attempting to fit a growth function to the results. 
\begin{center}
\begin{tabular}{c c c | c |}
 & $f(n)$ & $c$ & $f_{c}^{*}(n)$ \\ \hline
\textbf{a.} & $n-1$ & 0 & $\Theta(n)$ \\ \hline
\textbf{b.} & $\lg(n)$ & 1 & $\Theta(\lg(n))$ \\ \hline
\textbf{c.} & $n/2$ & 1 & $\Theta(\sqrt{2}^{\lg(n)})$ \\ \hline
\textbf{d.} & $n/2$ & 2 & $\Theta(\ln(n))$ \\ \hline
\textbf{e.} & $\sqrt{n}$ & 2 & $\Theta(\ln(n))$ \\ \hline
\textbf{f.} & $\sqrt{n}$ & 1 & $\Theta(10^{n})$ \\ \hline
\textbf{g.} & $\sqrt{n^{1/3}}$ & 2 & $\Theta(\lg(\lg^{*}(n))$ \\ \hline
\textbf{h.} & $n/\lg(n)$ & 2 & $\Theta(\ln(n))$ \\ \hline
\end{tabular}
\end{center}

\item \textbf{Problem 4-3, pg 108}
\begin{enumerate}
\item $T(n) = 4T(n/3) + n\lg{n}$ \\
Here, $a=4$, $b=3$, $f(n)=n\lg{n}$, and thus we have that $n^{\log_{b}{a}} = n^{\log_{3}{4}} = O(n^{1.26})$. Since $f(n) = \Omega(n^{\log_{3}{4}+\epsilon)}$ where $\epsilon \approx 0.74$, the ratio $f(n)/n^{\log_{b}{a}} = (n\lg{n})/(n^{2}) = \lg{n}/n$ is asymptotically less than $n^{\epsilon}$.
\item $T(n) = 3T(n/3) + n/\lg{n}$ \\
Here, $a=3$, $b=3$, $f(n)=n/\lg{n}$, and thus we have that $n^{log_{b}{a}} = n$. Since the ratio $f(n)/n^{\log_{b}{a}} = (n/\lg{n})/n$ is neither asympotically less or more than $n^{\epsilon}$, the Master Theorem cannot be applied. Via substitution:
\begin{align*}
T(n) &= 3T(n/3) + n/\lg{n} \\
& = 3(3T(n/9) + (n/3)/(\lg{n/3})) + n/\lg{n} \\
& = 9T(n/9) + n/\lg{n/3} + n/\lg{n} \\
& = 3^{i}T(n/3^{i}) + \sum_{j=1}^{i-1}{n/\lg(n/3^{j})}
\end{align*}
\item $T(n) = 4T(n/2) + n^{2}\sqrt{n}$ \\
Here, $a=4$, $b=2$, $f(n)=n^{2}\sqrt{n}$, and thus we have that $n^{\log_{b}{a}} = n^{\log_{2}{4}} = O(n^2)$ via the Master Theorem.
\item $T(n) = 3T(n/3 - 2) + n/2$ \\ 
Here, we must solve via substitution:
\begin{align*}
T(n) &= 3T(n/3 - 2) + n/2 \\
&= 3T(n/9 - 2/3 -2) + n/6 - 2 \\
&= 3T(n/9 - 8/3) + n/6 + 1 \\
&= 3T(n/27 - 8/9 - 2) + n/18 + n/6 - 2 - 2 \\
&= 3T(n/27 - 26/9) + n/18 + n/6 - 4 \\
\end{align*}
\item $T(n) = 2T(n/2) + n/\lg{n}$ \\
Here, $a=2$, $b=2$, $f(n)=n/\lg{n}$, and thus we have that $n^{log_{b}{a}} = n$. Since the ratio $f(n)/n^{\log_{b}{a}} = (n/\lg{n})/n$ is neither asympotically less or more than $n^{\epsilon}$, the Master Theorem cannot be applied. Via substitution:
\begin{align*}
T(n) &= 2T(n/2) + n/\lg{n} \\
& = 2(2T(n/4) + (n/2)/(\lg{n/2})) + n/\lg{n} \\
& = 4T(n/4) + n/\lg{n/2} + n/\lg{n} \\
& = 2^{i}T(n/2^{i}) + \sum_{j=1}^{i-1}{n/\lg(n/2^{j})}
\end{align*}
\item $T(n) = T(n/2) + T(n/4) + T(n/8) + n$ \\ 
\item $T(n) = T(n-1) + 1/n$ \\
\item $T(n) = T(n-1) + \lg{n}$ \\
\item $T(n) = T(n-2) + 1/\lg{n}$ \\
\item $T(n) = \sqrt{n}T(\sqrt{n}) + n$ \\
Here, $a$ is not a constant, so the Master Theorem cannot be used. 
\end{enumerate}
\item \textbf{Problem 4-5, pg 109}
For this problem, reference the following cases:
\begin{itemize}
    \item \emph{Case 1.} Both chips declared `good' -- both are either `good' or `bad'
    \item \emph{Case 2.} Both chips declared `bad' -- at least one is `bad'
    \item \emph{Case 3.} One chip declared `good', one `bad' -- at least one is `bad'
\end{itemize}

\begin{enumerate}
\item If more than $n/2$ chips are `bad', pairwise testing will not reveal all the `good' chips. Since the number of `bad' chips exceed the number of good chips, at every iteration of a pairwise testing methodology (like the one described in Part~\emph{b}), it can never be guaranteed that the number of `good' chips outnumber `bad' chips, and therefore a single `good' chip cannot be found.
\item If the goal is to find only one `good' chip amongst $n$ chips, of which the number of `good' chips $g$ is greater than the number of `bad' chips $b$, the problem set (originally size $n$) can be quickly reduced by half using $\lfloor n/2 \rfloor$ pairwise tests by simply rejecting every pair of chips which result in either \emph{Case 2} or \emph{Case 3}. 
\\
This step alone, however, does not guarantee a reduction in the problem set to size $n/2$ -- for example, it is possible that for an even-sized set (with even-sized amounts of both `good' and `bad' chips) would not result in the rejection of any chips at all. Therefore, it would be necessary to maintain groups of `A' chips and `B' chips; in the event that the rejection set is less than $n$ chips, both of these groups would contain indentical amounts of `good' and `bad' chips, and one could be rejected, resulting in a set of chips with size $< n/2$ where $g > b$ is still guaranteed to hold true.
\item \emph{This is assuming that ``Show that the good chips can be identified with $\Theta(n)$ pairwise tests'' in fact means `Show that each good chip can be individually found in $\Theta(n)$ pairwise tests' and not `Show that every good chip in a set of size $n$ can be found in $\Theta(n)$ pairwise tests'\dots} 
\\
Using the method described in Part~\emph{b}, the recurrence is simple: each iteration of the method performs $\lfloor n/2 \rfloor$ comparisons, and leaves (as a worst-case) $n/2$ chips for the next iteration. With the assumption that $T(n) = \Theta(n)$, we can induce:
\begin{align*}
T(n) &\leq T(n/2) + \lfloor n/2 \rfloor \\
& \leq (c(n/2))+n/2 \\
& \leq c(n/2 + n/2) \\
& \leq cn
\end{align*}
\end{enumerate}
\end{enumerate}
\end{document}
