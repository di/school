\documentclass{article}
\usepackage{booktabs}
\usepackage{amsmath}
\usepackage[noend]{algorithmic}
\usepackage[nothing]{algorithm}
\usepackage{tikz}
\usepackage{latexsym}
\providecommand{\e}[1]{\ensuremath{\times 10^{#1}}}
\renewcommand{\thealgorithm}{}
\title{CS 521: Data Structures and Algorithms I \\ Homework 4}
\author{Dustin Ingram}
\begin{document}
\maketitle
\begin{enumerate}
\item \textbf{Solution:}
Given a weighted, directed graph $G=(V,E)$ with no negative-weight cycles, $m+1$ would also represent the number of iterations of the Bellman-Ford algorithm, as this would mean that some shortest path from $s$ to any vertex $v$ has already been found, and any subsequent iterations would not modify the current paths. The \textsc{Relax} function can be modified to report changes as follows:
\begin{algorithm}[H]
\floatname{algorithm}{\normalfont\textsc{Relax}$(u, v, w)$}
\caption{}
\begin{algorithmic}
\IF{$v.d > u.d + w(u,v)$}
    \STATE $v.d = u.d + w(u,v)$
    \STATE $v.\pi = u$
    \RETURN \textsc{true}
\ENDIF
\RETURN \textsc{false}
\end{algorithmic}
\end{algorithm}
Therefore, modifying the Bellman-Ford algorithm as follows to terminate after an iteration where no edges were relaxed would terminate the algorithm after $m$ total iterations.
\begin{algorithm}[H]
\floatname{algorithm}{\normalfont\textsc{Bellman-Ford}$(G, w, s)$}
\caption{}
\begin{algorithmic}
\STATE \textsc{Initialize-Single-Source}$(G,s)$
\FOR{each edge $(u,v) \in G.E$}
    \IF{\textsc{Relax}$(u,v,w) == $ \textsc{false}}
        \STATE \textsc{Break} 
    \ENDIF
\ENDFOR
\end{algorithmic}
\end{algorithm}

\item \textbf{Solution:} 
For a graph which has a negative-weight cycle, we can run the Bellman-Ford algorithm once to produce the ``distance'' values and parent values for each node in the graph. \\
If we were to run the Bellman-Ford algorithm again on the graph, with the $v.d$ and $v.\pi$ values set from the previous iteration (instead of being re-initialized), these values would not change. However, if there is some negative-weight cycle in the graph, some vertex's $v.d$ and $v.\pi$ will change.\\
Once this is detected (by a successful call to \textsc{Relax}, which makes a change to these values), we can travel back through the negative-weight cycle from $v$ to $v.\pi$, etc., until we return to $v$, at which point we have listed all the vertices in the cycle.\\
The complexity of this modified algorithm is bounded by running the Bellman-Ford algorithm twice, then traversing a cycle of at most $V$ vertices, for an overall runtime of $O(2(VE)+V)=O(VE)$.

\item \textbf{Solution:} 
This solution iterates over each edge $(u,v) \in G(V,E)$. Each vertex holds a variable $v.p$, which represents the number of paths to that vertex at any given moment. As each new edge is added, the number of paths to the `destination' edge increases by the number of paths to the `origin' edge, plus one (for the current edge). After the algorithm finishes, the total number of paths is the sum of all paths to each vertex. The algorithm instead maintains $total$ as it iterates over the edges; therefore the complexity is $O(E)$.
    \begin{algorithm}[H]
    \floatname{algorithm}{\normalfont\textsc{Total-DAG-Paths}$(G)$}
    \caption{}
    \begin{algorithmic}
    \STATE $total = 0$
    \FOR{each edge $(u,v) \in G(V,E)$}
        \STATE $v.p$ += $u.p + 1$
        \STATE $total$ += $u.p + 1$
    \ENDFOR
    \RETURN $total$
    \end{algorithmic}
    \end{algorithm}

\item \textbf{Solution:}
Here, we can use Dijkstra's algorithm (modified to use \textsc{Extract-Max} instead) and a modified version of \textsc{Relax} as follows:

    \begin{algorithm}[H]
    \floatname{algorithm}{\normalfont\textsc{Relax}$(u, v, r)$}
    \caption{}
    \begin{algorithmic}
    \IF{$v.r < u.d \times r(u,v)$}
        \STATE $v.r = u.r \times r(u,v)$
        \STATE $v.\pi = u$
    \ENDIF
    \end{algorithmic}
    \end{algorithm}

Instead of minimizing the weight of a path (sum of weight of edges), this modified algorithm will maximize the reliability of a path (product of reliabilities of edges).

\item \textbf{Solution:} 
The solution is trivial: Instead of creating a new matrix for every iteration of the \texttt{while} loop, we keep two copies, $L^{(a)}$ and $L^{(b)}$, and use the latter to store the result of a call to \textsc{Extend-Shortest-Paths} using the former, then update it with the result.
\begin{algorithm}[H]
\floatname{algorithm}{\normalfont\textsc{Smaller-All-Pairs-Shortest-Paths}$(W)$}
\caption{}
\begin{algorithmic}
\STATE $n = W.rows$
\STATE $L^{(a)} = W$
\STATE $L^{(b)} = W$
\STATE $m=1$
\WHILE{$m<n-1$}
    \STATE $L^{(b)}=$ \textsc{Extend-Shortest-Paths}$(L^{(a)},L^{(a)})$
    \STATE $L^{(a)} = L^{(b)}$
    \STATE $m=2m$
\ENDWHILE
\RETURN $L^{(a)}$
\end{algorithmic}
\end{algorithm}
This uses only two matrices for a space complexity of $\Theta(n^2)$.
\item \textbf{Solution:} 
This modification of equation (25.7) in fact reveals a third case for the assignment of $\pi_{ij}^{(k)}$, which can be interpreted as follows:
\begin{equation*}
\pi_{ij}^{(k)}=
    \begin{cases}
        \text{$\pi_{ij}^{(k-1)}$} & \text{if $d_{ij}^{(k-1)} < d_{ik}^{(k-1)} + d_{kj}^{(k-1)}$}\\
        \text{$\pi_{kj}^{(k-1)}$} & \text{if $d_{ij}^{(k-1)} > d_{ik}^{(k-1)} + d_{kj}^{(k-1)}$}\\
        \text{$\pi_{ij}^{(k-1)}$ \textbf{OR} $\pi_{kj}^{(k-1)}$} & \text{if $d_{ij}^{(k-1)} = d_{ik}^{(k-1)} + d_{kj}^{(k-1)}$}\\
    \end{cases}
\end{equation*}
The last case simply represents the case where the existing path $i \leadsto j$ is the exact same distance as the new path $i \leadsto k \leadsto j$. In the original version of the equation, the algorithm kept the shortest path which was first found. The modified equation updates the path to use the edge $k \leadsto j$. The alternative definition of the predecessor matrix $\Pi$ is correct, but it will produce a different predecessor matrix if the third case is encountered.

\item \textbf{Solution:}
We can generate the strongly connected component graph $G^{SCC}$ from the general directed graph $G$ in $O(V+E)$ using the \textsc{Strongly-Connected-Components} algorithm. Since $G^{SCC}$ is a DAG by definition, we can then use the $f(|V|,|E|)$-time algorithm to compute $G^{+}$, the transitive closure of $G^{SCC}$ (which holds, because $G^{SCC}$ can have at most $V$ vertices and $E$ edges). Then, assuming that we have maintained knowledge of which original vertices in $G$ belong to which strongly connected component in $G^{SCC}$ (now represented by vertices in $G^{+}$), for each vertex $v$, we draw an edge between $v$ and each vertex in its strongly connected component, as well as between $v$ and each vertex in an SCC neighboring its own SCC. Since there are at most $V$ vertices and $E^{*}$ possible edges in the resulting graph, this leaves us with a running time of $f(|V|,|E|)+O(V+E^{*})$. 

\item \textbf{Solution:}
This algorithm maintains an array $S'$ as it iterates through all possible subsequences, where $S'[k]$ represents the size of the longest monotonically increasing subsequence ending at $S[k]$.
\begin{algorithm}[H]
\floatname{algorithm}{\normalfont\textsc{Longest-Monotonically-Increasing-Subsequence}$(S)$}
\caption{}
\begin{algorithmic}
\STATE $max=0$
\STATE $S'[0\dots (n-1)] = 0$
\FOR{$k = 0$ \TO $(n-1)$}
    \STATE $max'=0$
    \FOR{$j = 0$ \TO $(k-1)$}
        \IF{$S[k] > S[j]$}
            \STATE $max' = $ \textsc{Max}$(max', S'[j])$
        \ENDIF
    \ENDFOR
    \STATE $S'[k] = max' + 1$
    \STATE $max = $ \textsc{Max}$(max, S'[k])$
\ENDFOR
\STATE $R[0\dots (max-1)] = 0$
\FOR{$i = (n-1)$ \TO $0$}
    \IF{$S'[i] = max$}
        \STATE $max = max-1$
        \STATE $R[max] = S[i]$
    \ENDIF
\ENDFOR
\RETURN $R$
\end{algorithmic}
\end{algorithm}
 Once $S'$ is computed, it is necessary to traverse backwards through $S'$ to fill $R$, which is the result array.
\end{enumerate}
\end{document}
