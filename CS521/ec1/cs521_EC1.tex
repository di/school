\documentclass{article}
\usepackage{booktabs}
\usepackage{amsmath}
\providecommand{\e}[1]{\ensuremath{\times 10^{#1}}}
\title{CS 521: Data Structures and Algorithms I \\ Extra Credit}
\author{Dustin Ingram}
\begin{document}
\maketitle
\begin{enumerate}
    \item \textbf{(4-6) Solution:} 
    \begin{enumerate}
        \item \textbf{Base Case:} A $2\times 2$ Monge Array:
        \begin{align*}
            A[i,j] + A[i+1,j+1] &\leq A[i,j+1]+A[i+1,j] \\
            A[1,1] + A[2,2] &\leq A[1,2]+A[2,1]
        \end{align*}
        \textbf{Inductive Step (Rows):} A $m\times 2$ array, $m\geq2$:\\
        If $m=2$, $i+1=m=2$, $i=1$, and the base case holds:
        \begin{align*}
            A[i,j] + A[i+1,j+1] &\leq A[i,j+1]+A[i+1,j] \\
            A[i,1] + A[m,2] &\leq A[i,2]+A[m,1] \\
            A[1,1] + A[2,2] &\leq A[1,2]+A[2,1]
        \end{align*}
        Then, adding a new row so that $i=m$ and $i+1=m+1$:
        \begin{align*}
            A[i,j] + A[i+1,j+1] &\leq A[i,j+1]+A[i+1,j] \\
            A[m,1] + A[m+1,2] &\leq A[m,2]+A[m+1,1]
        \end{align*}
        The equalities can be transformed:
        \begin{align*}
            A[i,1] + A[m,2] \leq A[i,2]+A[m,1] &\Rightarrow A[i,1] - A[i,2] \leq A[m,1] - A[m,2] \\
            A[m,1] + A[m+1,2] \leq A[m,2]+A[m+1,1] &\Rightarrow A[m,1] - A[m,2] \leq A[m+1, 1] - A[m+1,2]
        \end{align*}
        Combining these:
        \begin{align*}
            A[i,1] - A[i,2] \leq A[m,1] - A[m,2] \leq A[m+1, 1] - A[m+1,2]
        \end{align*}
        Therefore:
        \begin{align*}
            A[i,1] - A[i,2] &\leq A[m+1, 1] - A[m+1,2] \\
            A[i,1] + A[m+1,2] &\leq A[m+1, 1] + A[i,2]
        \end{align*}
        \textbf{Inductive Step (Columns):} A $2\times n$ array, $n\geq2$:\\
        If $n=2$, $j+1=n=2$, $j=1$, and the base case holds:
            $$A[1,j] + A[2,n] \leq A[2,j]+A[1,n]$$
        Then, adding a new column so that $j=n$ and $j+1=n+1$:
            $$A[1,n] + A[2,n+1] \leq A[2,n]+A[1,n+1]$$
        The equalities can be transformed and combined as in the first step, therefore:
            $$A[1,j] + A[2,n+1] \leq A[1, n+1] + A[2,j]$$
        \textbf{Inductive Assumption:} A $m\times n$ array, $m<2$, $n<2$:
            $$A[m,n] + A[m+1, n+1] \leq A[m, n+1] + A[m+1, n]$$

        \item Using the inductive assumption above, it reveals that the bolded sub-array is invalid: 
        \[
        \left(
        \begin{array}{cccc}
        37 & \textbf{23} & \textbf{22} & 32 \\
        21 & \textbf{6} & \textbf{7} & 10 \\
        53 & 34 & 30 & 31 \\
        32 & 13 & 9 & 6 \\
        43 & 21 & 15 & 8 \end{array}
        \right)\]
        Therefore, one element within it must be changed. By assigning the following variables to this subarray:
        \[
        \left(
        \begin{array}{cc}
        a & b \\
        c & d \end{array}
        \right)\]
        The goal will be to satisfy
        $$a + d \leq b + c$$
        The following inequalities can be generated based on each element's local Monge array dependencies:
        \begin{align*}
            37 + 6 \leq a + 21 &\Rightarrow 22 \leq a \\
            b + 10 \leq 7 + 32 &\Rightarrow b \leq 29 \\
            37 + c \leq 23 + 21 &\Rightarrow c \leq 7 \\
            21 + 34 \leq c + 53 &\Rightarrow 2 \leq c \\
            c + 30 \leq 7 + 34 &\Rightarrow c \leq 11 \\
            6 + 30 \leq d + 34 &\Rightarrow 2 \leq d \\
            d + 31 \leq 10 + 30 &\Rightarrow d \leq 9 \\
            22 + 10 \leq 32 + d &\Rightarrow 0 \leq d 
        \end{align*}
        Which can be reduced to the following external dependencies:
        \begin{align*}
            22 \leq &a \\
            &b \leq 29 \\
            2 \leq &c \leq 7 \\
            2 \leq &d \leq 9
        \end{align*}
        The inequalities which show how the variable must be changed within the bolded sub-array are as follows:
        \begin{align*}
            a + 7 \leq 22 + 6 &\Rightarrow a \leq 21 \\
            23 + 7 \leq b + 6 &\Rightarrow 24 \leq b \\
            23 + 7 \leq 22 + c &\Rightarrow 8 \leq c \\
            23 + d \leq 22 + 6 &\Rightarrow d \leq 5 \\
        \end{align*}
        It is clear that $a$ and $c$ cannot be modified to hold all dependencies. The resulting inequalities show the possible ranges for $b$ and $d$:
        \begin{align*}
            24 &\leq b \leq 29 \\
            2 &\leq d \leq 5
        \end{align*}
        Therefore making $d=5$ results in the following Monge array:
        \[
        \left(
        \begin{array}{cccc}
        37 & 23 & 22 & 32 \\
        21 & 6 & \textbf{5} & 10 \\
        53 & 34 & 30 & 31 \\
        32 & 13 & 9 & 6 \\
        43 & 21 & 15 & 8 \end{array}
        \right)\]

        \item Take an array $A$ such that $f(i) > f(i+1)$; for example, where the minimum of the $i$-th row is found at A[i,j] and the minimum of the $i+1$-th row is found at A[i+1, j-1]. This means that the sum
        $$A[i,j] + A[i+1, j-1]$$
        is the lowest possible pair of values for the $i$-th and $i+1$-th rows. However, for the sub-array containing these values to be Monge, the following must be true:
        $$A[i, j-1] + A[i+1, j] \leq A[i,j] + A[i+1, j-1]$$
        Which requires that there exist another pair of values less than the pair of minimums. Therefore, $f(1) \leq f(2) \leq \dots \leq f(m)$ for any $m\times n$ Monge array.
        \item For a square $n\times n$ Monge array $A$, the left-most minimum for each odd-numbered row (given the leftmost minimum of the preceding even-numbered row) can be be in only one of two locations: if the leftmost minimum of the preceding even-numbered row is at $A[i,j]$, the leftmost minimum of the following odd-numbered row must be at $A[i+1,j]$ or $A[i+1,j+1]$ to satisfy the condition that $f(i) \leq f(i+1)$. Since there are $n/2$ odd-numbered rows to search, this results in a complexity of $O(2*(n/2)) = O(n)$. However, for a non-square $n\times m$ Monge array, there can be at most $m$ additional elements between $f(1)$ and $f(m)$, therefore the complexity is at most $O(n+m)$. 
    \end{enumerate}

\item \textbf{(7-5) Solution:} 
    \begin{enumerate}
        \item When partitioning $n$ elements at position $i$, the number of elements before and after the partition are $i-1$ and $n-i$, respectively. This results in $(i-1)(n-1)$ possible three-element subsets which will include pivot element at $i$. The total number of possible three-element subsets from a set of $n$ elements is simply ${n \choose 3}$. Therefore, the probability that any given three-element subset will contain the pivot element is the ratio of subsets containing the pivot element to the number of possible subsets, or:
        $$ \frac{(i-1)(n-i)}{{n \choose 3}} = \frac{(i-1)(n-i)}{\frac{n!}{3!(n-3)!}} = \frac{(i-1)(n-i)}{\frac{n(n-1)(n-2)}{6}}$$
        \item We can assume that $i$ is even, therefore let $i=\frac{n}{2}$. The resulting probability is: 
        \begin{align*}
        \frac{(\frac{n}{2}-1)(n-\frac{n}{2})}{\frac{n(n-1)(n-2)}{6}}
        &= \frac{6(\frac{n^{2}}{2}-\frac{n^{2}}{4}-n+\frac{n}{2})}{(n-1)(n^{2}-2n)} \\
        &= \frac{6(\frac{n^{2}}{4}-\frac{n}{2})}{(n-1)(n^{2}-2n)} \\
        &= \frac{\frac{6}{4}(n^{2}-2n)}{(n-1)(n^{2}-2n)}\\
        &= \frac{3}{2(n-1)}
        \end{align*}
        The ratio of this probability to the probability of the ordinary implementation, $\frac{1}{n}$, is simply:
        $$ \frac{\frac{3}{2(n-1)}}{\frac{1}{n}} = \frac{3n}{2n-2} $$
        As $n\rightarrow \infty$, this becomes $\frac{3}{2}$, or a $1.5\times$ improvement.
        \item The probability of getting a `good' split with the ordinary implementation is $\frac{1}{3}$... not sure how to approximate the sum by an integral.
        \item In the best-case scenario, the recursion tree of \textsc{Quicksort} will be perfectly divided, with height $\lg{n}$ and $n$ leaves, resulting in a $\Omega(n\lg{n})$ running time, so the median-of-three method cannot improve upon this scenario.
    \end{enumerate}

\end{enumerate}
\end{document}
