\documentclass{article}
\usepackage{booktabs}
\usepackage{amsmath}
\providecommand{\e}[1]{\ensuremath{\times 10^{#1}}}
\title{CS 521: Data Structures and Algorithms I \\ Homework 2}
\author{Dustin Ingram}
\begin{document}
\maketitle
\begin{enumerate}
\item \textbf{Solution:} In a full binary heap, every node (excluding the lowest level where $h=0$) has two children, thus, the number of nodes at any given height $h$ is twice that of it's preceding height $h+1$. Inductively, this means that each new level doubles the number of nodes in the tree, i.e., for a full tree of $n$ nodes, the last level will contain $\lceil n/2 \rceil$ of the nodes. Since $h$ (in this instance) is defined as the distance from the lowest level of the tree, this means that at any height $h$ there are $\lceil n/2^{h+1} \rceil$ nodes.

\item \textbf{Solution:} The original order of the array $A$ of length $n$ has no effect on the running time of \textsc{Heapsort}, since for every $n$ elements in the array, \textsc{Max-Heapify}, which makes at most $\lg{n}$ comparisons, must be called once, for a total running-time of $n\lg{n}$ in both cases.  

\item \textbf{Solution:} Since each of the $k$ lists are already sorted, two lists can be merged into one sorted list in linear time, regardless of size. If, at each iteration of the algorithm, we merge $k$ lists of size $n/k$ into $k/2$ lists of size $2n/k$ via pairwise merging, for a total of $k*(n/k) = n$ comparisons, we can produce a single sorted list in $lg(k)$ iterations, giving the algorithm a overall runtime of $O(n\lg{k})$.  

\item \textbf{Solution:} The complexity of correctly placing a single element in a list using \textsc{Insertion-Sort} depends directly on the number of adjacent elements the chosen element must be compared with. In a usual case of a initially random array of size $n$, this may be as many as $n$ elements, producing a worst-case performance of $O(n^{2})$. However, if we can assure that the maximum number of comparisons to adjacent elements will not exceed some relatively small constant $k$, as in the case of check-sorting, the performance can be improved to, at most, $O(nk)$. \\ \\
In the case of \textsc{Quicksort}, however, the single element's proximity to it's correct positioning does not improve the complexity of the algorithm, which still needs to perform $\lg{n}$ comparisons for every $n$ elements for a total complexity of $n\lg{n}$. As long as the constant $c < \lg{n}$, the \textsc{Insertion-Sort} algorithm will out-perform \textsc{Quicksort} for cases of almost-sorted input.

\item \textbf{Solution:} Continuing from the previous solution, we see that if we allow \textsc{Quicksort} to return a list full of nearly-sorted sub-arrays of at most size $k$, that running \textsc{Insertion-Sort} then guarantees a additional complexity of $O(nk)$ due to the aforementioned relative adjacency. Stopping \textsc{Quicksort} prematurely reduces the number of required partitioned iterations from $\lg{n}$ to $\lg{n/k}$ (because this is, essentially, sorting $n/k$ unsorted subarrays of size $k$ as singular elements), thereby reducing the complexity to $O(n\lg{n/k})$. Combining these two operations as one simply results in a combined running time of $O(nk + n\lg{n/k})$. \\ \\
For this hybrid algorithm to be successful, \textsc{Insertion-Sort} must be able to out-perform \textsc{Quicksort}; As mentioned in the previous solution, this only occurs when $k < \lg{n}$, so realistically, $k = \lfloor\lg{n}\rfloor$.

\item \textbf{Solution:} The \textsc{Select} algorithm will still work in linear time for groups of 7, or rather, for any odd number of groups $\geq 5$. If groups of 5 are selected, the number of elements greater than the median is as follows:
$$ 3\left(\bigg\lceil\frac{1}{2}\Big\lceil\frac{n}{5}\Big\rceil\bigg\rceil-2\right) \geq \frac{3n}{10}-6$$
Therefore, \textsc{Select} would be called recursively on $7n/10 + 6$ elements, resulting in the following recurrence:
\begin{align*}
T(n) &\leq c\lceil n/5 \rceil + c(7n/10 + 6) + an \\
& = cn/5 + c + 7cn/10 + 6c + an \\
& = 9cn/10 + 7c + an \\
& = cn + (-cn/10 + 7c + an)
\end{align*}
Which will not exceed $cn$ if 
\begin{align*}
0 &\geq -cn/10 + 7c + an \\
c &\geq 10a(n/(n-70))
\end{align*}
Similarly, if groups of 7 are selected, at least half of the groups contribute at least 4 elements greater than the median, so the number of elements greater than the median is as follows:
$$ 4\left(\bigg\lceil\frac{1}{2}\Big\lceil\frac{n}{7}\Big\rceil\bigg\rceil-2\right) \geq \frac{2n}{7}-8$$
Therefore, \textsc{Select} would be called recursively on $5n/7 + 8$ elements, resulting in the following recurrence:
\begin{align*}
T(n) &\leq c\lceil n/7 \rceil + c(5n/7 + 8) + an \\
& = cn/7 + c + 5cn/7 + 8c + an \\
& = 6cn/7 + 9c + an \\
& = cn + (-cn/7 + 9c + an)
\end{align*}
Which will not exceed $cn$ if 
\begin{align*}
0 &\geq -cn/7 + 9c + an \\
c &\geq 7a(n/(n-63))
\end{align*}
For which a suitable $c$ can be chosen, and the algorithm will run in linear time. However, if groups of 3 are selected instead, this does not hold true. In this case, the number of elements greater than the median is as follows:
$$ 2\left(\bigg\lceil\frac{1}{2}\Big\lceil\frac{n}{3}\Big\rceil\bigg\rceil-2\right) \geq \frac{n}{3}-4$$
Therefore, \textsc{Select} would be called recursively on $2n/3 + 4$ elements, resulting in the following recurrence:
\begin{align*}
T(n) &\leq c\lceil n/3 \rceil + c(2n/3 + 4) + an \\
& = cn/3 + c + 2cn/3 + 4c + an \\
& = cn + 5c + an \\ 
& = cn + (5c + an)
\end{align*}
For which $c \geq -an/5$, and there is no linear solution to the recurrence.

\item \textbf{Solution:} The median of a set of $n$ elements is the $\left(\frac{n}{2}\right)$-th order statistic; the $k$ elements closest to this median, therefore, are bounded by the $\left(\frac{n}{2}-\frac{k}{2}\right)$-th order statistic and the $\left(\frac{n}{2}+\frac{k}{2}\right)$-th order statistic, each of which can be found using \textsc{Select} in linear time. Next, the unsorted array $A$ can then be linearly scanned, and the algorithm can return any elements within this bound which are not the median, which is also found in linear time as the $\left(\frac{n}{2}\right)$-th order statistic. 

\item \textbf{Solution:}
It is trivial to use \textsc{Select} to individually determine each of the $i\cdot n/k$-th quantiles for $i=\{1,\dots,(k-1)\}$, however, such a solution would run with $O(nk)$ complexity (that is, using \textsc{Select} results in $O(n)$ complexity for every $k$ quantiles). However, by partitioning the $n$ elements each time a quantile is found, we can instead reduce the number of elements which \textsc{Select} must be called on by half every iteration: 
\begin{itemize}
    \item Iteration 1: We find the $\lfloor k/2*n/k\rfloor$-th order-statistic of the complete array of $n$ elements for a cost of $O(n)$ and partition, repeating the process on each half;
    \item Iteration 2: We find the $\lfloor k/4*n/k\rfloor$-th order-statistic of each array of $n/2$ elements for a total cost of $O(n/2)+O(n/2)=O(n)$ and partition, repeating the process on each half;
    \item Iteration $i$: We find the $\lfloor k/i*n/k\rfloor$-th order-statistic of each array of $n/i$ elements for a total cost of $i\cdot O(n/i)=O(n)$, until $k/i=1$.
\end{itemize}
Since $i$ is growing at a rate of $2i$ every iteration, the recursion tree is binary and has a height of $\lg{k}$; since the cost of each iteration sums to $O(n)$, this results in the desired complexity of $O(n\lg{k})$. 

\end{enumerate}
{\end{document}
