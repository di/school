\documentclass{article}

\usepackage{url}
\usepackage{tikz}
\usepackage{float}
\usepackage{amsmath}
\usepackage{amssymb}
\usepackage{enumitem}

\usetikzlibrary{matrix, shapes, snakes, arrows}
\tikzset{>=triangle 45}

\title{CS 612: Assignment 3\\Spring 2013}

\author{Dustin Ingram}

\date{\today}

\begin{document}

\maketitle

\section*{Written}

\begin{enumerate}

\item{} % 1

\textbf{Ontology for Cooking Utensils:} I developed my ontology based on the
principle that the key differences in kitchen utensils arise from their intended
uses, their materials, and the types of ``ends'' that they have. I further broke
down the materials (e.g., heat-resistant or non-heat-resistant) as well as the
uses as these were the dominant features driving the delineation between utensil
types. The kitchen utensils I used were all considered ``unitaskers'' in the
sense that they had only one intended use. For example, you \emph{could} cut a
piece of meat with a candy thermometer, but you probably wouldn't if you
actually had your wits about you. Furthermore, I considered a utensil's material
to simply be it's most important material (usually that which makes up the
business end), e.g. the material which makes up the handle on a steak knife does
not affect it's usefulness very much.

\item{} % 2

\textbf{Representing sentences using modal logic:} 
Here, I'm considering the $\in$ symbol to represent being ``in'' a place,
whether it is swimming in a pool or sitting in a cabana.
\begin{enumerate}
    \item $K_{Chris}(John \in robots)$
    \item $K_{John}(K_{Chris}(John \in robots)$
    \item $\lozenge robots \neg \in waterproof$
    \item $\square \phi \neg \in waterproof \rightarrow \phi \square \neg \in pool$
    \item $\square \phi \neg \in pool \rightarrow \lozenge \phi \in cabana$ 
\end{enumerate}

\end{enumerate}

\section*{Programming}

I based my rules off of the \texttt{washington.edu} example \texttt{.clp} file.
I adapted it to work in JESS. It can be run by doing:

\begin{verbatim}
$ jess dsi23_cam.clp
\end{verbatim}

I also modified the program to support arbitrary numbers of cannibals and
missionaries. It can be modified by changing the following lines:

\begin{verbatim}
;;;***********************
;;;* ARBITRARY VARIABLES *
;;;***********************

(defglobal ?*n-missionaries* = 3)
(defglobal ?*n-cannibals* = 3)
\end{verbatim}

The output for a 3-missionary 3-cannibal problem is as follows:

\begin{verbatim}
Jess, the Rule Engine for the Java Platform
Copyright (C) 2008 Sandia Corporation
Jess Version 7.1p2 11/5/2008

This copy of Jess will expire in 29 day(s).

Solution found:

Move 1 missionary and 1 cannibal to shore-2.
Move 1 missionary to shore-1.
Move 2 cannibals to shore-2.
Move 1 cannibal to shore-1.
Move 2 missionaries to shore-2.
Move 1 missionary and 1 cannibal to shore-1.
Move 2 missionaries to shore-2.
Move 1 cannibal to shore-1.
Move 2 cannibals to shore-2.
Move 1 missionary to shore-1.
Move 1 missionary and 1 cannibal to shore-2.

Solution found:

Move 1 missionary and 1 cannibal to shore-2.
Move 1 missionary to shore-1.
Move 2 cannibals to shore-2.
Move 1 cannibal to shore-1.
Move 2 missionaries to shore-2.
Move 1 missionary and 1 cannibal to shore-1.
Move 2 missionaries to shore-2.
Move 1 cannibal to shore-1.
Move 2 cannibals to shore-2.
Move 1 cannibal to shore-1.
Move 2 cannibals to shore-2.

Solution found:

Move 2 cannibals to shore-2.
Move 1 cannibal to shore-1.
Move 2 cannibals to shore-2.
Move 1 cannibal to shore-1.
Move 2 missionaries to shore-2.
Move 1 missionary and 1 cannibal to shore-1.
Move 2 missionaries to shore-2.
Move 1 cannibal to shore-1.
Move 2 cannibals to shore-2.
Move 1 missionary to shore-1.
Move 1 missionary and 1 cannibal to shore-2.

Solution found:

Move 2 cannibals to shore-2.
Move 1 cannibal to shore-1.
Move 2 cannibals to shore-2.
Move 1 cannibal to shore-1.
Move 2 missionaries to shore-2.
Move 1 missionary and 1 cannibal to shore-1.
Move 2 missionaries to shore-2.
Move 1 cannibal to shore-1.
Move 2 cannibals to shore-2.
Move 1 cannibal to shore-1.
Move 2 cannibals to shore-2.
\end{verbatim}

\end{document}
