\documentclass{article}

\usepackage{url}
\usepackage{tikz}
\usepackage{float}
\usepackage{amsmath}
\usepackage{enumitem}

\usetikzlibrary{matrix, shapes, snakes, arrows}
\tikzset{>=triangle 45}

\title{\textbf{Simulation of Emergent Behaviors in Coral Spawning Synchrony} \\
CS 612: Final Assignment Proposal \\Summer 2014 }

\author{Dustin Ingram}

\date{\today}

\begin{document}

\maketitle

\section{Problem Statement}
About 75\% of all hermatypic corals ``broadcast spawn'' by releasing
gametes--eggs and sperm--into the water to spread
offspring\footnote{\url{http://en.wikipedia.org/wiki/Coral#Reproduction}}. Over
millions of years of evolution, these ``broadcast'' species have developed an
emergent behavior known as a \emph{synchronous spawning event}: a single night
each year on which all the members of one (or more) species spawn at the same
time.

\section{Hypothesis}
The multi-species synchronization implies that the synchronous spawning event is
not based on any communication between same-species corals, and is in fact due
to other influences within their shared environment.

The main influence on a coral's environment is the moon, through two means:
\begin{itemize}
    \item Light dynamics -- the amount of light the moon sheds on the ocean;
    \item Tide dynamics -- the ``activeness'' of the tide at any given time.
\end{itemize}

The tide dynamics, while they cannot be immediately sensed by the coral, play an
important role in the reproduction of the coral's ``broadcast'': the more active
the tide is, the less time the reproductive cells have to unite before being
washed
away\footnote{\url{http://bio.fsu.edu/~levitan/publication_pdfs/Evolution\%20et\%20al\%202011\%20genetic.pdf}}.

Light dynamics, on the other hand, have been shown to be ``sensed'' by the
coral\footnote{\url{http://micheli.stanford.edu/pdf/effects\%20of\%20light\%20dynamics.pdf}}
and also have a role in protecting the reproductive cells of the coral from
predators: the less light that is available in the environment, the less likely
the reproductive cells are to be consumed by predators before uniting.

\section{Planned Empirical Approach}
The approach will be to simulate the reproductive cycles of multiple coral
species simultaneously. These species will differ by the difference in timing
they choose to reproduce, based on the factors of their environment--namely, the
number of lunar cycles they have experienced, and the current brightness of
their environment.

Coral species will be randomly populated in the world, given random initial
reproduction decision factors.

Coral reproduction will occur when the reproductive cells of two species with
relatively similar reproduction decision factors meet after spawning from their
respective hosts and randomly flowing through the environment. A new species
will evolve from the two, averaging their reproduction decision factors at the
location which they have met, and eventually begin reproducing.

Coral reproduction will be hindered by the brightness level during the
reproduction period (mimicking predators consuming reproductive cells) and the
activeness of the tide (which will shorten the reproduction period).

The goal will be to show that a common behavior will emerge in which one (or
more) species reproduces at the same time.

\section{Metrics}
Interesting metrics to collect will be:
\begin{itemize}
    \item The amount of time (lunar cycles) for a single species to emerge and
    synchronize;
    \item The rate of occurrence of multiple synchronous species after a given
    amount of time;
    \item The factors on which eventually synchronizing coral settle on;
    \item and more!
\end{itemize}

\end{document}
